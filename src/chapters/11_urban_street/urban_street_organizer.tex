\chapter{Event Organizer Rules}

\section{Venue}

\subsection{Slopestyle Street Comp Performing Area \label{sec:flat-street_street-performing-area}}
The course is to be composed of a single ``zone''.

The zone should be large and have a large array of obstacles encouraging a range of specific skills.
The list below is an example of  typical things that can be used for the zones; however designers of the Slopestyle Street comp area should not limit themselves to the exact list.

\begin{itemize}
\item A ramp with a skate park rail in the middle, and a ledge on either side.
This zone will encourage technical grinds, without giving an advantage to a right or left footed grinder.
\item Two different manny pads (a smooth platform of at least 3 m x 0.5 m and between 7 cm and 15 cm in height), one with two revs of length, and one with just one rev of length.
This will encourage the ability to perform technical flip tricks and other Street moves while having to set up quickly for the move down.
\item A set of 5 stairs and a set of 7 stairs with a handrail in the middle of each (that are of a similar size to one that you would find in a city, not extremely steep).
This section would encourage the ability to perform bigger moves of all types.
\end{itemize}

It is also possible to use a real street environment if that is possible.
This may result in having some different obstacles than specified above but it provides a 100\% real street atmosphere.
Independent from the setup a host can go for, they should always take care to offer room for:
\begin{itemize}
\item Technical street
\item Grinds
\item Big tricks off larger drops
\end{itemize}
The descriptions of the zones above should give a good idea about the requirements while offering a lot of room for being creative.

\subsection{Problems With Required Obstacles}
The required obstacles must be built strong enough to endure many hours of heavy use.
They need to survive the competition without changing their shape or stability.
If one of the required obstacles is broken or made unusable during the competition, it must be repaired if one or more competitors say they need to use the damaged part.
If no competitors are impacted by the damage, no repair is necessary except for safety reasons, such as in the event of sharp exposed parts.

\subsection{Postponement due to Weather}

In the case of rain or bad weather and an uncovered Street area, the organizers should postpone the event and exchange all the affected parts of the course for dry ones (replacing pallets for example).
Events should be canceled if considered dangerous.
If postponed or moved to an indoor location, the organizers must try to keep the allowances the same as outdoors competitions with metal pedals and marking tires allowed.
The event host should try to place events that may be influenced by weather conditions in the first days of the event, giving a larger period of time to reschedule it.

\subsection{Music}
In Slopestyle Street, a DJ plays music for the competition.

\section{Officials}

The host must designate the following officials for street:
\begin{itemize}
\item Slopestyle Street Director
\item Chief Judge
\end{itemize}

The host must designate the Slopestyle Street Director well in advance of the event.
For an international events, it is recommended that the Slopestyle Street Director is chosen at least one year in advance so that they may be consulted on scheduling.
The Slopestyle Street Director must select the Chief Judge.
The Chief Judge may be the same person as the Slopestyle Street director.

\section{Communication}

Hosts must publicize details of the available competition area as far in advance of the competition as possible.
Organizers of international championships must publish this information at least three months prior to the event.
For other events, the organizers must specify the venue for the Slopestyle Street competition by the beginning of the convention/competition at the latest.

\section{Categories}
Male and female competitions should be offered in each of the following categories: Junior Expert (0-14), and Expert.
The Advanced category is optional however it is not allowed at Unicon.
If there are less than three Junior Expert competitors, they may choose whether to compete in Expert or Advanced (if offered).
If there are less than three females or less than three males overall, the male and female categories may be merged.

\section{Practice}

Event organizers must arrange that the course for the Slopestyle Street competition is set up and available to be practiced on before competition.
With different time frames depending on the time frame and duration of the convention/competition.
Courses should be completed at least 2 days prior to the day of the competition for events greater than 4 days long.
For logistical reasons, events of 1 - 3 days can provide the required practice time at the discretion of the competition organizers.
This may be provided prior, but must be available on the day of the competition.
If practicing on the competition grounds is not possible prior to the competition day, the organizers must build similar objects on another location for the riders to train on.
