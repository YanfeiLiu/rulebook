\chapter{Judges and Officials Rules}

\section{Slopestyle Street Director}

The Slopestyle Street Director is the head organizer and administrator of Slopestyle Street.
With the convention host, they are responsible for the event logistics, the equipment and the system used to run the event.
They must select the Chief Judge.
They are in charge of keeping the event on schedule and answering questions about the event.
The Slopestyle Street Director is the highest authority on everything to do with the Slopestyle Street competition, except for decisions on rules and results.

\section{Chief Judge}

The Chief Judge is the head Slopestyle Street official.
They are responsible for selecting the judges, overseeing the competition, handling protests, ensuring the event rules are followed, and answering questions about the rules and judging.
The Chief Judge is also responsible for the accuracy of all judging point tabulations and calculations.

An interruption of judging can result from material damage, injury of a competitor, or interference with a competitor by a person or object.
The Chief Judge determines if the rider is at fault.
If they are not, the Chief Judge chooses when to insert the rider back into the competition, along with the rider's remaining time.
The Chief Judge may be the same person as the Slopestyle Street director.

\section{Judges}

\subsection{Judging Panel}

There are three judges per section for the preliminary rounds, and five judges for the finals.

\subsection{Selecting Judges}
A person should not judge an event if he or she is:
\begin{itemize}
\item A parent, child or sibling of a rider competing in the event.
\item A coach, manager, trainer or colleague of the same club/team as a rider competing in the event.
\item More than one judge from the same country on the same judging panel.
\end{itemize}

If the judging pool is too limited by the above criteria, restrictions can be eliminated starting from the bottom of the list and working upward as necessary, but only until enough judges are available.

\subsection{Judging Panel May Not Change}

The individual members of the judging panel must remain the same for an entire category.
In the event of an emergency, this rule can be waived by the Chief Judge.

\subsubsection{Rating Judge Performance}
Judges are rated by comparing their scores to those of other judges at previous competitions.
If a judge's performance is determined to be too weak, they may be removed from the judging panel.

Characteristics of Judging Weaknesses:
\begin{itemize}
\item \textbf{Excessive Ties:}
Using ties frequently (it defeats the purpose of judging.)
\item \textbf{Bias:}
Placing members of certain groups or nations significantly lower or higher than other judges.
\item\textbf{Inconsistence:}
Ranking a large number of riders significantly different from the average of other judges.
\end{itemize}

\subsection{Training}
The judges workshop is set by the Slopestyle Street Director or Chief Judge.
Either the Slopestyle Street Director or the Chief Judge run the workshop.
The workshop must be help before the competition.
Judges should have read the rules prior to the start of the workshop.
The workshop will include a practice session.
Each judge will read the rules, attend the workshop, agree to follow the rules and agree to their potential removal from the list of available judges if they show excessive judges weaknesses, as determined by the Chief Judge.

\section{Slopestyle Street Comp Judging}
After prelims, the highest scoring 5-8 competitors will move on to the finals.
In finals, the rider with the most points is the winner.

\subsection{Timed Runs}
For each judging category, judges will assign 0 to 10 points. These categories are combined with the following weights to create a run score out of 100 points:\\
Difficulty 35\% \\
Consistency 20\% \\
Variety 25\% \\
Flow 20\% \\

\subsection{Best Tricks}
For each judging category, judges will assign 0 to 10 points. These categories are combined with the following weights to create a run score out of 100 points:\\
Difficulty 80\% \\
Style 20\% \\

\subsection{Preliminary Round}
Timed Runs only.
If two runs are made, the highest scoring run is counted.
If three runs are made, the combination of the two highest scoring runs is counted.

\subsection{Final Round}
Competitors are judged on Timed Runs and Best Trick, with 70% of the final score based on Timed Runs and 30% based on Best Trick.
The final score of the Timed Runs is based on the combination of the each competitor's two highest scoring runs.

****
\section{Sportsmanship}
If a rider distracts or delays the competition or shows unsportsmanlike conduct, the Chief Judge may choose to warn or eliminate that rider.
