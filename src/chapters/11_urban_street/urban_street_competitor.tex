\chapter{Competitor Rules}

\section{Safety}

Riders must wear shoes and helmet.
Shin guards are recommended, but not mandatory.

\section{Unicycles}

Any number of regular unicycles (see definitions in chapter \ref{chap:general_definitions}) may be used.

\section{Rider Identification}

No rider identification is required.

\section{Protests}

Protests must be filed on an official form within 15 minutes of the posting of event results.
Protest is only possible against mistakes in calculation or other mistakes not connected to a judge's subjective score.
The Chief Judge must resolve all protests within 30 minutes of receipt of the written form.

\section{Results}
Preliminary results will be posted once the calculation of the points is done.
Results sheets will be posted after each category of an event.
The protest periods begins at this point.
Finals results may be held off until the award ceremony if it is on the same day.

\section{Event Flow}

\subsection{Riders Must Be Ready}
The Chief Judge chooses how to handle the riders who are not ready at their scheduled competition time.
They may be disqualified or allowed to perform after the last competitor in their category

\subsection{Timed Runs}
Timed runs are completed anywhere on the course and last 40-60 seconds at the director's discretion.
Remounting is allowed in the event of a failed trick.
Timed Runs will begin at a countdown or signal from an official and cease at the end of the allotted time.

\subsection{Best Trick}
During finals, Best Tricks are performed anywhere on the course.
Best Tricks are performed after all competitors' Timed Runs have been completed.
Competitors attempt Best Tricks in the same order as Timed Runs.
Each rider makes all of their Best Trick attempts consecutively, and does not have to use all of their attempts, and can attempt a different trick on each attempt.
A Best Trick must be a single trick completed on the course.

\section{Preliminaries}
All competitors will be placed in an order.
The order should be presented in writing as well as announced before the competition.
Competitors will each complete two or three timed runs.
The number of runs will be chosen by the director before the competition.
Riders may choose to skip their turn in the event of an injury or any other reason.
Each competitor will complete their first run in order, before repeating the order again for each consecutive run.

\section{Finals}
The top 5-8 competitors will compete in a Finals round held at least 3 hours after the preliminary rounds.
To ensure riders have adequate warm up time and to increase spectator numbers the final round should preferably be held in the afternoon/evening.
The existing course from preliminary rounds may be used, or changes may be made to the course.
The riders will be assigned an order which should be presented in writing as well as announced before the competition.
Competitors will each complete four timed runs.
Each competitor will complete their first run in order, before repeating the order again for each consecutive run.

After completion of the timed runs, each competitor will have three attempts at a Best Trick.
Riders may choose to skip their turn for any reason.
Best Trick score is based on the sum of the each competitor's 2 highest scoring attempts.
Judges can include preliminary round judges as well as riders that did not make the finals round.
