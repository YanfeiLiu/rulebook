\chapter{General}

\section{Scope of Rules}

\begin{comment2016}
Mary and I think that this section could be clarified by splitting out scope, use for national events, how to get approval for unicon qualification, and how the rules can be modified.
\end{comment2016}

This rulebook is intended to govern all unicycle competition sanctioned by the International Unicycling Federation, and can be used as a guideline for other competitions.

\subsection{Unicon}

All IUF Unicons (International Unicycling Conventions) must abide exclusively by these rules.
Further rules may be added to cover specific situations, but they may not override the IUF rules without prior approval by the IUF Board of Directors.
All additional rules must be published well in advance of international competition, and published together with registration forms.

\subsection{Other Uses for These Rules}

National or local unicycling organizations may have their own rules.
Though they may use IUF rules as a basis for their own rules, in national or local competitions, those rules can no longer be called IUF rules.
To get proper results for Unicon qualification it is needed to follow the IUF Rulebook as described above.

Any national organization that wishes to get its modified rules approved by the IUF for a national competition must submit a proposal to the IUF Executive Board at least 90 days before the start of the given event.
If approved, the national competition can still be recognized as an official IUF event.

To host an ``official IUF event'' means that the results of this event are comparable with results from other official IUF events and can count for possible qualification restrictions.
Rules, which are approved for use at a national or local competition by the IUF Board to be used, must not be referred to as IUF rules to prevent confusion for the riders.

\section{How to Use This Rulebook}

This IUF Rulebook is organized by discipline.
Each chapter includes a overview, competitor rules, officials rules, and organizer rules.
Additionally, Chapter 1 covers general rules, not tied to a particular event.

How you approach the information in this rulebook will depend on what role you play.
The information has been organized with a goal toward separating rules by role, so that you can safely ignore sections that do not pertain to you.
For example, if you are only interested in muni as a competitor, you can ignore other chapters and you can mostly ignore the officials and organizer rules within the muni chapter.

The following table gives some guidance for what information you need for various roles:

\begin{longtable}{|l|p{8cm}|}
\hline
\textbf{Role} & \textbf{Applicable Sections} \\
\hline
Competitor & Chapter 1 and the overview and competitor rules for your events\\
\hline
Official & Chapter 1 and the overview, competitor rules, and officials rules for the event you will officiate\\
\hline
Event director & Chapter 1 and all sections of the chapter for your event \\
\hline
Host & All sections of the rulebook\\
\hline
\end{longtable}

\section{How the Rulebook is Updated}

This publication should be updated after every Unicon.
The IUF Rulebook Chair will head the committee, but may optionally name a sub-committee.
The Rulebook Committee will officially start meeting at the close of the Unicon, though the Chairperson can open it before, to take advantage of having so many persons physically together.
The Committee should finish their business and make their specific proposals within three (3) months of the close of the Unicon.
If they need more time, they may ask the IUF President for a time extension.
This is meant to be the only time that changes to the Rulebook are made, although exceptions are possible in extraordinary cases.
The IUF President is responsible for making sure that the Rulebook committee stays focused and on schedule.

Anyone may submit a potential change to IUF Rulebook at any time.
These will not be official proposals, but suggestions for potential topics during the next Rulebook session.
A forum will also be provided to discuss potential changes throughout the year.
The Rulebook Committee voting time frame and official members of the Rulebook Committee, however, will still be determined by the IUF Rulebook Committee Chair and the IUF Executive Board.

\section{World Champions}

The Male and Female winners of each individual event at Unicon are the World Champions for that event.
There is no age limit to winning the overall title.

Age group winners can use the title `Age-Group Winner', and the term `World Champions' generally refers to winners of Overall, Finals or Expert class.

Winners in the Expert category at of each event at Unicon are the World Champions.
In the individual events, separate titles are awarded for male and female.
Winners in the Jr.~Expert category at Unicon are the Junior World Champions.

The male/female winner of the Expert category at Unicon is the Male/Female World Champion.
The male/female winner of the Junior Expert category at Unicon is the Male/Female Junior World Champion.
In the absence of any of these categories, no title will be awarded.
No title is awarded for the Advanced category.

\section{World Records, IUF Records}

The host should ensure that the competition conditions are conducted and recorded according to the IUF Rulebook and the IUF World Record standards.
If world record standards cannot be used, then the competition results cannot be used for new world records.

\section{Ownership of Data}

Each Unicon or other large unicycling convention is a piece of history.
At the conclusion of a Unicon or other international event, or within one month thereafter, the convention host must supply the IUF with a list of competition and other results.
This list will include all data collected to determine placement and winners at all levels and in all events held at the convention.
This data is considered public, and is not the sole property of the host.
Copies of attendee registration details, judging sheets, protest forms, and related paperwork are not necessarily public, but are the shared property of the host and the International Unicycling Federation, and must be made available upon request.
If the host wishes to discard any of this paperwork or data, it must be turned over to the IUF, not thrown away.
If requested, the host and convention officials must also provide further information, not necessarily in writing, about decisions made, methods used, and other details covered in the process of planning and running the convention.
This information can be invaluable to future hosts, and must not be hidden or lost.

\section{Hierarchy of Officials}

These people make the competition events work.
All of the tasks detailed below must be covered for the events to work.
Names must be assigned for all the jobs listed below, to create a hierarchy of authority for the convention.
All officials are expected to work objectively and impartially.

\subsection{General Officials}

\textbf{IUF Board Of Directors:} The IUF Board represents the interests of the IUF on convention requirements, both in the area of competition rules and the necessary spaces and facilities for them, and for any other requirements that go along with putting on an IUF convention.
If problems arise in meeting the IUF requirements, the IUF Board and Convention Host work together to find solutions or compromises.
The bulk of this should happen during the early planning stages for a convention, when facilities and schedules are being assembled.

\textbf{IUF Convention Liaison:} The Liaison is an optional person who can represent the IUF Board when communicating with convention hosts.
The Liaison essentially has the same powers as the IUF Board, but must report to the IUF Board and take direction from it.

\textbf{Convention Host:} This is a single person, or a collective group, that has made the commitment to host a unicycle convention using IUF rules and guidelines.
By agreeing to host an IUF convention, they also agree to follow those rules and guidelines wherever possible.
If known problems arise in the arrangement of facilities, schedules and events, the Host and the IUF will work together to resolve the problems.
For the most part, the Convention Host is the ultimate authority for what happens, and does not happen, at the convention.
The exception is any IUF requirements for convention facilities or contents, and rules for IUF competition events.

\textbf{Top Competition Officials:} Each discipline has a set of authority positions which may include Race or Event Director, (Chief) Referee, Artistic Director and/or Chief Judge.
They are not autonomous, and must answer to the Convention Host.
For Artistic freestyle, it is highly recommended that none of these jobs be combined, and that there be at least one separate person for each.
For other disciplines, combining these roles may be the easy way to organize those events.

\begin{comment2016}
The top officials needs to be reworded to include other events.
\end{comment2016}

\chapter{Competitor Information}

\section{Personal Responsibility}
All attendees should remember that they are guests of the convention hosts, and ambassadors of our sport to all new riders, visitors from far away, and to people in the hosting town.
Remember that the Host is renting the convention facilities, and attendees are expected to treat them well.
Each rider is responsible for the actions of his or her family and non-riding teammates.
Riders may lose placement in races, risk disqualification from events, or be ejected from the convention if they do not work to minimize disruptions from these people.

\section{Supervision of Minors}
A parent, guardian or other designated person, must supervise all minors.

\section{Knowing The Rules}
Lack of understanding of rules will be at the disadvantage of riders, not officials or the IUF.
The IUF is also not responsible for any errors that may occur in the translation of rules and information into languages other than those in which they were originally written.

\section{Your Privilege}
Entry in the competition is your privilege, not your right.
You are a guest at the Host's event.
You may be in an unfamiliar country, with different customs that are considered the norm.
The Host and convention officials determine whether certain events, age groups, or policies will be used.
As an attendee, you are obligated to obey all rules and decisions of convention officials and hosts.

\section{Nations Represented}
For events where the number of participants is limited by country, there may be some question of what country a rider, pair or group may represent.
Riders must represent the country in which they hold citizenship, or in which they are a legal resident.
For example, if a rider is attending school in a different country, and is in that country legally, the rider can either represent that country or the rider's home country.

If necessary, citizenship or residence may be established with a passport, driver's license, or legal ID for the country the rider wishes to represent.
Riders on extended vacation, exchange students, and other temporary residents of other countries are not eligible to represent those countries, except in multi-rider events (see below).

For Pairs Freestyle or other two-person events, the pair can represent any country that either rider is eligible to represent.

For Group Freestyle, sports teams or other multi-rider events, the group must represent the country that the greatest number of the group's riders is eligible to represent.
If there is a tie in this number, the group can represent either of the tied countries.

\chapter{Host Information}

\section{Convention Aspect}

All competitions at a Unicon need to make every effort to have equal time for the convention side of Unicon by involving as many competitors as possible and making the event spectator-friendly for other Unicon participants as well as non-unicyclists.
Any of the following are examples to achieve this goal:
\begin{itemize}
  \item Workshops related to the event
  \item Fun competitions based on the event
  \item Instant results for the spectators
  \item Ways for other competitors to be introduced to the event
  \item Entertainment during breaks in the competition (such as half time entertainment)
  \item Schedule of the events posted in multiple places
\end{itemize}

\section{Required Events at Unicon}%comment2016 this section is really outdated

Unicon should include at least one event from each of the following event groups.
Hosts are free to add events, age groups or variations that do not appear here, as long as there is no conflict with the existing rules.
When in doubt contact the IUF Rules Committee.
\begin{itemize}
  \item Track Racing: the required races from section \ref{sec:track-field_minimum-racing-events}.
  \item Other Racing: Road, specialty/novelty races; see chapters \ref{part:track-other} and \ref{part:road-racing}.
  \item Team Games: Unicycle Hockey, Unicycle Basketball; see chapters \ref{part:hockey} and \ref{part:basketball}.
  \item Field events: Long Jump, High Jump, Gliding/Coasting; see chapter \ref{part:jumps} and chapter \ref{part:track-other}.
  \item Non-competition events: workshops, fun games, sightseeing rides, muni rides.
  \item Artistic events: Freestyle, Standard Skill, Flatland, Street; see chapters \ref{part:freestyle}, \ref{part:standard}, \ref{part:flat}, and \ref{part:street}.
  \item Muni: Cross Country, Orienteering, Uphill, Downhill, Trials; see chapter \ref{part:muni} and chapter \ref{part:trials}.
\end{itemize}

\section{Sponsors}

The convention host has the option to seek and obtain private sector sponsorship; for example The Unicycle.com Freestyle Awards, the Coca-Cola Hockey Cup, etc.
This will allow opportunities for external funding to defray costs for host organizations and competitors.
Sponsors are limited to organizations that would not bring the IUF into disrepute and are consistent with the aims and objectives of the International Unicycling Federation, Inc.

\section{Materials \& Equipment}
The Host must supply all necessary materials and equipment to run the competitions, such as a timing system, starting posts, cones for the IUF Slalom, etc.
Other materials such as paper and writing materials, judging tables, printers, basketballs, hockey sticks, etc.\ are also necessary.

\section{Early Announcement of Rule Changes}
For international competitions, written rules are needed for any planned events not described in the IUF Rulebook, and for events where additional rules are required.
These special rules could be variations on the optional events found in this Rulebook.
Such rules should be published at the same time as registration forms, or earlier, and must be published at least one month before the start of the event.
These rules can be published along with registration forms, and/or on the convention web site.
Competitors need to know the specific rules so they can train for those specific events!
Hosts also need to decide on rules early, so there is less to worry about near competition time.
Rule changes may be a necessary reality, for reasons such as changes in venue, weather or available equipment.
When this happens these changes must be posted to the convention web site immediately.
Examples: Dismount rules or timing details for off-track races, obstacle information for Street Comp, planned age divisions or combination awards.

If competition events or games not found in the IUF Rulebook are planned, written rules must be provided.
These rules, if not pre-existing, should be published at the time of announcement of those events.
This generally means at or before the posting of registration forms.
For competitors to properly train, and be on an equal footing with local riders, all must be aware of the rules to be used.

\section{Registration Forms}
Because of the various options available to riders in different events, riders may enter different events in different age groups.
A properly structured registration form is essential for making these choices clear to the participants.
For example, a rider may enter Pairs as an Expert with an older rider, but may wish to compete in Individual Freestyle in his or her own age group.
Before publishing, a Unicon registration form should be examined and approved by members of the IUF Rules Committee or Board of Directors.
No rider may enter any event until his or her registration form has been completed, including payment and completion of waivers and/or signatures.
No minor may compete until a parent or legal guardian has signed his or her release.

\section{Combining Age Groups \label{subsec:general_host's-option-unicon_combining-age-groups}}

In a competition with more than 50 riders, six riders are needed to complete an age group.
In competitions with less than 50 riders, six in each age group are still highly recommended, however three riders are the minimum to complete an age group.
Riders generally enter all events with their age group except for events similar to artistic competitions where there are Junior Expert and Expert categories.

The convention host must combine age groups with less than six riders (three riders for smaller conventions) if needed.
This means that published age groups are not guaranteed.
This can be done on a per-event (= per-discipline) basis.

When combining, combine the smallest age group (that is, the age group with the smallest number of participants) with its smallest neighboring age group (either up or down).
If more than one age group is the smallest, choose the age group with the smallest neighbor for combining.
Continue this process until all resulting age groups (combined and/or original) have at least the minimum required/recommended number of participants.
Male age groups are never combined with female age groups.

\section{Awards}

Because awards are paid for out of the convention budget, the type, number, and quality of awards are the choice of the convention host.
However, these awards must abide to these stipulations:
\begin{itemize}
\item Male and female competitors must be awarded equitably
\item Standard Class and Unlimited Class competitors must be awarded equitably within the same competition (i.e.\ 10k Standard and 10k Unlimited)
\item Awards should be equal within the ranks of Teams, Age Groups, and Champions (i.e.\ 0-14 in Freestyle must be awarded equitably to 16-17 Muni Cross Country)
\end{itemize}

Generally there are trophies for Champions or for 1st-3rd places in finals, medals for 1st-3rd places in each Age Group for each event, and optionally ribbons or certificates for lower places.
The IUF has most frequently awarded 1st-3rd place in most events, but this too is up to the convention host.
Once the competition has finished, a personal certificate must be made available to each competitor with a summary of their complete results.
This can be done as an online download, and/or sent through e-mail, and/or made available as a physical copy on the event itself.
The design of the certificates is up to the convention host.

\section{Safety Equipment}

Safety equipment worn by riders must meet the definitions for each, which are found in chapter \ref{chap:general_definitions}.
Hosts may only deviate from these rules for safety equipment if this is inevitable.
The status of ``inevitable'' has to be documented and must be approved by the IUF executive board.
Any deviation from the IUF safety equipment requirements must be approved and announced at least two months before the event.
Additional inevitable changes that arise just before or during an event cannot be approved by the event director alone.
The approval of two IUF representatives is required in addition to the event director's approval.
These changes are once again only allowed in the case of the inevitable, and not, for example, due to the wishes of the competitors or judges.

\section{Protests}
An official protest/correction form must be available to riders at all times.
All protests against any results must be submitted in writing on the proper form within two hours after the results are posted, unless there is a shorter time specified for certain events (for example: track racing).
The form must be filled in completely.
This time may be extended for riders who have to be in other races/events during that time period.
Every effort will be made for all protests to be handled within 30 minutes from the time they are received.
Mistakes in paperwork and interference from other riders or other sources are all grounds for protests.
Protests handed in after awards have been delivered will not be considered if the results have been posted for at least three hours before the awards.
If awards are delivered before results are posted, it is recommended to announce the schedule of posting and the deadline for protests at the awarding ceremonies.
All Chief Judge or Referee decisions are final, and cannot be protested.

The host may decide to make official video of some competitions, for example at the start line and/or the finish line, or the 5-meter-line in case of the 50~m one-foot race.
This must be announced before the competition to let the competitors know about their option to protest through this video.

Regardless of whether official video is available, all possible sources of evidence are generally allowed as a means of verification in case of a protest, including (but not limited to) private photos/videos and eyewitness reports.
If someone submits a protest and has evidence that he wants to be considered, he must state that with his protest.
If possible, it is recommended that digital material is copied onto an `official' computer for analysis.
As an alternative, the evidence must be readily accessible, e.g.\ through a contact person and phone number.

In case of video evidence (regardless of its origin), a referee without good skills in video analysis should ask for a skilled assistant in order to prevent incorrect interpretations.

The referee decides which evidence he will consider, and the `value' he assigns to the various pieces of information.
Generally, official camera footage and judge reports will have higher `value' than private evidence.
The objective is that all riders will be judged as fairly as possible.

\section{Open Practice Area}
For Unicon and other large competitions, at least one area with a smooth safe riding surface, sheltered from the weather, must be made available for all or part of the day on most or all days of the convention.
These areas are to be used for non-competition events such as workshops, skills exchange and free practice.

\section{Program Book}

At Unicons, all registrants shall be provided with a package of pre-printed information containing a full schedule of all events, maps and directions to all event locations, and as much rule and background information as possible.
This information shall be provided when registrants first check in at Unicon.
Unicon organizers should consider placing as much of this information as is practical in an official Program Book.
This can make excellent reading for family members and spectators, and gets them more involved in our sport.
It's also a great place to sell ads as a source for convention revenue.
At other unicycling events, it is recommended that pre-printed information be provided to all participants.

\section{Availability of Rulebook}
The host must make sure there are plenty of copies of the rulebook for officials to study on the spot.

\section{Photography and Videography}
The following rules are required for Unicon and are highly recommended for other large international competitions.

In events with closed perimeters, it may be necessary to limit the number of photographers and filmers (hereafter called ``shooters'') allowed inside.
We want great documentation of the events, but not at the expense of safety, and of spectators' ability to see as well.

The following guidelines apply:
\begin{itemize}
\item Shooters must either register ahead of time to be inside the perimeter of an event, or have actual press credentials (professional photojournalists, TV news people, etc.).
\item Registered shooters must have some form of ID given to them, whether it be a pass on a lanyard, a volunteer shirt, or something else to help identify them.
\item The Referee or Head Official for the event has the final say on shooting that can affect the riders and/or spectators' view.
\item The Referee or Head Official should appoint a Media Manager to manage this task.
\item If a Media Manager is used, that person is still under the authority of the Referee or Head Official of the competition.
\item Media Managers must have a good understanding of the needs of shooters to get the job done.
\item Shooters must follow the instructions of the Media Manager or Referee/Head Official, and of the officials at the location.
\item Shooters must generally stay aware at all times of the movements around them.
\item If shooters continue to get in the way and/or not follow instructions they are to be ejected from the perimeter.
\item It is greatly appreciated, but not required, that the shooter submits his or her top shots to the Media Manager during or directly following the convention to be used for the press.
\item Flash is never allowed unless specific permission is given by the event director.
\end{itemize}

\section{Publication of Convention Information}

Convention dates and other information must be announced and/or published at the earliest possible date.
The best way to control the publication of convention information is with a convention web site, with regular updates to provide all the latest information.
For Unicon and other large events, registration forms should be made available no less than eight months before the convention start date.
A list of all planned competition events, including all rules and information pertinent to quality training, should be published at the same time with newly available data to be added as soon as it is known.
Wherever possible, hosts should provide maps, directions and other information to help make people's convention as enjoyable as possible.

\section{Publishing Results \label{sec:publishing_results}}
Results of national and international championships must be published including details such as time, distance, and total score.
For each event, the names and represented nationality of competitors as well as the names and nationality of all officials shall be published.
In the artistic events, countries and names of the entire judging panel must be published. %comment2017 this should be in artistic chapter

\section{Option to Remove People From Events}
The host is allowed to remove an individual or a group if they are acting aggressively or abusively against others.
These individuals/groups should be given a first warning, followed by removal from the specific event by the Host or the Chief Judge/Referee who is in charge for the competition where the problem appears.
The person(s) should only be removed from that competition to have a chance to calm down.
If the aggressive or abusive behavior continues, it is also possible to remove the individual or group from the rest of the convention.

\section{IUF Public Meeting}

The host will provide time in the convention schedule for the IUF Public Meeting.
At this meeting, the IUF will elect officers or other volunteers, and otherwise do business and encourage the opinions and assistance of all interested convention attendees.

The meeting time should be as close to the end of the convention as possible, excepting on the final day, as people may have to leave before that time.
At minimum, the meeting should be during the second half of the convention.

A minimum of two hours should be allocated, during which no other official convention events, other than open gym or other informal activities, should take place.

A meeting room must be provided that has adequate space/seating, lighting and acoustical properties to communicate and conduct the meeting.
A lecture hall or theater are optimal locations, and a sound and/or projection system would be very helpful.

Other IUF meetings may be held during the convention, both public and private, but the strict requirements apply only to the big public meeting.

\section{Changes and Cancellations}

The host reserves the right to make changes, if necessary, to ensure the success of a convention or competition.
Sometimes these changes must be made at the last minute, such as in switching outdoor events for indoor in the event of rain.
Sometimes activities must be cancelled due to events beyond the host's control, such as weather or power outages.
When changes or cancellations are made, notification must be posted, communicated and/or distributed as early as possible.

\chapter{Terminology}

Event hosts must learn and use the proper names and terminology for our sport and competition events.
They should take care not to continue the misuse of outdated or incorrect names and terminology.
The correct ones must be used in all announcements, advertising, publicizing, internal and external documents, and especially in any official documents, such as those within, and printed out by, convention software.
For example, the specific artistic event names are Individual Freestyle, Pairs Freestyle, Group Freestyle, Flatland, Street Comp, and Standard Skill.
Note that the word Artistic is not part of any of the individual event names.

While we call our event ``Unicon'' (Unicycling Convention), remember this word is unfamiliar to the general public.
Remember to spell out the full name of your event so people know what it's about.
If it doesn't say unicycle or unicycling, the general public may not know what your event is about.

\section{Definitions \label{chap:general_definitions}}

\textbf{Age:} Rider's age for all age categories is determined by their age on the first day of the convention.

\textbf{Expert:} The top category in events that don't have a system to determine Finalists.
When no other limitations are present, riders can choose to compete in this category against the other top riders.
Limitations on this may be if top riders are chosen at previous competitions, such as national events, or if there is a limit on the number of competitors per country.
The category is called Expert, and riders entered in it can be called Experts.
The distinction of Experts over Finalists is that they are not chosen based on competition results at the current convention.

\textbf{Figure:} (noun) 1.~A unicycle feat or skill, such as walking the wheel or riding backward, used to describe skills in the Standard Skill event.
2.~A riding pattern, such as a circle or figure 8.

\textbf{Finalist, Finals:} A Finalist is a person, and ``the Finals'' is the last category or group in any event that has multiple rounds.
For example in Track racing, the top riders from the age groups compete against each other in the Finals of most events.

\textbf{Freewheel:} Mechanism allowing the wheel to rotate while the cranks are stationary.

\textbf{Gearing:} Any mechanism that transfers the rotation speed of crank arms to a different rotation speed of wheel.

\textbf{Gloves:} (For racing) Any glove with thick material covering the palms (Leather is acceptable, thin nylon is not).
Gloves may be fingerless, such as bicycling gloves, provided the palm of the hand is completely covered.
Wrist guards, such as those used with in-line skates, are an acceptable alternative to gloves.

\textbf{Helmet:} Helmets must be of bicycle quality (or stronger), and should meet the prevalent safety standards for bicycle (or unicycle) helmets, such as ASTM, SNELL, CPSC, or whatever prevails in the host country.
Helmets for sports other than cycling or skating are not permitted, unless the Referee makes exceptions.
Helmets are required for some events as described in the Safety section of each chapter.

\textbf{IUF:} International Unicycling Federation.
The IUF sponsors and oversees international competitions such as Unicon, creates rules for international competition, and promotes and provides information on unicycling in general.

\textbf{Junior Expert:} Same as Expert, but open only to riders age 0-14.
Riders in this age range may optionally enter Expert instead, to compete in the highest/hardest category.

\textbf{Knee pads:} Any commercially made, thick version is acceptable, such as those used for basketball and volleyball, or any with hard plastic caps.
Knee pads must cover the entire knee and stay on during the whole length of the competition.
Long pants, bandages or patches are not acceptable substitutes.

\textbf{Muni:} Mountain unicycling, or mountain unicycle.
The previous term for this was UMX.

\textbf{Non-unicycling Skills:} (for Freestyle judging) The riding of any vehicle with two or more wheels on the ground, and any skills not performed on a unicycle.
Any skill with more than one support point on the riding surface, such as standing on the unicycle with it lying on the floor, or hopping while standing on the frame (seat on floor); two contact points with the riding surface (wheel and seat), both carrying part of the rider's weight.
The term also refers to skills such as dance, mime, comedy, juggling, playing music or riding vehicles that do not meet the definitions of unicycles.

\textbf{Prop:} Almost anything other than the unicycle(s) being ridden by competitor(s) in a Freestyle performance.
A unicycle being used for a non-unicycling skill (such as a handstand on it while it's lying down) is a prop at that moment.
A hat that is dropped and picked up from the floor is a prop.
A pogo stick or a tricycle (unless ridden on one wheel) is a prop.

\textbf{Shoes:} Shoes with full uppers are required.
This means the shoe must cover the entire top of the foot.
Sandals or thongs are not acceptable.
Shoelaces must not dangle where they can catch in crank arms.

\textbf{Shin guards:} Any commercially made, thick version is acceptable, such as those used for football or bicycling, or any with hard plastic shell.
Shin guards must cover the shin and stay on during the whole length of the competition.
Long pants, bandages or patches are not acceptable as substitutes.

\textbf{Unicycle, Standard:} A Standard Unicycle has only one wheel.
It is driven by crank arms directly attached to the wheel's axle/hub, with no gearing or additional drive system.
Pedals and cranks rotate to power the wheel.
Is balanced and controlled by the rider only, with no additional support devices.
Brakes and extended handles/handlebars are permitted.
For some events, such as track racing, standard unicycles have restrictions on wheel size and/or crank arm length.
Other events may specify other restrictions.
When not noted otherwise, there are no size limitations.

\textbf{Unicycle, Unlimited:} A Unlimited Unicycle is powered, balanced and controlled by the rider only.
Gearing, shiftable or not, and/or freewheel are allowed.
(This may also be referred to as `Transmission.')
Multiple wheels are permitted, but it must not be possible to ride the unicycle when more than one wheel touches the ground.

\textbf{Unicycle, Wheel Size Classes: \label{def:general_terminology_wheel-size-classes}}
The IUF defines standardized wheel sizes classes for unicycling competitions. Each class can have a limit on the maximum allowable outer wheel diameter (maximum diameter), the minimum allowable crank arm length (min crank length), and allowable transmission system, as defined above in the Standard and Unlimited Unicycle definitions.

\begin{longtable}{|p{3cm}|p{3cm}|p{4cm}|p{3cm}|}
\hline
\textbf{Unicycle Class} & \textbf{Max Diameter} & \textbf{Min Crank Length} & \textbf{Transmission}\\
\hline
16 Class & 418mm & 89mm & standard \\
\hline
20 Class & 518mm & 100mm & standard \\
\hline
24 Class & 618mm & 125mm & standard \\
\hline
24+ Class & 640mm & No limit & standard \\
\hline
29 Class & 778mm & No limit & standard \\
\hline
Unlimited Class & No limit & No limit & unlimited \\
\hline
\end{longtable}

For any tire in question, its outside diameter must be accurately measured.

Crank arm length is measured from the center of the wheel axle to the center of the pedal axle.
Longer sizes may be used.

The maximum diameter for the 24+ Class and 29 Class are defined such that virtually any commercially available tire, 24 inch or 29 inch respectively, should fit under these limits. However, this is not guarenteed and the referee should still be aware of the limits.

\textbf{Ultimate wheel:} A special unicycle consisting of only a wheel and pedals, with no frame or seat.

\textbf{UMX:} Unicycle Motocross.
This term has been replaced by muni.

\textbf{Unicycling skill:} (noun, for Freestyle judging) Also known as `figure.' Any skills (feats of balance) performed on a vehicle with one support point in contact with the riding surface, this being a wheel, the movement of which is controlled by the rider, thus maintaining balance.
All mounts are also `unicycling skills.'

\textbf{Unintentional dismount:} In most cases, any part of a rider unintentionally touching the ground.
A pedal and foot touching the ground in a sharp turn is not a dismount as long as the foot stays on the pedal while the pedal is on the ground.
Dismounts during many races disqualify the rider.

\textbf{Unicon:} Unicycling Convention.
This word usually refers to the IUF World Unicycling Championships conventions.

\textbf{Wheel walking:} Propelling the unicycle by pushing the top of the tire with one or both feet.
Feet touch wheel only, not pedals or crank arms.
A non-pushing foot may rest on the fork.
