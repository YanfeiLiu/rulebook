\chapter{Competitor Rules}

\section{Unicycles}

Only standard unicycles may be used.
The maximum outer diameter of the wheel is 640\,mm (24+ Class) as defined in section \ref{def:general_terminology_wheel-size-classes}.
In addition, the unicycles must not have sharp or protruding parts anywhere that might cause injuries.
This refers especially to quick-release levers and bolts.
The pedals must be plastic or rubber.

\section{Rider Identification}

All players of a team must wear shirts of the same color.
The color must be clearly different from the opponent's color.
At tournaments and other large events each team should have two different colored sets of shirts.

\section{Protests}

Protests must be filed on an official form within two hours of the posting of event results.
Every effort will be made for all protests to be handled within 30 minutes from the time they are received.

\section{Sticks}

All sticks legal for playing ice-hockey or floorball (apart from those for the goalkeeper) can be used.
Cracked or splintered sticks must be taped or repaired before play.
An upper end made of rubber is recommended.

\section{Number Of Players}

A team on the field consists of up to five players with a team requiring a minimum of three players to begin a match.
Player substitutions are possible at any time with the substituting player entering the field at the same location only after the other has completely left it.
It is not necessary to indicate substitutions to the referee.
Each player can be the goalkeeper at any time.
The goalkeeper has no special rights.

\section{Penalties}

In every instance of a violation of the rules the referee must penalize the offending team or play the advantage.
When playing the advantage the referee does not blow the whistle but should display the hand sign for a free shot and shout “Advantage!”
In the event that an advantage was not gained, the referee should enforce the appropriate penalty from the initial point of infringement or, when the penalty has occurred within the goal area, the closest corner mark or 1\,m in front of goal line.
Additionally, at the referee's discretion, offending players may be sent off after advantage has been played.
The referee should not enforce this penalty until the offending team gains possession of the ball and should resume the game with a face off at the point of possession change.
When two or more players fall and/or it is unclear whether a foul occurred, the referees can interrupt the game and restart it with a face-off.


\subsection{Free Shot}

The free shot is the standard penalty for all violations of the rules.
It is applied in all cases except for those explicitly mentioned in sections \ref{subsec:hockey_penalties_65m}-\ref{subsec:hockey_penalties_face-off}.
The free shot is executed from the point where the violation was done.
Exceptions: If a team receives a free shot within the opponents' goal area, the free shot is executed at the closest corner mark (corner shot).
If a team receives a free shot within their own goal area, the free shot is taken at a distance of 1\,m in front of the goal line (goalkeeper's ball).
In the instance that a delay of game penalty is given, the penalty will be taken from the center mark.

The free shot is indirect.
The player executing the free shot may only touch the ball once until a contact by another player occurs.
The ball shall be hit with the stick, not dragged, flicked or lifted on the stick.
Opposing players must keep a distance with their unicycles and their sticks of at least 2.0\,m from the ball.

\subsection{6.5\,M \label{subsec:hockey_penalties_65m}}

If legal play would have led to a direct chance to score a goal, a ``6.5\,m'' is given.
The following situations are a prevention of a direct chance to score and should be punished with a 6.5\,m penalty:
\begin{itemize}
\item An attacking player is fouled in the opposition goal area while in a strong position to score.
\item An attacking player is fouled when moving towards the opposition goal with a single opponent in front.
\end{itemize}
The ball is placed at the 6.5\,m mark.
A player of the defending team goes to the goal and must sit with the bottom of the wheel of their unicycle within 0.5\,m of the goal line.
The other team chooses a player to shoot the 6.5\,m.
All other players must leave the goal area.
After the referee's whistle the goalkeeper must ride the unicycle freely and not rest on the goal.
The 6.5\,m is direct.
The player executing the 6.5\,m may only touch the ball once.
The ball shall be hit with the stick, not dragged, flicked or lifted on the stick.
If no goal is scored, play continues as soon as the ball touches the post, the keeper touches the ball or the ball crosses the extended goal line.
A 6.5\,m awarded at the end of, or after a time period has ended, is still executed but play does not continue after an unsuccessful shot.

\subsection{Penalty Goal}
If the defending team prevents a goal from being scored through an illegal play and if, in the opinion of the referee, the ball was traveling directly toward the goal and would definitely have entered the goal without being touched by another player, a penalty goal may be awarded to the attacking team.
If there is any doubt as to the certainty of a goal, a 6.5\,m must be awarded as described in section \ref{subsec:hockey_penalties_65m}.

\subsection{Face-off \label{subsec:hockey_penalties_face-off}}
To resume the game without penalizing one of the teams, a face-off can be used.
For the face-off, the referee drops the ball between two opposing players.
The ball should be dropped from below hip height of players in the vicinity.
One player from each team may take part in the face-off with all other players' unicycles and sticks at a distance of at least 2\,m from the ball.
Play starts when the ball touches the ground as signalled by the referees whistle.
A face-off during the game is executed where the ball was when the game was interrupted.
Exception: Within the goal area, the face-off is executed at the closest corner mark.

\subsection{Penalty Box}
The referee can send a player off the field for two minutes, five minutes or for the remainder of the game.
When a player is sent off for the remainder of the game they may not take part in the current match or their teams following match.
However, after a five minute period the penalised team may bring a player on.
These penalties are given in the case of unsporting behavior and also for intentional or dangerous disregard of the rules.
While a player is in the penalty box, the team may not substitute a replacement for that player.

The referees should consider the following guidelines when punishing a player.
The timer should be stopped while referees discuss the appropriate punishment and explain their ruling to players:

2 minutes:
\begin{itemize}
\item Intentional delay of the game
\item Repeated fouls by the same player
\item Intentional foul
\item Dangerous play
\item Backchat to referee (Constant backtalking to the referee or questioning decisions)
\item Intentional usage of incorrect equipment and clothing
\item Intentionally having too many players on the field
\end{itemize}
5 minutes:
\begin{itemize}
\item Repeated fouls by a player who has previously received a 2 minute penalty
\item Intentional dangerous foul
\item Violent conduct against other players, their team officials or spectators
\end{itemize}
Off for the remainder of the game:
\begin{itemize}
\item Repeated fouls by a player who has previously received a 5 minute penalty
\item Repeated violence of a player who has already received 5 minutes before
\item Violence against referees
\end{itemize}

\section{Event Flow}

\subsection{Game Duration}

The play time is given by the playing schedule and is a relative play time.
The time stops only at the request of the referee.
The teams change sides during the break.
At the start of each period, all players must be in their own half of the field.
Each period starts with a face-off at the center mark.
If the game ends in a draw and a decision is necessary, play is continued with extended time.
If it's still a draw, a decision is reached with a penalty shootout.

\subsection{Penalty Shootout}
Three of the players from each team get one penalty shot each.
If it is still a draw, each team shoots one more penalty until there is a decision.
It is possible that one player can take more than one shot.
However, in all cases at least two other players have to take a shot before the same player can shoot again.

For the penalty, all players except for a defending goalkeeper leave the corresponding half of the playing field.
The goalkeeper must be close to the goal line, at least until the attacking player has had contact with the ball.
The referee places the ball on the center point and the player taking the shot will, after the whistle of the referee, play the ball from there, trying to score a goal.
The player must remain in motion towards the goal line with no backwards movement or stopping allowed.
Once the ball has been shot, the play shall be considered complete.
No goal can be scored on a rebound of any kind (an exception being the ball off the goal post and/or the goalkeeper and then directly into the goal), and any time the ball crosses the goal line, the shot shall be considered complete.

\subsection{Riding The Unicycle}
The player has to be riding the unicycle freely.
He or she may use the stick as support but must not rest on the goal or the wall or something similar.
It is not sufficient to release the goal only quickly for the time while the goalkeeper takes part in the game.
A short support on the wall to avoid a dismount can be tolerated.
A player who is falling off the unicycle may take part in the game until touching the ground.
A remounting player must sit on the seat and have both feet on the pedals before participating in the game again.
If a player who is not riding a unicycle shoots into their own goal, the advantage rule applies for the attacking team and the goal is valid.

\subsection{Contact With The Ball}
The stick, the unicycle and the whole body can be used to play the ball.
It all counts as a contact.
Players are allowed to play the ball with the body twice in a row only if one of the contacts is passive.
When the ball is played with the body, the player must not catch or otherwise hold the ball and the contact with the ball should be instantaneous.
For arms and hands see also section \ref{subsec:hockey_goal-shots_with-arms-or-hands}.

\subsection{Start and Stop}
Starting and resuming the game is always initiated by the referee's whistle.
If a team starts to play before the referee's whistle, it is stopped immediately by two or more quick consecutive blows of the whistle.
Then, the previous referee ruling is repeated.
When the referee blows the whistle during the game, it is interrupted immediately.

\subsection{Restart After A Goal}
After a goal, the non-scoring team gets the ball.
All players must go to their own half.
After the referee's whistle, the game resumes when the ball or a player of the team in possession crosses the center line.
It is legal to directly shoot a goal after passing the center line, for example without passing the ball to another player first.

\subsection{Ball Out Of Bounds}
If the ball leaves the field, the game is interrupted immediately (even if the ball comes back in).
The team opposite to that of the player who last touched it gets a free shot.
The free shot is done 1.0\,m in from the side line.

\subsection{Ball In Spokes}
If the ball gets stuck between the spokes of someone's unicycle, the opposing team gets a free shot (not a 6.5\,m penalty).

\section{Fouls}

\subsection{General Considerations}
All players must take care not to endanger others.
The game is non-contact: the opponents and their unicycles may not be touched.
The players must take care not to hit an opponent with their stick, especially after a shot.
Only in the vicinity of the ball (defined as the ball within the radius of the outstretched arm length plus stick) may a player touch an opponent's stick with their stick to block them.
However, this contact may not be hard.
It is illegal to turn the blade of the stick upside down in order to hook into an opponent's stick.
Raising the opponent's stick is allowed in principle, if not done using exaggerated roughness.
If the opponent's stick is raised to a high stick (see section \ref{subsec:hockey_safety_stick}), it is always considered exaggerated roughness.
Intentional delay of the game is not permitted and may result in a penalty and the stoppage of time.

\subsection{Right Of Way}
To keep the game going, rule violations that do not influence the course of the game should not be penalized.
The following rules apply when riders come into contact with each other:
\begin{itemize}
\item No player may endanger another player by forcing them to give way (for example, to push them toward the wall).
\item A player who is idling or resting on the stick must be evaded.
  However, the idling or resting player must ensure the stick does not SUB players as per rule \ref{subsec:hockey_fouls_sub}.
\item The leading of two players riding next to each other may choose the direction of turns.
If both are evenly side-by-side, the one in possession of the ball may choose the direction.
\item If two players are approaching each other directly or at an obtuse angle, both must take care to avoid contact.
  If contact occurs, the referee will penalise the player deemed to have caused the contact.
\item In all cases not mentioned above, it is up to the referee to make a decision.
\end{itemize}

\subsection{SUB (Stick Under Bike) \label{subsec:hockey_fouls_sub}}
A player who holds his or her stick in a way that someone else rides over or against it is always committing a foul regardless of the situation.

\subsection{SIB (Stick In Bike)}
If a stick gets into the spokes of an opponent, the holder of the stick is committing a foul.

\subsection{Insults}
A player must not insult the referee or other players.

\subsection{Moving The Goal}
The players are not allowed to move the goal.

\subsection{Obstacle}
A player who is off the unicycle must not be an obstacle for opponents.
The player is considered an obstacle if the player, the unicycle or stick is hit by the ball and also if an opponent cannot move around freely.
The player should remount at the same spot, but if necessary move out of the way of play first.

\section{Goal Shots}

\subsection{Goal Shot With Arms Or Hands \label{subsec:hockey_goal-shots_with-arms-or-hands}}
A goal is disallowed if scored with arms or hands.
The defending team gets a free shot (goalkeeper's ball).
This rule does not apply if the ball is shot into one's own goal.

\subsection{Long Shot}
A goal is disallowed if the last contact with the ball was made when the ball was in one's own half.
The defending team gets a free shot (goalkeeper's ball).
This rule does not apply if the ball is shot from the opponents' half into one's
own goal.

\subsection{Ball In The Outside Of The Net}
If the ball becomes lodged in the outside of the goal net, or if the ball entered the goal through a hole in the back or side of the net, a free shot is given against the team whose player last played the ball.

\section{Safety}

Attention must be drawn to the safety of the players and spectators.
Thus, the safety rules have to be obeyed strictly and all equipment must be in good condition.

\subsection{Clothing \label{subsec:hockey_safety_clothing}}
All items that protrude from the body that may cause injury (for example watches, necklaces, earrings) must be removed.
In instances where this is impossible, the items must be covered sufficiently to remove likelihood of injury.
Shoes must be worn and shoelaces must be short or tucked in.
The following optional clothing is suggested: knee pads, gloves, helmets, safety glasses and dental protection.

\subsection{Throwing Sticks}
A player must not intentionally drop or throw his or her stick.

\subsection{Top Of The Stick}
The upper end of the stick must always be covered with one hand to avoid injury to other players.
A brief removal of the upper hand from the stick to play the ball with that hand is acceptable provided that this is done in a safe manner.

\subsection{High Stick \label{subsec:hockey_safety_stick}}
The blade of the stick must always be below the players' own hips and the hips of all players in the vicinity who might be endangered.
Exception: When defending a shot on goal in the direct vicinity of one's own goal, the lower end of the stick can be raised as high as the crossbar of the goal.

\subsection{Exaggerated Roughness}
Exaggerated roughness can lead to injuries and must therefore be avoided.
