\chapter{Competitor Rules}

\section{Safety}

All riders must wear a helmet and shoes as defined in chapter \ref{chap:general_definitions}.

Gloves, shin guards, and knee protection are recommended.

\section{Unicycles}

Any unicycle may be used.
There is no restriction on changing unicycles during the competition.

\section{Rider Identification}

The rider number must be visible on the rider or unicycle.

\section{Protests}

\begin{comment2016}
The below rule doesn't really make sense.
When results are posted is too late to protest something that happened on a particular line during the competition.
If a protest is against something that happened on a line, it seems like it should be addressed while the competition is going on.
I would think a more likely protest would be that your final score does not match the number of lines you know you did, so your card must be re-checked.
\end{comment2016}

A protest can be lodged by anyone against an Line Judge's ruling.
Protests typically arise when a bystander (another rider, or a spectator) observes a rider make an infraction that is not recorded by the Line Judge, or when an Line Judge gives the wrong penalty.

Protests must be lodged with the event director within fifteen minutes of the official results being posted.
Protests must be in writing, and must note the rider, and section number and a description of the protest.

\section{Event Flow}

\subsection{Rider Responsibility}

The rider is responsible for knowing where a section starts and ends, and which route he or she is supposed to take.

If there is a lineup for a section, the rider must go to the end of the line after each attempt.

If two or more riders are on overlapping sections at one time, the rider who started first has the right-of-way.


\subsection{Score Card}
The rider is responsible for his or her scorecard.
If it becomes damaged, the rider can ask the Event Director for a new one.
If it becomes lost, the rider will be issued a new card but their score will return to zero.

\subsection{Scoring Points}
The preliminary course is divided in different sections of easy, medium and hard lines.
Easy lines are worth one point, medium lines are worth three points and hard lines are worth seven points.
The objective is to score as many points as possible by successfully riding (``cleaning'') sections within the specified time period.

\textbf{Preliminary Round Points}

\begin{tabular}{|l|l|}
\hline
\textbf{Difficulty} & \textbf{Points}\\
\hline
Easy & 1 point\\
\hline
Medium & 3 points\\
\hline
Hard & 7 points\\
\hline
\end{tabular}

During finals, the completion of a line with zero pedal grabs will be worth three points.
During a pedal grab, the foot may touch an obstacle as long as the foot is in full contact with the top of the pedal/crank.
A deduction of 0.5 points will be made for each pedal/crank grab used during completion of the line, with a maximum deduction of two points.

\textbf{Final Round Points}

\begin{tabular}{|p{5cm}|p{4cm}|}
\hline
\textbf{Number of Pedal Grabs Used for Line Completion} & \textbf{Total Points \newline Received for Line}\\
\hline
0 & 3 points\\
\hline
1 & 2.5 points\\
\hline
2 & 2 points\\
\hline
3 & 1.5 points\\
\hline
$\geq$4 & 1 point\\
\hline
\end{tabular}

A pedal/crank grab is defined as the rider placing their weight on an obstacle through the bottom of the pedal/crank which is in contact with the obstacle (see Definition Of ``Cleaning'' below).

A pedal/crank grab is considered complete after a clear takeoff by pushing through the pedal/crank and not though the tire.

The pedal/crank may be re-positioned during a pedal/crank grab without being considered a new grab as long as the pedal does not move more than the width of the pedal away from the initial position on the obstacle.
Traversing an object in continuous half pedal width grabs will result in multiple pedal/crank grabs recorded.

A time stamp (to the minute) of each line completion will be recorded by each line judge during finals to be used in the event of a tie.


\subsection{Definition Of ``Cleaning''}
Cleaning a section is defined as follows:

\begin{enumerate}
\item \textbf{Riding into a section.} This is defined as the moment a rider's tire crosses over the start line.
\item \textbf{Riding through the section without ``dabbing''.} Dabbing is defined as follows:
	\begin{enumerate}[a.]
	\item Allowing any part of the rider's body to touch the ground or obstacle.
	If loose clothing brushes against the ground or obstacle but does not influence the rider's balance, then this is acceptable (does not constitute a dab).
	\item Allowing any part of the cycle except the tire, rim, spokes, crank arms, pedals,or bearing caps to touch the ground or obstacle.
	\item Riding or hopping outside the boundaries of the defined section.
	The unicycle must be within the boundaries of the section at all times, even if the rider is in the air (for example, a rider cannot hop over a section boundary that turns a corner, even if they land back inside the section).
	\item Breaking the flagging tape or other markers that are delineating a section boundary.
	Touching or stretching the tape does not constitute a dab, as long as the unicycle remains inside the section boundary.
	\item Riding a section in any way that is not consistent with the instructions outlined for that problem.
	\end{enumerate}
\item \textbf{Exiting the section.} A rider exits a section when their wheel fully cross over the finish line, or are within a defined finish area (such as a taped circle on top of a boulder).
If there is no clearly defined finish area or finish line, the rider has deemed to have exited the section when they are back on the ground at the end of the section.
The rider must finish in control as demonstrated by remaining mounted for a 3-second count from a judge after exiting the section.
\end{enumerate}

When hooking a pedal on an obstacle, it is acceptable for a rider's heel and/or toe to initially contact the ground, as long as most of the rider's foot is still on the pedal.
However, after a rider is established in position, weighting the heel or toe on the ground constitutes a dab.

It is acceptable for a rider's body to be entirely on one side of the centerline of the unicycle.

\subsection{Multiple Attempts}
Riders may attempt any problem multiple times until they succeed or decide to abandon the section.
During preliminary rounds, it is not possible to earn additional points by cleaning a section more than once, and no points are awarded if the rider does not clean the entire section.
During finals a rider may re-complete a line with fewer pedal grabs to receive a higher score.
Only the rider's best result at each line will be recorded.

Similarly, the time of the best result will be recorded as the finishing time for each line.
For example, if a rider finishes their final line at 25 minutes receiving 1 point but retries the line finishing the same line at 45 minutes with 2.5 points, their 2.5 points and 45 minutes time of completion will be recorded in the results.
If the rider re-completes a line without improving their original score, the original time of completion will remain.

\subsection{Time Limit}

All riders must stop riding at the end of the time limit.
If a rider is mid-way through an attempt when the time limit is reached, they are allowed to finish that attempt.

The rider must gauge their time.
No allowance will be made for riders who spend too much time at one obstacle and cannot complete the course before the end of the competition time period.

\subsection{Prohibited Activities}
No rider may attempt any obstacle prior to the start of the competition.

Intentional modification of a section by riders or spectators is prohibited.
Note that kicking objects to test stability does not constitute intentional modification if an object moves.
If a section is unintentionally modified or broken by a rider, they should inform the Event Director or Course Setter who will return the obstacle to its original form if possible.%comment2016 is course setter defined elsewhere or only used here?
