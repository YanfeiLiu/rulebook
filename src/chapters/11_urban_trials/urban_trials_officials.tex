\chapter{Judges and Officials Rules}

\section{Trials Officials}

\subsection{Trials Director}

The Trials Director is the head organizer and administrator of trials events.
With the Convention Host, the Trials Director determines the course, obtains permis, interfaces with the community, and determines the system used to run the event.
The Trials Director is responsible for the logistics and equipment for all trials events.
Trials Director is in charge of keeping events running on schedule, and answers all questions not pertaining to rules and judging.
The Trials Director is the highest authority on everything to do with the trials events, except for decisions on rules and results.

\subsection{Chief Judge}

The Chief Judge is the head Trials official, whose primary job is to make sure the competitors follow the rules.
The Chief Judge makes all final decisions regarding rule infractions.
The Chief Judge is responsible for resolving protests.

\subsection{Line Judge}

The line judges are responsible for judging whether a rider has successfully cleaned a section.

\section{Safety}

If an Line Judge or the Trials Director feels the safety is compromised by a rider attempting a section that is beyond their ability, they may prohibit the rider from attempting that obstacle.
In cases where a fall from an obstacle could be particularly dangerous, the Trials Director may also permit attempts only by highly skilled riders who believe they will qualify for the Finals.

\section{Scoring Methods}

\subsection{Method 1}
Method 1 is mandatory for all major competitions and is the recommended method for all other competitions.

Each rider is issued a scorecard (see example) at the beginning of the competition, and must give their card to a Line Judge prior to attempting a section.
If the competition is self-judged, the rider attempting the section gives their card to another rider who must observe them attempt the section.
If they clean the section, the line judge indicates that they have completed the section by initialing or punching the box corresponding to that section.
At the end of the competition, riders hand in their cards to the Trials Director or to a designated person for tallying of scores.

\textbf{Example scorecard:}

\begin{tabular}{|l|l|l|}
\hline
\textbf{Rider Name:} & \textbf{Category:} & \\
\hline
Section Number & Section Number & Section Number \\
\hline
1 & 6 & 11 \\
\hline
2 & 7 & 12 \\
\hline
3 & 8 & 13 \\
\hline
4 & 9 & 14 \\
\hline
5 & 10 & 15 \\
\hline
\end{tabular}

\subsection{Method 2}
This method is intended for small events, and is not appropriate for larger events.
Major events such as Unicon or national meets must not use this system of scoring.

In this method, one or two line judges keep track of scores for numbered sections on a computer or paper spreadsheet such as this:

\begin{tabular}{|c|c|c|c|c|c|c|c|c|c|c|c|c|c|c|c|}
\hline
 & \textbf{Section:} & & &  & &  &  &  &  &  & & & &  &   \\
\hline
\textbf{Rider} & \textbf{Category} & 1 & 2 & 3 & 4 & 5 & 6 & 7 & 8 & 9 & 10 & 11 & 12 & 13 & 14 \\
\hline
Jane Smith & Expert &  &  &  &  &  &  &  &  &  &  &  &  &  & \\
\hline
John Anderson & Sport &  &  &  &  &  &  &  & &  &  &  &  &  &  \\
\hline
 &  &  &  &  &  &  &  &  &  &  &  &  &  &  & \\
\hline
 &  &  &  &  &  &  &  &  &  &  &  &  &  &  & \\
\hline
\end{tabular}
