\chapter{Competitor Rules}

\section{Safety}

For Long Jump on Track and Long Jump on Platform, riders must wear shoes, a helmet and knee pads.
In addition, for Long Jump on Track gloves are required.

For High Jump and High Jump to Platform, riders must wear shoes and a helmet.
Shin guards are strongly recommended while gloves and knee pads are optional.

Definitions of all safety gear can be found in chapter \ref{chap:general_definitions}.

\section{Unicycles}

Regular unicycles must be used (see definition in chapter \ref{chap:general_definitions}).
No restriction on wheel or crank size.
Metal pedals are allowed for their strength and better grip.
This may make it impossible to hold this event on a sensitive track surface.

\section{Rider Identification}

The rider number must be visible on the rider or unicycle.

\section{Protests}

Protests must be filed on an official form within two hours of the posting of event results.
Every effort will be made for all protests to be handled within 30 minutes from the time they are received.

\section{Number of Attempts}
For both the preliminary and final round, the maximum number of attempts per rider at any one distance is three.
Attempts can be made at any provided distance and riders may skip distances as they choose.
If a rider attempts any distance, they must successfully complete it before attempting a higher distance.
However each attempt must be at the same or higher distance.
This means that a rider cannot attempt a longer distance, fail, and then attempt a shorter distance.
The best successfully completed attempt is the rider's result.

In addition, for the preliminary round riders have \textit{maximum of 12 total attempts} to complete their best performance.

\section{Broken Unicycle}
If the unicycle breaks during an attempt, a new attempt must be given to the rider.

\section{Height and Distance Settings}

In any High Jump competition, the starting height must be set at a whole number of centimeters.
In any Long Jump competition, the starting distance must be set at a whole fivefold of centimeters, i.e.\ ending in 5 or 0.
Any increment in set height/distance must be a whole number of centimeters.
In any case, heights or distances must be set as accurately as reasonably possible.

\section{Event Flow: High Jump over Bar}
The rider and unicycle jump over a bar, without knocking it down, and ride away without a dismount.
There are three parts to a successful jump:
\begin{enumerate}
\item Riders must mount before the start line, to show they are on the unicycle and in control.
The attempt starts when the rider crosses the start line.
The rider may break off from a jumping attempt before leaving the ground, but must then start again from behind the start line.
That attempt then doesn't count.
\item Riders must jump over the bar without knocking the bar off the apparatus.
The bar can be hit as long as it does not fall.
If the bar falls before the rider crosses the finish line, it counts as an unsuccessful attempt.
\item After landing, the rider must stay in control of the unicycle until he cross the finish line without dismounting, touching a hand to the ground or any other object, or knocking down the bar or any of the high jump apparatus.
\end{enumerate}
The rider starts at a low height and after each successful attempt, the height increases at set intervals.
The maximum height that was completed is recorded as the rider's result.

Around the High Jump over Bar apparatus a circle with a radius of 3 meters must be marked.
This circle is start and finish line.
The rider can cross it wherever he wants.
Riders must ride or hop across the finish line in control for the attempt to count.

\section{Event Flow: High Jump onto Platform}
The object is to jump from the ground to a platform, with no pedal or crank grabs.
Riders must remain in control of the unicycle (stay mounted) for 3 seconds after landing.

The rider and unicycle jump up to a landing surface on a platform (see setup for definitions) and remain on the landing surface for a 3-second count from a judge.
The rider begins at a low height, and after each successful attempt, the height increases at set intervals.
The maximum height that was completed is recorded as his or her result.

There are three parts to a successful jump:
\begin{enumerate}
\item Riders must mount their unicycles on the ground.
A jump attempt occurs when any part of the unicycle or rider touches any part of the platform or landing surface.
The rider may break off from a jump attempt before touching the platform or landing surface.
This does not qualify as a jump attempt.
\item Riders must jump ``to rubber'' on the landing surface.
No part of the rider, or any part of the unicycle other than the tire, may touch the platform or landing surface.
\item After landing, the rider must remain mounted and on the landing surface for 3 seconds, as counted by a judge.
The rider may do any form of idling, hopping or stillstanding during the 3 seconds.
Once the judge has counted 3 seconds, the rider may return to the ground in any fashion they choose.
\end{enumerate}
If a rider completes all of the requirements listed in items 1 through 3 above, the jump is deemed successful.
Otherwise, it is deemed a failed jump attempt.

\section{Event Flow: Long Jump on Track}
The rider jumps as far as possible from a jump marker, to a landing without a dismount.
The rider must then continue riding across a finish line to show control.
Riders must clear 3 markers (jump marker, landing marker and finish line) to make the jump count.
Riders may jump with the wheel going forward or sideways.
After landing, the rider must stay in control of the unicycle for the remainder of the distance from the jump marker to the finish line without dismounting, or touching a hand to the ground or any other object.
If the tire touches the jump marker before takeoff or the landing marker, it counts as a foul.
Riders may break off in a run as long as they are between start line and jump marker but if they cross or touch the jump marker, the attempt counts, including fouls.
The farthest non-fouling, successful jump is recorded.

The rider must clear the jump marker and the landing marker without touching them; they must has clear the finish line to make it a valid jump.
Jump distance is measured between the outer edges of the jump and landing marker.

To avoid endless competitions, the length to jump will always increase by 5cm for each round.
Once there are only 5 riders left, it's up to the riders to decide in which steps they continue.
For each age group the minimum length should be adjusted to a useful level such as 150cm for 15+ and 70cm for 0-15.
The Event Director can adjust this depending on the level at the competition.

\section{Event Flow: Long Jump on Platform}
In the Long Jump on Platform competition, the rider attempts to jump as far as possible from a take-off platform to a landing platform without a dismount.
The rider must remain mounted and in control on the unicycle for 3 seconds on the landing platform (described in setup below).

Riders may jump with the wheel facing forward or sideways.
The rider may break off the attempt as long as they are still on the take-off platform.
As soon as they jump in any direction (landing anywhere but the take-off platform), it counts as an attempt.
The farthest non-fouling, successful jump is recorded.

The rider must begin stationary on the take-off platform and must land on the landing platform without touching the ground.
The rider must land with their wheel on top of the landing surface.
They may not pedal grab then go to tire.
After landing, the rider must remain mounted and on the landing platform for 3 seconds, as counted by a judge.
The rider may do any form of idling, hopping or still-standing during the 3 seconds.
Once the judge has counted 3 seconds, the jump is complete.

To avoid endless competitions, the length to jump will always increase by 5cm for each round.
Once there are only 5 riders left, the final starts and it's up to the riders to decide in which steps they continue.
