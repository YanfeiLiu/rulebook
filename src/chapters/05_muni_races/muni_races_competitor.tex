\chapter{Competitor Rules}

\section{Safety}
For all muni events, riders must wear shoes, knee pads, gloves/wrist-guards and helmets (see definitions in chapter \ref{chap:general_definitions}).
Additional equipment such as shin, elbow or ankle protection are optional.

Water and food are the responsibility of the rider.
Hosts may offer food and water stations at their discretion.

\section{Unicycles}

There are no restrictions on wheel size, crank arm length, brakes, or gearing.

\section{Rider Identification}

Riders must wear their race number clearly visible on their chest so that it is visible during the race and as the rider crosses the finish line.
Additionally, the rider may be required to wear a chip for electronic timing.

\section{Protests}

Protests must be filed on an official form within two hours of the posting of event results.
Every effort will be made for all protests to be handled within 30 minutes from the time they are received.

\section{Event Flow}

\subsection{Uphill Race \label{sec:muni_uphill}}

An Uphill muni race challenges a riders ability to climb.
Courses may be short and steep or longer, endurance-related challenges.

\begin{comment2016}
We decided to remove the following variant, because it doesn't fit the scope of this chapter.
If such a race were to be run, the rules would have to be thought out and documented.

Generally it is a timed event, but on an extremely difficult course, riders can be measured as to how far they ride before dismounting.
The race can be offered as a no-dismounts challenge, which either measures who gets the farthest, or disqualifies anyone who doesn't complete the distance without a dismount.
Multiple tries can be allowed, or the race can be a simple timed event.
\end{comment2016}

\subsection{Downhill Race \label{sec:muni_downhill}}

A Downhill muni race is a test of speed and ability to handle terrain while riding downhill.

\subsection{Cross Country (XC) Race\label{sec:muni_xc}}

The Cross Country race is an off-road distance race that challenges a rider's fitness and ability to ride fast on rough terrain.

\subsection{Starting}

Riders start with the fronts of their tires (forwardmost part of wheel) behind the nearest edge of the starting line.

\subsection{False Starts}
A false start occurs if a rider's wheel moves forward before the start signal, or if one or more riders are forced to dismount due to interference from another rider or other source.

\subsection{Passing}
Riders must pay attention while passing and avoid physical contact as much as possible.
Violations of this passing rule may result in disqualification or a time penalty, to be determined and announced before the start of the race.

If a faster rider comes from behind, the rider in front does not need to yield to the rider behind, as long has he/she is mounted.
The faster rider should try to pass when safe.
A mounted rider always has priority over an unmounted rider.

\subsection{Dismounts}
Dismounts are allowed in all muni races unless otherwise noted.
In mass-start events, dismounted riders must yield to mounted riders behind them as quickly as possible after a dismount, and until re-mounted.
Riders may not impede the progress of mounted riders when trying to mount.
If necessary they must move to a different location so mounted riders can pass.
If riders choose not to ride difficult sections of the course, they must not pass any mounted riders while walking or running through them.
In time trial-type events, see below for variations based on the other event details.
Violations of these non-riding rules may result in disqualification or a time penalty, to be determined and announced before the race start.

\subsubsection{Dismounts: Uphill}
Riders must ride the entire course.
In the event of a dismount, the rider must remount the unicycle at the location of the wheel at the moment of the dismount.
Riders may also choose to back up (toward the start line) to remount, if they prefer.

\subsubsection{Dismounts: Downhill}

Dismounted riders must not impede the progress of, or pass mounted riders.
They must remain aware of riders coming from behind, and not block them with their
unicycles or bodies.

Running and fast walking are not allowed, except momentarily to slow down after an unplanned dismount.
After a dismount, riders have to come to a complete halt before mounting the unicycle again.
If a rider falls in front of their unicycle, they may run back up the hill to retrieve it, but must come to a complete halt before remounting.
Riders may generally walk slowly if necessary.
A rider may choose to dismount for a difficult section, but must walk slowly through the section until stopping to remount.
The following penalties apply if riders disregard this:
\begin{itemize}
\item Riders get an immediate time penalty of five seconds when they intentionally run or walk fast, not recovering from a fall.
A judge must clearly indicate when the time penalty starts and when the rider may continue, for example by blowing a whistle and counting down from five.
\item Riders get disqualified immediately when they do not stop and wait five seconds after the judge's indication.
The disqualification should be signaled to the rider immediately by a judge, for example by blowing a whistle twice.
\item Judges must be trained and tested to correctly enforce these rules.
Riders must be informed about the type of signaling prior to the race.
\end{itemize}

\subsubsection{Dismounts: Cross Country}

If the event is held as a time trial, dismounted rider restrictions must be announced before the start of the race.
Depending on course length and difficulty, dismounted riders may be required to walk, or walk only limited distance, or have no restrictions at all.

\subsection{Finishes}

\subsubsection{Finishes: Uphill}
Riders must cross the finish line mounted on the unicycle, having both feet on the pedals.
In the event of a dismount at the finish line the rider must back up, remount and ride across the finish line again.
\subsubsection{Finishes: Cross Country and Downhill}
Riders can cross the finish line mounted as well as walking. %comment2017 what happens if they run?
Any finish where the rider is not mounted on the unicycle, having both feet on the pedals, will be penalized by adding a 10 second penalty.
