\chapter{Event Organizer Rules}

\section{Venue}

\subsection{Uphill}
Uphill courses must be primarily uphill but may include flat or downhill sections.
At Unicon, if the terrain is technically easy (i.e.\ smooth to ride, no or only very small rocks and roots), the main course has to be longer than 500 m or have a height difference greater than 100 m.
If the terrain is more difficult, the course may be shorter or have less elevation gain.
The event hosts may consider additionally offering Beginner/Advanced categories competing on shorter and easier tracks or on selected parts of the main course.
It is recommended that Uphill courses at other types of events or competitions adhere to these rules as well.

\subsection{Downhill}
Downhill courses must be primarily downhill but may include flat or uphill sections.
Recommended course length is 2.5 km, or 1 km at a minimum, depending on available terrain, trails and schedule time.
The overall course difficulty must be rated with the Unicycle Downhill Scale.
A minimum score of 20 points is recommended.
Courses with scores below 15 points should be labeled ``beginner downhill'' for clarity.

\subsection{Cross Country}
A Cross Country race should be at least 10 km or longer, depending on available terrain, trails and schedule time.
The overall course difficulty must be rated with the Unicycle XC Scale.
If multiple laps need to be completed, then the whole distance is the basis for the rating.
A minimum score of 20 points is recommended.
Courses with scores below 15 points should be labeled ``beginner XC'' for clarity.
If only shorter trails are available, riders can be required to complete two or more laps of the course.

Very dangerous sections should be secured (for example by removing sharp stones/branches from areas where riders are likely to fall/run into due to the physics of the course).

Downhill and Cross Country courses must be rated in advance by two people using the appropriate IUF Muni Difficulty Scale (see the \href{https://unicycling.org/files/iuf-muni-difficulty-scale.pdf}{IUF publication} and \href{https://muni-scale.info/}{online calculator}).
Ratings and their underlying data must be published at least seven days prior to the event.

\section{Officials}

The host must designate the following officials for each muni race:
\begin{itemize}
\item Muni Director
\item Referee
\item Course Marshall
\end{itemize}

\section{Communication}

\begin{comment2016}
Some thoughts on what might need to be communicated:
\begin{itemize}
\item false start method
\item age groups
\item starting method
\item timing method?
\item course map?
\item results
\end{itemize}
\end{comment2016}

The host must publish two lists of results for each discipline after the competition: Age group based ranking and overall ranking (separating
male/female).

If the hosts wish to include events other than the first three (Up, DH, XC), they must remember to provide detailed rules for these events at the same time the events are announced.

Details of all non-track racing events, or other events with unique courses or details must be published as soon as they are known.
This is to provide competitors with the information they need to train, and to help them prepare the appropriate unicycles.
These are major needs for attendees from far away.
Necessary details depend on the event, but include things like course length, elevation and elevation change, steepness, level of terrain difficulty, amount of turns, riding surfaces, course width, etc.
Maps should be provided when possible.
While sometimes courses cannot be planned until weeks or days before the convention, as soon as they are known the details must be posted to the convention web site and/or all places where convention information is posted.
It is acceptable to publish tentative courses while waiting for permits to be approved, etc.

\section{Age Groups}

\begin{comment2016}%comment2016 these age group rules are awful
Is this really meant to be ``maximum''?
\end{comment2016}

Age groups must be offered as male and female age group.
There must not be any age group specific restrictions on equipment.
The following age groups are the maximum allowable for muni competitions:

\begin{tabular}{|l|l|}
\hline
Under 15 & Youth \\
\hline
15-16 & Juniors \\
\hline
17-18 & Rookies \\
\hline
19-29 & Elites \\
\hline
30-49 & Masters \\
\hline
50+ & Veterans \\
\hline
\end{tabular}

\section{Route Signaling}
Courses must be clearly marked.
At each intersection, the correct direction must be indicated by at least one of the following options:
\begin{itemize}
\item \textbf{Option 1:} Painting or chalk marking (only if authorized by authorities).\\
Arrows showing the correct course direction must be drawn 5 to 10 meters before the intersection, at the intersection, and 5 to 10 meters after the intersection.
Crosses must be drawn on each wrong direction at the intersection and 5 to 10 meters after the intersection on the wrong paths.
\item \textbf{Option 2:} Using bands (also known as barrier tape).\\
Small pieces of bands (shorter than 1m) are used instead of arrows to show the right way to go.
Longer pieces of bands barring the wrong paths over their entire width are used instead of crosses.
These bands can lie on the ground if they cannot be hung in the air because of any restriction.
\item \textbf{Option 3:} Using any other clear signaling method.\\
Chipped wood or large signs with printed arrows or crosses are examples of other acceptable method.
Similar to options 1 and 2, signaling must be placed 5 to 10 meters before, at, and 5 to 10 meters after any intersection, as well as signaling very clearly any wrong direction at the intersection, and 5 to 10 meters after it on the wrong paths.
\end{itemize}
Any element of the route such as the ground, trees, rocks, or barriers can be used as marking or for hanging bands, as long as the result is easily visible and not likely to be erased/removed by the passage of riders, other occupants, or weather.
If the weather forecast predicts rain, option 2 or 3 is preferred over option 1.
At major intersections, having a volunteer signaling the correct way, in addition to marks or bands, is highly recommended.
Any signaling option can also be used on any long section between intersections, in order to confirm to riders that they are still on the right track.

If authorized by authorities, an effort must be made to mark the courses a few days before they occur, so riders can practice on it.

Any change in the track from the one given in the website/program book must be announced to the riders by email as soon as this is known, even if this is a short amount of time before the race.
If the change occurs the day of the race, riders must also be told on the start line that there has been a change.

\section{Practice}

For all muni races, every rider must get the chance of at least one test run to get familiar with the track before the actual race.
If possible, the track should be open for training during all days of the event prior to the race.
For multi-day events the muni competitions should take place during the second half of the event in order to give riders more time to practice on the course.

\begin{comment2016}
The following rule came from chapter 1.
This rule is poorly written.
What happens between 1 day and 6 days?
\end{comment2016}

If the course is open for practice to all riders for at least 7 days leading up to the event, then there are no restrictions on who can compete.
If the course is not open for practice until the day of the event, then anyone who has pre-ridden the course is not allowed to compete.
Organizers must therefore ensure that course marking and set-up are done by non-competing staff/volunteers.

\section{Race Configuration}

For uphill and downhill races riders should race one at a time, released at regular time intervals.
If the schedule has a small time window for the race, riders should be run in heat sizes that allow passing on the course, and do not bottleneck at the beginning.

For a downhill course length less than 2 km, two separate runs should be held.
In this case the ranking of the riders is based on the fastest of the two runs.

For the uphill race, either one or two runs can be held.
In the case of two runs, the ranking of the riders is based on the fastest of the two runs.

\section{Starting Configuration}

There are three different types of starting modes, that can be used in muni races.
\begin{enumerate}
\item \textbf{Mass starts:}\\
All riders start at the same time.
Mass starts must not be used when the race duration is expected to be shorter than 30 minutes.
The track must provide sufficient space for passing in the first section, so that the field of starters is aligned before the track narrows down.
Space for passing must be given along the track.
Mass starts with more than 40 riders have to be split to avoid accidents.
\item \textbf{Heat starts:}\\
Groups of riders start at intervals that can vary from 30 seconds to a few minutes.
The maximum number of riders per heat is determined by the average width of the first 100m of the track.
There can be one rider for each meter in width.
The first heats must be separated based on gender with the first heat consisting of the top males and the second heat consisting of the top females.
After the top males begin, there must be a minimum 10 minute time interval before the top females start.
After the top females start there must be a minimum 5 minute time interval before the next heat begins.
\item \textbf{Individual starts:}\\
Individual riders start at intervals that can vary from 30 seconds to a few minutes.
\end{enumerate}

\section{Starting Order}

The fastest riders should always start first, regardless of the starting mode.
The order can be determined by seeding runs or another method.
