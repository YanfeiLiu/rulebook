\chapter{Event Organizer Rules}

\section{Venue}

These are races held usually on roadways or bike paths.

For any road race over a fixed (traditional) distance, it is encouraged that the course complies with the requirements for IUF World Records.
In short: \\
(1) The start and finish points of the course, measured along a theoretical straight line between them, shall not be further apart than 50% of the race distance. \\
(2) The overall decrease in elevation between the start and finish shall not exceed 1:1000, i.e.\ 1m per km (0.1%).

Please refer to the IUF World Records Guidelines for details.

\section{Officials}

The host must designate the following officials for each road race:
\begin{itemize}
\item Race Director
\item Referee
\item Starter
\end{itemize}

\section{Communication}

\begin{comment2016}
Some thoughts on additional required communication:
\begin{itemize}
\item any changes to the standard rules
\item heat assignment/seeding method
\item false start method
\item cut-off time
\item age groups
\item starting method
\item whether the event qualifies for a world record
\item timing method
\item course map
\end{itemize}
\end{comment2016}

The host must announce the false start method at least two months before the event.

Details of all non-track racing events, or other events with unique courses or details must be published as soon as they are known.
This is to provide competitors with the information they need to train, and to help them prepare the appropriate unicycles.
These are major needs for attendees from far away.
Necessary details depend on the event, but include things like course length, elevation and elevation change, steepness, level of terrain difficulty, amount of turns, riding surfaces, course width, etc.
Maps should be provided when possible.
While sometimes courses cannot be planned until weeks or days before the convention, as soon as they are known the details must be posted to the convention web site and/or all places where convention information is posted.
It is acceptable to publish tentative courses while waiting for permits to be approved, etc.

\section{Age Groups}

The following age groups are the minimum required by the IUF to be offered at the time of registration for any Road Racing discipline: 0-13, 14-18, 19-29, 30-UP.
For any discipline for which there is a 24 Class wheel size category, also an age group 0-10 (20 Class) must be offered.
All age groups must be offered as male and female age group.

\section{Practice}

\begin{comment2016}
The following rule came from chapter 1.
This rule is poorly written.
What happens between 1 day and 6 days?
\end{comment2016}

If the course is open for practice to all riders for at least 7 days leading up to the event, then there are no restrictions on who can compete.
If the course is not open for practice until the day of the event, then anyone who has pre-ridden the course is not allowed to compete.
Organizers must therefore ensure that course marking and set-up are done by non-competing staff/volunteers.

\section{Ungeared Awards}
At Unicon, if there are five or more geared male riders in an Unlimited event, the fastest three ungeared male riders will be awarded with an ungeared title for that event.
Similarly, if there are five or more geared female riders in an Unlimited event at Unicon, the fastest three ungeared female riders will be awarded with an ungeared title for that event.
This is only for the overall classification, not for Age Groups.
Other events can choose to award the fastest three, one, or none of the ungeared riders as they wish.

\section{Criterium}
Criterium can be held around city block(s) or within a large parking lot.
The recommended lap length is 500 to 1000 meters.
It is recommend that the course has left and right turns, with consideration given to the safety and number of riders in the race.
The race can be run as a set distance race (all riders will complete the same number of laps), or based upon time.
A set distance race is strongly recommended for larger events.
Different categories (e.g.\ Unlimited and Standard 24 Class) can have different race lengths.
The starts are a mass start.
If a rider falls due to other riders falling ahead or nearby, this is generally considered part of the race.
Intentional interference with other riders is not permitted and may result in disqualification by the Referee.

\subsection{Set Distance Length}
If the race is run as a set distance event, the number of laps should be announced clearly to riders before the start of the race.
There should be three to ten laps.
Each rider is responsible for counting their laps; organizers are not responsible for disqualified riders who do not complete the required number of laps.

\subsection{Time-based Length}
The Criterium can also be run as a time-based event.
Using the time from the top rider’s first two laps, the referee will determine how many laps could be completed in the desired time limit.
From this point on, the number of remaining laps (for the leaders) will be displayed and this will be used to determine when finish of the race occurs.
A bell will be rung with one lap to go.
Lapped riders in the race will all finish on the same lap as the leader and will be placed according to the number of laps they are down and then their position at the finish.

\section{Race Configuration}

Riders are usually divided by age group and unicycle type, such as 24 Class versus 29 Class unicycles, and/or Standard (a Regular Unicycle, any size wheel and cranks) versus Unlimited.

\section{Starting Order}

The goal in determining the starting order is to sort racers fairly by speed while still making sure that males and females race amongst themselves.
Unless otherwise noted below, the fastest riders start first, and also within a start group (heat or mass start), riders should be positioned in the line-up by speed with the fastest in front.
Starting order can be determined by seed time, or from the results of a previous Road Race in that competition.
For example, if the Marathon follows the 10k, the results of the 10k can be used to determine the starting order for the Marathon.
In the case that a racer does not have a seed time, and is signed up for a particular event (such as the Marathon) and did not participate in the previous race (such as the 10k), the Racing Clerk has the right to assign a starting position where they see fit.

\section{Starting Configuration}

Line-up order and heats must be assigned prior to the race.
There are three allowable formats for designating the starting configuration of a Road Race: individual start (section \ref{subsec:road_heat-assignment_individual-start}), heat start (section \ref{subsec:road_heat-assignment_heat-start}), or mass start (section \ref{subsec:road_heat-assignment_mass-start}).

To determine which start configuration to use, read the following rules from top to bottom.
Once you have an outcome, \emph{disregard} the remaining rules.
\begin{itemize}
\item If this is an ``Individual Time Trial'' format race, use individual start.
\item If the course is too narrow to allow for racers to safely and fairly start in heats, use individual start.
\item If you cannot safely start five or more riders across, use individual start.
\item If the starting field consists of 30 riders or less, use a mass start.
\item If the course does not allow for ten riders to ride abreast for at least 500 meters before the course narrows, use heats of 12 or more riders.
\item If the starting field consists of more than 50 riders, use heats of 20 or more riders.
\item In all other cases, use a mass start.
\end{itemize}

The various classes may share the race course, but Standard racers should always start separately from Unlimited racers, also in the case of mass starts.
Unlimited racers should start first, unless there is no risk that Unlimited riders have to pass Standard riders (for example they race on different days).

In the sections below, ``fastest rider'' means ``fastest rider by seed time.'' Seed time is defined as an estimated finish time, preferably based on past performance in similar event(s).
If no seed time is submitted by the rider or their coach, the organization can assign a seed time.

\subsection{Individual Start \label{subsec:road_heat-assignment_individual-start}}

Each rider is individually started at a fixed time interval, such as every 20 or 30 seconds.
Riders are sorted by speed with the fastest rider going first.
(Except in the case of an Individual Time Trial, where the race can start with either the fastest or slowest rider.)

\subsection{Heat Start \label{subsec:road_heat-assignment_heat-start}}

Heats should consist of at least 12 riders, either male or female (no mixed heats).
Heats may vary in size.
Heats are sorted by speed with the fastest heat going first.
The first heat should be devoted to the fastest males.
The second heat should be devoted to the fastest females.
The top males and the top females must have equivalent racing conditions.
The following heats should be sorted by speed.
The time intervals between heats should run as follows:
\begin{itemize}
\item For non-lapped races, there should be a time interval of at least 5 minutes (for the 10k) or 10 minutes (for the Marathon) between heats 1 and 2, as well as between heats 2 and 3.
  This is to ensure safe and fair racing for the top male and top female heats.
\item For lapped races (and races other than the 10k and Marathon), the time intervals between heats 1, 2, and 3 should be set up such that following heats have the least chance of interfering with the top male and female riders.
\end{itemize}

\subsection{Mass Start \label{subsec:road_heat-assignment_mass-start}}

A mass start is a start in which all racers of a certain class (such as Standard or Unlimited) start together.
Males and Females of the same class start at the same time.

\section{Starter Responsibilities}

If a verbal (spoken) count is used, there should be about 3/4 second between each element in the count, with the same amount of time between each of them.
Starters should practice this before the races begin.
Timing of the count is very important for an accurate start.
This count can be in the local language, or a language agreed upon before competition starts.

\section{Finishes}

If finish times for a race are timed using microchips or other non-photographic electronic equipment, finish order must be verified by photo timing equipment if the finishers are within 0.1 seconds of each other.
Also, in the case where a world record is suspected of being set, the time must be verified with photo timing equipment.

\section{Optional Race-End Cut-Off Time}

It may be necessary to have a maximum time limit for long races, to keep events on schedule.
When this is planned in advance, it must be advertised as early as possible, so attending riders will know of the limit.
Additionally, at the discretion of the Racing Director, a race cut-off time may be set on the day of or during an event.
The purpose of this is to allow things to move on if all but a few slow racers are still on the course.
These cut-offs need not be announced in advance.
At the cut-off time, any racers who have not finished will be listed as incomplete (no time recorded, or same cut-off time recorded for all).
Optionally, if there is no more than one person on the course per age category and awards are at stake, they can be given the last place in the finishing order.
But if each participating age category has had finishers for all available awards (no awards at stake), there is no need to wait.

\section{Special Marathon Events}

Exceptions from the default rules may be allowed for a marathon race that is embedded in a big city marathon.%comment2016 not only big cities
This allows the unicycling organizer to follow some requirements of the main marathon organizer in order for the unicycling marathon to fit within the larger event.

The following exceptions to the rules may be made:
\begin{itemize}
\item Mass start / Group start (Mass start could be forced by the main host for schedule requirements)
\item Start groups do not have to separate males/females and/or wheel sizes
\item Netto times (time from when the rider's wheel crosses the start line) can be used for placements while the Brutto time (time from when the race is started) counts for records.
\end{itemize}

\section{Race Distances and Distance Measurement}

\subsection{Fixed Distance and Free Distance Races}

The recognized fixed distance races are the 10km, Marathon (42.195km) and 100km.

A free distance race can be any race distance that is greater or less than 3\% of the distance of a recognized fixed distance race.

It is expected that Unicon will have at least two road events, of which at least one is a recognized fixed distance event

\subsection {Distance Measurement for Fixed Distance Races}

In the case of fixed distance races, the course must be accurately measured along the shortest possible path.
The course must be guaranteed to be no shorter than the advertised distance.

The following procedure is acceptable for accuracy.
A more accurate method is of course allowed.
\begin{enumerate}
\item Set out a calibration course on straight, flat asphalt, with a minimum length of 100 meters, using a steel measuring tape of 5 meters or longer.
\item Ride the calibration course at least once with a bike or unicycle (minimum wheel size 24 inch).
Ride normally, without too much wobble, and at normal speed.
Take care that mounting and dismounting don't cause the wheel to swerve, or be lifted from the surface.
Carefully count the number of wheel revolutions required to ride the calibration course.
Include partial wheel revolutions (for example through counting the number of spokes passed for the last partial revolution).
\item Calculate the wheel roll-out (meters per revolution) from step 2.
\item If you are going to use a cycle computer: enter the wheel roll-out value to the nearest millimeter in a reliable cycle computer with a wheel sensor (such as a magnet).
\item Fit the cycle computer, or a wheel revolution counter, to the same bike or unicycle used in Step 2.
\item Ride the actual race course, following the shortest possible path.
Take care to ride in the same way as in step 2.
\item Read the distance from the cycle computer, or calculate from wheel revolutions and wheel roll-out.
\item Calculate the applicable safety margin by adding up (1) $0.4\%$ of the measured distance, and (2) the resolution of the cycle computer distance readout.
\textbf{Example:} if your cycle computer shows $10.15\unit{km}$, the safety margin is $0.4\% \cdot 10.15\unit{km} + 0.01\unit{km} = 0.0506\unit{km} = 50.6\unit{m}$.
\textbf{Note:} you can skip (2) if you use a wheel revolution counter that can resolve single wheel revolutions.
\item Add the safety margin to the actual course (for example shift the start and/or finish line), to guarantee that the course is at least the advertised distance.
\end{enumerate}
Note that Steps 2 through 7 must be done without breaks.
The same rider should ride the calibration course and the race course.
The tire pressure should not be altered in the mean time.

\subsection {Distance Measurement for Free Distance Races}

In the case where a free distance is used, the course must be measured with an accuracy of plus or minus 3\% or better.
\textbf{Example:} if a race is advertised as 80\unit{km}, the actual distance must be between 77.6\unit{km} and 82.4\unit{km}.
A good consumer-type GPS unit is acceptable, provided the track shows continuous reception of sufficient satellites (no `stray' data points, or missing points).
Also acceptable is the Distance Measurement Tool of Google Maps.
A car odometer, on the other hand, might easily be off by more than 3\%, and is therefore not acceptable unless you know how to correct it.
Obviously, using a more accurate measurement is allowed, such as the method described for `Fixed Distance Races'.

\section{Accuracy of Results}

For all road race results, unless the measured time is an exact whole second, the time shall be converted and recorded to the next longer whole second, e.g.\ 1:33:47.153 shall be recorded and published as 1:33:48.

In the event that there is a tie where an award is at stake, if a photo finish system or other accurate method was used, the results of this shall be used to decide on the placings. In this case, the note (Photo Finish: +0.XX) is printed on the results list next to the official time.
In other cases it shall be determined to be a tie and the tie shall remain and gets published as such.

\textbf{Example:} If two riders have reached a time of 1:33:48 and the image of the Photo Finish System shows a difference of 0.456 seconds, the following will be printed on the result list:\\
\begin{tabular}{l l l}
1st Place & Rider 1 & 1:33:48 \\
2nd Place & Rider 2 & 1:33:48 (Photo finish: +0.456)\\
\end{tabular}
