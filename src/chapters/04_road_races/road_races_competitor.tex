\chapter{Competitor Rules}

\section{Safety}
Riders must wear shoes, gloves and a helmet (see definitions in chapter \ref{chap:general_definitions}).
Knee pads and elbow pads are suggested optional safety gear.

Personal music systems are not allowed for any races on public roads where there may be motorized traffic.

Water and food are the responsibility of the rider.
It is recommended riders carry their own water on races longer than 10k.
Hosts may offer food and water stations at their discretion.

\section{Unicycles}

Riders may use different unicycles for different racing events, as long as all comply with the rules for events in which they are entered.
See \ref{subsec:road_unicycle-changes} for the rules around using multiple unicycles in a single race.

For Unicon, if a 10 km race is organized, it must have two categories: 24 Class (including 20 Class for riders under 11 years old) and Unlimited Class.
For other conventions than Unicon, this is recommended.

For Unicon, if a 42.195 km race and/or a 100 km race is organized, it must have two categories: 29 Class and Unlimited Class.
For other conventions than Unicon, this is recommended.

Using 24 Class and smaller wheels is not allowed in races longer than 20km without express permission of the racing director.
The following chart defines the unicycle size limitations.

\begin{longtable}{|p{3cm}|p{3.5cm}|p{4cm}|p{3cm}|}
\hline
\textbf{Unicycle Class} & \textbf{Diameter Range} & \textbf{Min Crank Length} & \textbf{Transmission}\\
\hline
16 Class & 0--418mm & 89mm & regular \\
\hline
20 Class & 419--518mm & 100mm & regular \\
\hline
24 Class & 519--618mm & 125mm & regular \\
\hline
29 Class & 619--778mm & No limit & regular \\
\hline
Unlimited Class & No limit & No limit & unlimited \\
\hline
\end{longtable}
Any unicycles in question must be checked for compliance within their wheel class (wheel diameter, crank length and gearing), with the tire pressure that will be used in the race.
Preferably, this check is carried out immediately before the race.

Unless otherwise specified, it is allowed to ride in any particular Class with a unicycle that fully conforms to a smaller Class
(e.g.\ a 20 Class unicycle is allowed in a 24 Class race).

\section{Rider Identification}

Riders must wear their race number(s) fixed at the four corners, clearly visible on their chest and, when required, on their back so that it is visible during the race and as the rider crosses the finish line.
Riders must use the officially provided race number unmodified in any manner.
Numbers should not be folded, trimmed, or otherwise defaced.
Referee approval must be sought to modify a number plate if it cannot otherwise be attached securely due to hydration pack, rider physique or posture when riding.
Lost or damaged race numbers must be replaced with approval by referee.
Additionally, riders may be required to wear a chip for electronic timing.

\section{Protests}

Protests must be filed on an official form within two hours of the posting of event results.
Every effort will be made for all protests to be handled within 30 minutes from the time they are received.

\section{Event Flow}

\subsection{Riders Must Be Ready}

Riders must be ready when called for their races.
Riders not at the start line when their race begins may lose their chance to participate.
The Starter will decide when to stop waiting, remembering to consider language barriers, and the fact that some riders may be slow because they are helping run the convention.

\subsection{Starting}

Riders may start mounted, holding onto a starting post or other support, or onto each other.
Riders may mount after the start signal, if they wish.

Usually, a start-beep apparatus is used.
This provides a six-count start: ``beep - beep -beep - beep - beep - buup!''
The timing between (the start of) successive beeps is one second.
The first five beeps have all the same sound frequency.
The final tone (buup) has a higher frequency, so that the competitors can easily distinguish this tone from the rest.
The proper moment to start is the beginning of the final tone.

As an alternative, the Starter will give a three-count start before firing a starting gun on the fourth count.
Example: ``One, two, three, BANG!''
The time between each of these elements should be the same, and approximately 3/4 seconds.
This allows riders to predict the timing of the gun, for a fair start.

Riders start with the fronts of their tires (forward most part of wheel) behind the edge of the starting line that is farthest from the finish line.
Rolling starts are not permitted in any race.
Riders may start from behind the starting line if they wish, provided all other starting rules are followed.
Riders may lean before the start, but their wheels may not move forward during the start beeps or counting down.
Rolling back is allowed.

\subsection{False Starts}

A false start occurs if a rider's wheel moves forward before the start signal, or if one or more riders are forced to dismount due to interference from another rider or other source.
In any case, only the earliest false starter will be assigned a false start, with the associated penalty (warning, time penalty or disqualification).

There are three options on how to deal with false starts:
\begin{itemize}
\item \textbf{One False Start Allowed Per Rider:}\\
In case of a false start, the heat is restarted.
Any rider(s) who caused their personal first false start may start again.
Any rider(s) causing their personal second false start are disqualified.
\item \textbf{One False Start Allowed Per Heat:}\\
In case of a false start, the heat is restarted.
For the first false start of a particular heat, all riders may start again.
Thereafter, any rider(s) causing a false start are disqualified.
\item \textbf{Time Penalty:}\\
In case of a false start, the heat is not restarted.
If a false start occurs by one or multiple riders, these riders receive a time penalty (such as 10 seconds).
\end{itemize}
If a heat has to be restarted, the Starter will immediately recall the riders, for example by firing a gun or blowing a whistle or any other clear and pre-defined signal.

If the race is started using individual starts or heat starts (see sections \ref{subsec:road_heat-assignment_individual-start} and \ref{subsec:road_heat-assignment_heat-start}) a time penalty is the recommended option.
In the case of a mass start (section \ref{subsec:road_heat-assignment_mass-start}), any option is viable.

\subsection{Passing}

An overtaking rider must pass on the outside, unless there is enough room to safely pass on the inside.
Riders passing on the inside are responsible for any fouls that may take place as a result.
No physical contact between riders is allowed.
The slower rider must maintain a reasonably straight course, and not interfere with the faster rider.

\subsection{Dismounts}

Dismounting and remounting is allowed.
If a rider is forced to dismount due to a fall by the rider immediately in front, it is considered part of the race, and both riders should remount and continue.

\subsection{Illegal Riding}

Illegal riding includes intentionally interfering in any way with another rider, deliberately crossing in front of another rider to prevent him or her from moving on, deliberately blocking another rider from passing, or distracting another rider with the intention of causing a dismount.

\subsection{Repair, Change, or Replace a (Broken) Unicycle \label{subsec:road_unicycle-changes}}
In Road Races, riders may make modifications to their unicycles, but must be self-sufficient in this.
The rider must carry all necessary parts and tools needed for the modification(s), and do all the work without any assistance.
For example, a rider may change cranks but must carry the new cranks and all tools from the start of the race.

Assistance is allowed in the event of a breakdown or damage to the unicycle.
Outside tools and hands-on help may assist the rider to continue, including replacing the unicycle if necessary.
The Referee must confirm that the situation was unplanned and was indeed ``accidental''.
If the Referee determines otherwise and the rider used outside assistance for changes to the unicycle, the rider will be disqualified.

The rider may continue the course on foot (walking, not running) with the broken unicycle.
If the rider exits from the course, they must reenter the course at or before the point where they exited from the course.
When the rider is off course, he may run or use any other form of transportation.

Any modifications made to the unicycle must still adhere to the requirements of the category that the rider is entered in.
For example, if a rider broke a crank in the 24 Class 10k race, they are only allowed to install a new crank of 125 mm or longer.


\subsection{Finishes}

Finish times are determined when the front of the tire first crosses the vertical plane of the nearest edge of the finish line.

Riders are always timed by their wheels, not by outstretched bodies.
If riders do not cross the line in control, they are awarded a 5 second penalty to their time.
``Control'' is defined by the front of the wheel crossing the vertical finish plane (as defined above) with the rider having both feet on the pedals.
(Note: a rider is not considered in control if the unicycle crosses the finish line independent of the rider.
The finish time is still measured by when the wheel crosses the vertical finish plane and the 5 second penalty is applied.)

In the case where a rider is finishing with a broken unicycle, the rider must bring at minimum the wheel to the finish line, and time is still taken when the wheel crosses the finish line.
The 5 second penalty is applied.

\section{Criterium}
A Criterium race is a short road race with distances of 5k to 10k.
Courses should have left and right turns and multiple laps.
