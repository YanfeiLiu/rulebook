\chapter{Event Organizer Rules}

\section{Venue}

It will be a multi-lap event featuring a bit of cross country trail, grassy fields and natural and man-made obstacles where dismounting will be necessary.
Courses should be designed such that the course does not favor any type of unicycle.
A course should have no fewer than two and no more than six obstacle or barrier sections where riders normally dismount and run with the unicycle.
The starting and finishing stretches shall be free of obstacles within 10 meters.
The course should be designed to avoid bottlenecks and give riders enough room to pass each other, especially after the start.

It is suggested that the length of the course not be much shorter than 1 km in length and no longer than 2.5 km in length.
Organizers should keep in mind that most of the course should be visible from several vantage points.

\section{Officials}

The host must designate the following officials for each cyclocross race:
\begin{itemize}
\item Cyclocross Director
\item Referee
\end{itemize}

\section{Communication}

\begin{comment2016}
Some thoughts on what might need to be communicated:
\begin{itemize}
\item age groups
\item results
\end{itemize}
\end{comment2016}

The cyclocross event is exempt from the rule about early publication of course details.
This is because the cyclocross course is typically set up immediately before the race.

\section{Age Groups}

Age groups must be offered as male and female age group.
There must not be any age group specific restrictions on equipment.
The following age groups are the maximum allowable for muni competitions:

\begin{tabular}{|l|l|}
\hline
Under 15 & Youth \\
\hline
15-16 & Juniors \\
\hline
17-18 & Rookies \\
\hline
19-29 & Elites \\
\hline
30-49 & Masters \\
\hline
50+ & Veterans \\
\hline
\end{tabular}

\section{Course Availability for Practice}

The cyclocross course must be available for practice at least one hour immediately prior to a cyclocross event.
This will ensure that course setters do not have an advantage over other riders, and may compete.

\section{Race Configuration}

It is suggested that the Elite race be close to 45 minutes in length and the Beginner/Intermediate race be close to 30 minutes in length.
Using the time from the top rider's first two laps, the referee will determine how many laps could be completed in the desired time limit (e.g.\ 45 minutes).
From this point on, the number of remaining laps (for the leaders) will be displayed and this will be used to determine when finish of the race occurs.
A bell will be rung with one lap to go.

Lapped riders in the race will all finish on the same lap as the leader and will be placed according to the number of laps they are down and then their position at the finish.
