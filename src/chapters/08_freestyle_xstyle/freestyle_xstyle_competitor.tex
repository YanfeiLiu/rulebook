\chapter{Competitor Rules}

\section{Safety}

No safety gear is needed.

\section{Unicycles}

Any type and any number may be used.

\section{Rider Identification}

No rider identification is required.

\section{Protests}

Protests must be filed on an official form within two hours of the posting of event results.
Every effort will be made for all protests to be handled within 30 minutes from the time they are received.

\begin{comment2016}
There is no section under X-Style, so it inherits the rule from Chapter 1.
Does X-Style need a rule to allow protests between the announcement of a battle/preliminary round results and the start of the next round? Probably we should ask Gossi about this.
\end{comment2016}

\section{Event Flow}

\subsection{Definitions}
\textbf{Unicycle Skill}: The main action of a movement is done on and with the unicycle.

\emph{Example 1}: If you do a handstand, while the unicycle is laying on the floor, this is a non-unicycle trick, because the main action is the handstand.

\emph{Example 2}: If you do a handstand on the unicycle, while it is upright with only the tire on touching the ground, this is a unicycle trick, because the main action is balancing the unicycle while doing the handstand.

\subsection{Music}

A DJ plays music for each run.
Riders can also hand in their own music.

Any performance music must be provided on CD, or only those other media types supported by the event host.

\subsection{Starting Groups}
The group of competing riders is split randomly into starting groups of about the same size with a maximum 10 riders.

\subsection{Runs}

Every competitor has a run in which the rider shows his/her skills.
After the run the DJ waits until the current skill is finished and stops the music.
Riders can ask the DJ to give a time-remaining notification during his/her run.

\subsection{Run Length}

The length of a competitor's run is determined by the round:
\begin{itemize}
\item Final Round: 2 minutes \\
\item Semi-final round: 1 min 30 seconds \\
\item Previous rounds: 1 minute \\
\end{itemize}

If fewer there are three rounds, use only the run lengths that are relevant.

\textbf{Example:} If there is one round, it is considered the final round, and would be 2 minutes in length.

In smaller competitors, the director may alter the run lengths due to time constraints.

\subsection{Advancement}
The best 3 riders of each group advance to the next round, forming a new pool of competitors.
If there is a tie on one of the first 3 places which extends to more than the 3 riders, all involved riders advance.
This pool again gets split into starting groups and the next round begins.
The final round consists of only one starting group.
In the final round the best 3 riders are awarded for the first, second and third place of the competition.

\subsubsection{Ties}
If the competition doesn't allow ties (e.g.\ Unicon), the tied placed riders of the final round will be given another run.
The run will be one minute.
The judges bring all tied riders in order.
If there is still a tie between riders, the tie rules will be applied again, until all ties are resolved.
