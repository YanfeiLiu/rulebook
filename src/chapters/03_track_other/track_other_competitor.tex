\chapter{Competitor Rules}

\section{Safety}
Helmets are required for Downhill Gliding.

\section{Event Flow}


\subsection{Track Coasting}
An event to determine which rider coasts the furthest distance.
There is a 30 meter speed-up distance.
Riders' coasting distances are measured from a `starting line' with a 5 meter minimum, which will be marked by a `qualifying line.'
If the rider does not cross the qualifying line it will count as a failed attempt.
The farthest distance from the line wins.
The distance is measured to the rearmost part of the rider that touches the ground when dismounting, or to the tire contact point where the rider stops coasting.
Remounting is not allowed.
Riders get two attempts.
If a rider crosses the coasting line (tire contact point) not in coasting position, he or she is disqualified in that attempt.
The riding surface should be as smooth and clean as possible, and it may be straight or curved.
This event is held only on a track, or a very level, smooth surface.
Ample time must be allowed for all competitors to make some practice runs on the course before the official start.
Crank arm rules do not apply.
This event is held only on a track, or a very level, smooth surface.
Wind must be at a minimum for records to be set and broken.

\subsection{Track Glide}

In Gliding, the balance has to be kept all the time by the braking action between one or both feet and the top of the tire.
If, for example, the foot loses contact with the tire due to small bumps, the contact must be restored immediately.
The rules are the same as for the coasting events above, with the addition that the riding surface must be dry.
It is held on a track with the same rules as Track Coasting (see above).

\subsection{Downhill Glide}
In Gliding, the balance has to be kept all the time by the braking action between one or both feet and the top of the tire.
If, for example, the foot loses contact with the tire due to small bumps, the contact must be restored immediately.
A downhill race for speed.
Riders start from a standstill, or speed up to the `starting line.'
Riders are timed over a measured distance to the finish line.
Dismounts before the finish line disqualify the rider in that attempt.
Helmets are mandatory.

\subsection{Slow Balance Forward}

In Slow Balance Forward, the rider rides a distance of 10 meters in a continuous forward motion as slowly as possible without stopping, going backward, hopping or twisting more than 45 degrees to either side.
Any age group with riders of 11 years or older must use a board of 15 cm wide.
Any age group with no riders of 11 years or older must use a board of 30 cm wide at Unicon; in other conventions the host may choose to use either a 15 cm wide board or a 30 cm wide board for this age group.
Tires may overlap the edges of the board, but if the tire contacts the ground next to the board, that would be the end of that attempt.
There are no crank arm length or wheel size restrictions for this event.

Riders must wear shoes.
No other safety gear is required.

\subsubsection{Timing}
The position of the unicycle during Slow Balance is defined by the tire contact point.
In Slow Balance, the rider starts behind the starting line.
On command by the starter, the rider has 10 seconds to start forward motion and let go off the starting post.
The timer starts recording time when the tire contact point crosses the starting line.
At this moment, the rider may not be in contact with the starting post anymore.
Timers must watch the hands and the feet/wheel at the same time at that moment.
The time stops when the tire contact point crosses the finish line.

\subsubsection{Optional Penalty Rules}
At any bigger conventions where there is a large pool of judges (such as Unicon) it is recommended that the host uses a system wherein the judges may give penalties to riders who seem to make ``micro-errors'' or if the judges are in doubt whether an error was made.
Examples of micro-errors are twisting about 46 or 48 degrees, or vibrations of the wheel.
Each penalty subtracts one second from the ridden time.
Riders are still disqualified for clear errors, such as riding off the board, dismounting or twisting 90 degrees.
Using these penalty rules is especially discouraged for possible errors for which a reliable objective detection system is being used.

\subsubsection{Age Group and Final Rounds}
Age Group and Final rounds are always required.

\textbf{Age Group Round:}
\begin{itemize}
\item All riders must participate in the Age Groups.
Riders get two attempts.
\item The best 8 female and the best 8 male riders qualify for the finals.
\item For Unicon a minimum of 20 seconds is required to achieve a valid result.
For any age group with no riders of 11 years or older the minimum time is 15 seconds.
Riders who don't reach this threshold are automatically disqualified.
If your net time after penalties brings you below the minimum time, you are also disqualified.
For other competitions than Unicon, the host may adjust the threshold to a lower time or have no threshold at all.
\end{itemize}

\textbf{Final Round:}
\begin{itemize}
\item The Judging team for the Finals must consist of a single group of people that watch every rider, or (insofar available) an accurate and reliable technical means to check adherence to the rules.
\item Riders get two attempts.
\item The champion is the rider who performs the best result in the final round.
\end{itemize}

\subsection{Slow Balance Backward}
This is the same as Slow Balance Backward, with the following differences \textit {in italic}:
\begin{itemize}
\item Riders ride \textit{backward}.
\item It is an error to ride \textit{forward}.
\item For Unicon a minimum of \textit{15 seconds} is required to achieve a valid result.
For any age group with no riders of 11 years or older the minimum time is \textit{10 seconds}.
\item Any age group with riders of 11 years or older must use a board of \textit{30 cm} wide.
Any age group with no riders of 11 years or older must use a board of \textit{60 cm} wide at Unicon; in other conventions the host may choose to use either a \textit{30 cm} wide board or a \textit{60 cm} wide board for this age group.
\end{itemize}

\subsection{Stillstand}
Stillstand is a competition in which the rider attempts to balance as long as possible.
The rider cannot hop or turn the tire more than 45 degrees, and must remain on a 25 cm long, 10 cm wide, and 3 cm tall block of wood.
The competition should take place indoors on a level surface
The only required safety gear is shoes.

Each participant has 2 attempts that can be done at any time during the time window set by the host.
The host can decide to add to each of the 2 attempts a window up to 20 seconds, in which the competitor can start the number of tries needed.

The starting post is placed anywhere the participant prefers.
Time starts running when the competitor lets go of the starting post.
After time starts running, the starting post will be taken away.
Time stops at the moment when the participant rides off the board, dismounts, starts hopping or turns the tire more than 45 degrees.

There are no finals for the Stillstand competition.
The overall results will be determined by the best results for males and females respectively.

\subsection{Other Wheel Sizes}
The host can choose to offer additional track events based upon other wheel size requirements.
Two examples include 700c racing and Unlimited.
Exclusive of unicycle requirements, all other track racing rules apply.

An unlimited race is one in which there are no unicycle size restrictions.
Any size wheels, any length crank arms, giraffes or any types of unicycles (see definition in chapter \ref{chap:general_definitions}) are allowed.

In the 700c wheel category, unicycle wheels must be greater than 618mm in diameter, have a maximum bead seat diameter (BSD) of 622 mm, and there are no restrictions on crank length.
