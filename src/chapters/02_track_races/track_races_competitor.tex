\chapter{Competitor Rules}

\section{Safety}

Riders must wear shoes, knee pads and gloves (definitions in chapter \ref{chap:general_definitions}).

Riders on wheels larger than 24 Class (or with gearing) must also wear helmets.

\section{Unicycles}

Only standard unicycles may be used.
Riders may use different unicycles for different racing events, as long as all comply with the rules for events in which they are entered.

For events divided by wheel size, there is a maximum allowable tire diameter and minimum crank arm length for each category:

\begin{longtable}{|p{3cm}|p{3cm}|p{4cm}|p{3cm}|}
\hline
\textbf{Unicycle Class} & \textbf{Max Diameter} & \textbf{Min Crank Length} & \textbf{Transmission}\\
\hline
16 Class & 418mm & 89mm & standard \\
\hline
20 Class & 518mm & 100mm & standard \\
\hline
24 Class & 618mm & 125mm & standard \\
\hline
29 Class & 778mm & No limit & standard \\
\hline
Unlimited Class & No limit & No limit & unlimited \\
\hline
\end{longtable}

For any tire in question, its outside diameter must be accurately measured.

Crank arm length is measured from the center of the wheel axle to the center of the pedal axle.
Longer sizes may be used.

In all track racing events on standard unicycles, shoes must not be fixed to the pedals in any way (no click-in pedals, toe clips, tape, magnets or similar).

\section{Rider Identification}

Riders must wear their race number clearly visible on their chest so that it is visible during the race and as the rider crosses the finish line.
Additionally, the rider may be required to wear a chip for electronic timing.

\section{Protests}

Protests must be filed on an official form.
Mistakes in paperwork, inaccuracies in placing, and interference from other riders or other sources are all grounds for protests.
All Referee decisions are final, and cannot be protested.
For a large event such as Unicon or continental championships, the default protest time is 60 minutes (counting from the posting of results), the minimum is 30 minutes.
For smaller events, the default protest time is 30 minutes, the minimum is 15 minutes.
Every deviation from the default protest time has to be clearly announced when the results are posted, including stating the protest deadline on the results list itself.
The protest time may be extended for riders who have to be in other races during the protest period.
All protests will be acknowledged within 30 minutes from the time they are received, and an effort will be made to settle the issue within those 30 minutes.

\section{Wheel Size Categories}

Wheel sizes for track racing are 20 Class, 24 Class and 29 Class.
Additional groups for 16 Class or other wheels can be added.
When not otherwise specified, 24 Class is the maximum wheel size above age 10.
For age groups with a maximum age of 10 or younger, the maximum wheel size is 20 Class (or smaller, if smaller sizes are also used).
The youngest age group for 24 Class wheels should have a minimum age of 0, so riders 10 and younger have the option of racing on 24 Class with those groups (e.g.\ 0-13 or 14-16).

\section{Event Flow}

These races should be part of every Unicon:

\subsection{100m Race}

In the 100m race, riders must stay in their lane.

\subsection{400m Race}

The 400m race is started with a stagger start, where riders are started in separate lanes, at separate locations.
In the 400m race, riders must stay in their lane.

\subsection{800m Race}

There are two different ways to run an 800m race, remounting after a dismount is allowed in both ways:

\textbf{1. 800m Race with Stagger Start:} Riders are started in separate lanes, at separate locations.
The race shall be run in lanes as far as the nearer edge of the break line where riders may leave their respective lanes.
The break line shall be an arced line marked after the first bend across all lanes other than lane 1.
To assist athletes identify the break line, halved tennis balls can be placed on the lane lines immediately before the intersection of the lines and the break line.
After the break line, non-lane racing rules apply.

\textbf{2. 800m with Waterfall Start:} Riders are started at a curved starting line that places all riders an equal distance from the first turn.
If a waterfall start is used, non-lane rules apply from the start.

\subsection{One Foot Race}
The distance of the One Foot Race is 50m.
Riders may pedal with both feet for the first 5 meters, but must be pedaling with only one foot after crossing the 5m line.
The non-pedaling foot must have left the pedal when the tire contact point crosses the 5m line on the track.
The non-pedaling foot may or may not be braced against the unicycle fork.

\subsection{Wheel Walk Race}

Riders start mounted, with one or both feet on the tire, and propel the unicycle only by pushing the tire with one or both feet.
No contact with pedals or crank arms is allowed.
No crank arm restrictions.
Riders in age groups with a maximum age of 10 or younger will race a 10m Wheel Walk.
All other riders will race a 30m Wheel Walk.

\subsection{Riders Must Be Ready}

Riders must be ready when called for their races.
Riders not at the start line when their race begins may lose their chance to participate.
The Starter will decide when to stop waiting, remembering to consider language barriers, and the fact that some riders may be slow because they are helping run the convention.

\subsection{Starting \label{subsec:track-field_starting}}

This procedure is used for all Track Races, unless noted otherwise.

Riders start mounted, holding onto a starting post or other support.
Unicycle riders need to be leaning forward before the starting gun fires, so the Starter will give a four-count start.
Example: ``One, two, three, BANG!''
This allows riders to predict the timing of the gun, for a fair start.

As an alternative a start-beep apparatus can be used.
In that case we have a six-count start.
Example: ``beep - beep -beep - beep - beep - buup!''
The timing between beeps is one second.
The first 5 beeps have all the same frequency.
The final tone (buup) has a higher frequency, so that the racer can easily distinguish this tone from the rest.
The proper moment to start is the \textit{beginning} of the final tone.

Riders start with the fronts of their tires (forward most part of wheel) behind the edge of the starting line that is farthest from the finish line.
Rolling starts are not permitted in any race.
However, riders may start from behind the starting line if they wish, provided all other starting rules are followed.
Riders may lean before the gun fires, but their wheels may not move forward at any time.
Rolling back is allowed, but nothing forward.
Riders may place starting posts in the location most comfortable for them, as long as it doesn't interfere with other riders.

\subsection{False Starts}

A false start occurs if a rider's wheel moves forward before the start signal, or if one or more riders are forced to dismount due to interference from another rider or other source.

\subsection{Lane Use}

In most races, a rider must stay in his or her own lane, except when the rider has to swerve to avoid being involved in a crash.
In all other cases, a rider who goes outside their lane is disqualified.
Going outside a track lane means that the tire of the unicycle touches the ground outside his assigned lane.
Riding on the marking is allowed.
No physical contact between riders is allowed during racing.

\subsection{Passing in Non-Lane Races \label{subsec:track-field_lane-use_non-lane-races}}

This applies to 800m and other events without lanes.
No physical contact between riders is allowed.
In track races, an overtaking rider must pass on the outside, unless there is enough room to safely pass on the inside.
Riders passing on the inside are responsible for any fouls that may take place as a result.
Riders must maintain a minimum of one (24 Class) wheel diameter (618 mm as judged by eye) between each other when passing, and at all other times.
This is measured from wheel to wheel, so that one rider passing another may come quite close, as long as their wheels remain at least 618 mm apart.
The slower rider must maintain a reasonably straight course, and not interfere with the faster rider.

\subsection{Dismounts}

A dismount is any time a rider's foot or other body part touches the ground.
Except for the 800m, Relay races, and other races where this is announced in advance, after a dismount the race may not be continued and will be considered as not finished (DNF - Did Not Finish).
In races where riders are allowed to remount and continue, riders must immediately remount at the point where the unicycle comes to rest, without running.
If a dismount puts the rider past the finish line, the rider must back up and ride across the line in control, in the normal direction.

\subsection{Assisting Racers}

In races where riders are allowed to remount, the riders must mount the unicycle completely unassisted.
Spectators or helpers may help the rider to his or her feet and/or retrieve the dropped unicycle, but the rider (and the unicycle) may not have any physical contact with any outside object or person, including a starting block under the wheel, when mounting.

\subsection{Illegal Riding}

This includes intentionally interfering in any way with another rider, deliberately crossing in front of another rider to prevent him or her from moving on, deliberately blocking another rider from passing, or distracting another rider with the intention of causing a dismount.
A rider who is forced to dismount due to interference by another rider may file a protest immediately at the end of the race.
Riders who intentionally interfere with other riders may receive from the Referee a warning, a loss of placement (given the next lower finishing place), disqualification from that race/event, or suspension from all races.

\subsection{Second Attempt After Hindrance or Interference}

If a rider is hindered due to the actions of another rider, or outside interference, either during the start or during the race, he or she may request to make a second attempt.
The Referee decides if the request is granted.
A second attempt must not be granted to a rider who is disqualified based on something that happened before they were hindered.

No complete definition of hindrance or interference can be given, but it does include cases where a rider swerves, hesitates and/or decelerates because this is arguably necessary in order to avoid a crash or potential crash.

If the request is granted, the Referee has two options:

\textbf{Option 1:}
Re-run the whole heat in question.\\
In general, this option is preferred only if the heat includes the fastest riders within an age group.
For the other riders in the heat, riding again is optional.
If they decide to ride again, they agree to discard their previous result.
If they don't ride again, their previous result stands.
If none of the other riders want to ride again, the Referee reverts to option 2.

\textbf{Option 2:}
Do any of (a), (b) or (c), depending on the conditions.\\
In general, this option is preferred if the heat in question did not include the fastest riders within an age group:
\begin{enumerate}[label=(\alph*)]
\item If possible, the rider is added to an upcoming heat in his own age group; or
\item If possible, the rider is added to an upcoming heat in another age group; or
\item If none of the above is possible, the rider does his second attempt in a dedicated heat.
\end{enumerate}
In option 2, the rider decides if he wants company or not.
He can pick the riders, but cannot hold up the proceedings to wait for them if other riders are available.
The Referee has the final say as to which extra riders are allowed to participate in such a heat.
It must be stated clearly to any accompanying riders that their result is not official.

In all cases, if the hindered rider is allowed to do a second attempt and decides to do so, the first run is canceled and only the second run counts regardless of the result.
In the case where a second attempt was incorrectly granted, for example when the rider was disqualified based on something that happened before the hindrance in question occurred, the result of the second attempt for that rider does not count and the result from the first run stands.

In non-lane races, if a rider is forced to dismount due to a fall by the rider immediately in front, it is considered part of the race -- not a reason to grant a second attempt -- and all riders involved may remount and continue.
The Referee can override this rule if intentional interference is observed.

\subsection{Finishes \label{sec:track-field_finishes}}

The finish moment is when the front of the tire crosses the finish.
The exact location of the finish is the edge of the finish line that is nearest to the starting line.
Riders are thus not timed by outstretched bodies.
At the finish moment, riders must be mounted and in control of the unicycle.
``Control'' is defined as follows:
\begin{enumerate}[label=(\alph*)]
\item in regular races: the rider has both feet on the pedals; or
\item in one-foot races: the rider has one foot on a pedal; or
\item in wheel walk races: the rider continues to wheel walk.
\end{enumerate}
In races where dismounting is allowed (800m, Relay, etc.\@), in the event that a rider does cross the finish line but not in control, the rider must back up on foot, remount and ride across the finish line in control.
In races where dismounting is not allowed, the rider is disqualified.

\section{Finals}

\begin{comment2016}
This section is wordy, and should be rewritten.
Some text belongs elsewhere.
\end{comment2016}

At Unicons, a `final' must be held for each of the following races: 100m, 400m, 800m, One Foot, Wheel Walk, and IUF Slalom.
For any other Track \& Field discipline, a `final' may be held at the discretion of the organizer, after all age group competition for that discipline has been completed.

For disciplines that are run in heats, such as 100m races or relay races, this will take the form of a final heat.
For disciplines that are not run in heats, such as IUF slalom or slow race, the final will take the form of successive attempts by the finalists.

The riders posting the best results regardless of age in the age group heats are entitled to compete in the final.
They can be called ``finalists''.
For each final, the number of finalists (finalist teams in case of relay) will be eight, unless for an event that uses lanes, the number of usable lanes is less than eight.
In that case the number of finalists equals the number of usable lanes.
Finals are composed regardless of age group, but male and female competitors are in separate finals.

Finals are subject to the same rules as age group competition, including false start rules and number of attempts.

The best result in a final determines the male or female Champion for that discipline (World Champion in the case of Unicon).

If a finalist disqualifies, gets a worse result, or doesn't compete in the final, his/her result in age group competition will still stand.
The male and female winners of the finals will be considered the Champions for those disciplines, even if a different rider posted a better result in age group competition.
Speed records can be set in both age group competition and finals.

In disciplines for which no finals are held, finalist status will still be awarded on the basis of results in age group competition.
Accordingly, riders posting the best results in each discipline are the Champions for that discipline.

\section{IUF Slalom}

\begin{figure}[h]
\begin{center}
\includegraphics{iuf_slalom}
\end{center}
\vspace{-20pt}
\caption{IUF Slalom Course \label{fig:iuf_slalom}}
\vspace{-10pt}
\end{figure}
Pictured here is the IUF Slalom, in which you must ride around 10 cones in the correct pattern.
Arrows marked on the ground should indicate the direction of the turns for riders unfamiliar with the course.
The rider has to start directly behind the Start line.
The Starter gives the opening, and then the competitor has to start during the next 3 seconds.
The timer is started when any defined point of the tire (for example the part that crosses a low light beam) crosses the start line, and stops when a similar point of the tire crosses the finish line.
If the rider has not yet started after 3 seconds, the timer will start counting anyway.
The rider is not disqualified for this.
Time measurement at start and finish line must be identical to insure accurate time measurement.
It must be secured that riders do not gain momentum before crossing the start line (no flying starts).
Remounting is not allowed.
Cones may be hit, but not knocked over.
The course must be followed correctly, including the direction of turns.
The last cone must be completely circled before the rider's time is taken at the finish line.
Riders who go the wrong way around a cone can go back and make the turn the correct way with the clock still running.
The cones used are plastic traffic cones.
For official competition, cones must be between 45 and 60 cm tall, with bases no more than 30 cm square.
The course must be set up accurately.
The proper positions of the cones should be marked on the ground for a cone to be replaced quickly after it has been knocked over.
Riders get two attempts.
