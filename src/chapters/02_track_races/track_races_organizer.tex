\chapter{Event Organizer Rules}

\section{Venue}

A track must be made available for conducting the track races.
The track must be marked in meters, and should be prepared in advance with start and finish lines for the various racing events that are unique to unicycle racing (such as 50, 30, 10 and 5 meter lines).
In addition to the track, a smooth area of sufficient size must be set aside to run the IUF Slalom.
A public address system must be provided to announce upcoming events and race winners.
Bullhorns are usually not adequate for the track environment.

If the track is outdoors, plans must be made to deal with inclement weather.
Using an indoor track can eliminate this problem.
The track must be available for enough days to allow for inclement weather.

\section{Officials}

The host must designate the following officials for track racing:
\begin{itemize}
\item Track Director
\item Referee
\item Starter
\item Finish Line Judges
\end{itemize}

\section{Communication}

\begin{comment2016}
Some thoughts on additional required communication:
\begin{itemize}
\item any changes to the standard rules
\item false start method
\item cut-off time
\item age groups
\item whether the event qualifies for a world record
\item whether helmets are required
\end{itemize}
\end{comment2016}

\begin{comment2016}
There should be a rule about when and how results are posted, so that the
protest period can begin.
\end{comment2016}

A Host is allowed to make helmets and/or knee pads mandatory for track races but it must be announced when registration is opened and must appear as an extra point to check for each discipline the competitor registers for.

\section{Age Groups \label{subsec:track-field_racing-categories_age-groups}}
The following age groups are the minimum required by the IUF to be offered at the time of registration for any Track \& Field discipline: 0-10 (20 Class), 0-13, 14-18, 19-29, 30-UP.
Convention hosts are free to offer more age groups, and often do.
For example, a full range of offered age groups might look like 0-8 (20 Class), 9- 10 (20 Class), 0-12, 13-14, 15-16, 17-18, 19-29, 30-39, 40-49, 50-59, 60-UP.
All age groups must be offered as male and female age group.

\begin{comment2016}
\section{Practice}

Does there need to be a rule to allow practice on the track?
\end{comment2016}

\section{Minimum Racing Events \label{sec:track-field_minimum-racing-events}}
The following races: 100m, 400m, 800m, One Foot, Wheel Walk, and IUF Slalom, are to be part of every Unicon.
Convention hosts are free to add more racing events.

\section{Track Combined Competition}%comment2016 is this in the right place?
The best finishers combined from the 6 racing events listed above will win this title.
Points are assigned for placement in each of the above races, based upon best times in the final heats.
In smaller events, the finishing age group times in the IUF Slalom can be used if no additional final is run.
1st place gets 8, 2nd place 5, 3rd place 3, 4th place 2, and 5th place 1.
Highest total points score is the World Champion; one each for male and female.
If there is a tie, the rider with the most first places wins.
If this still results in a tie, the title goes to the better finisher in the 100m race.
Points are not earned in age group heats.

\section{Race Configuration}

Racing competition is held in two separate divisions: Male and Female.
No heat of any race shall be composed of both male and female riders without the approval of the Racing Referee.

There will be no mixing of age groups, or males and females, in heats except with permission from the Racing Referee.

Track events must have both a preliminary and final round.

\section{Lane Assignments}

At some conventions, lanes are preassigned at time of registration.
The following heat and lane assignments must be used for Unicon and international competitions.
Also for other competitions it is recommended to do the assignments accordingly.
The rule is applied for each age group independently.
\begin{enumerate}
\item The riders with the fastest seed times will be placed in the last heat, the next riders in the second last heat, etc.\ until all riders are distributed over the heats.
\item In age group races the distribution is done according to the seed times.
Riders for whom no seed times are given will be placed without time behind the rider with the slowest seed time.
The order in which riders with the same time are seeded will be decided by lot.
\item In final races the distribution is done according to the times achieved in the age group races.
The order in which riders with the same time will be seeded shall be decided by lot.
\item The lane assignment in lane-bound races is carried out according to the subsection below.
\item In each heat at least three riders should be seated if possible; however, this number can be undercut due to cancellations.
\end{enumerate}

\subsection{Lane assignments in lane-bound races}

\begin{enumerate}
\item In races up to and including 100m the lanes are to be distributed as follows in each heat:
  \begin{itemize}
  \item[$-$] If the number of lanes is odd, the rider with the fastest seed time in the race will be placed on the middle lane.
  The rider with the next fastest seed time will be placed on the lane to the right of the middle lane (number of the middle lane +1) and all other riders will be placed alternately to the left and right of the middle lane according to their seed times.
  \item[$-$] If the number of lanes is even, the rider with the fastest seed time will be placed on the lane with half the number of lanes.
  The rider with the next fastest seed time will be placed to the right of this lane (half lane number +1) and all other riders will be placed alternately to the left and right according to their seed times.
  \end{itemize}
\item In races from 200m the lanes are to be distributed as follows in each heat:
  \begin{itemize}
  \item[$-$] The rider with the fastest seed time in the race will be placed on lane 1, the rider with the next fastest seed time will be placed on lane 2 and all other riders will be placed one lane higher according to their seed times.
  \end{itemize}
\item If a lane cannot be used, due to poor quality or other reasons, skip it and proceed as described above.
\end{enumerate}

Example for the seeding of a 8-lane track from faster to slower seed time:\\
100m and shorter: 4,5,3,6,2,7,1,8\\
200m and longer: 1,2,3,4,5,6,7,8

\section{Optional Race-End Cut-Off Time}
It may be necessary to have a maximum time limit for long races, to keep events on schedule.
When this is planned in advance, it must be advertised as early as possible, so attending riders will know of the limit.
Additionally, at the discretion of the Racing Director, a race cut-off time may be set on the day of or during an event.
The purpose of this is to allow things to move on if all but a few slow racers are still on the course.
These cut-offs need not be announced in advance.
At the cut-off time, any racers who have not finished will be listed as incomplete (no time recorded, or same cut-off time recorded for all).
Optionally, if there is no more than one person on the course per age category and awards are at stake, they can be given the following place in the finishing order.
But if each participating age category has had finishers for all available awards (no awards at stake), there is no need to wait.

\section{Timing, Photo Finish and False Start Monitoring}

A Fully Automatic Timing and Photo Finish System must be used for the track races at Unicon and is strongly recommended for track races at all other competitions.
The system must have been tested, and have a certificate of accuracy issued within 4 years of the competition, including the following:
\begin{enumerate}
\item The System must record the finish through a camera positioned in the extension of the finish line, producing a composite photo finish image of at least 100 images per second, ideally 1000 images per second. The image must be synchronized with a uniformly marked time-scale graduated in 0.01 seconds.
\item The System shall be started automatically by the Starter’s signal, so that the overall delay between the start signal and the start of the timing system is constant and equal to or less than 0.001 second.
\end{enumerate}

For the track races at Unicon a false start monitoring system, which is able to reliably detect a crossing of the start line before the start signal, must be used and is strongly recommended for track races at all other competitions.
