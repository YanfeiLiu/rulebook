\chapter{Competitor Rules}

\section{Safety}

No safety gear is needed.

\section{Unicycles}

Any type and any number.

\section{Rider Identification}

No rider identification is required.%comment2016 Can Expert and Jr remove wristbands? Who decides? Or always allowed.

\section{Protests}


Protests must be filed on an official form within 15 minutes of the posting of event results.
Protest is only possible against mistakes in calculation or other mistakes not connected to a judge's subjective score.
Protest against judges' scores is not permissible.
The Chief Judge must resolve all protests within 30 minutes of receipt of the written form.

\section{Deadline For Signing Up}

\begin{comment2016}
What is allowed in the case of injury closer than 36 hours?
\end{comment2016}

These events have a deadline for entry, which must be specified in the registration form.
If not specified in the registration form, the deadline is one month before the official convention start date.
A maximum of ten Individuals, ten Pairs routines, and two groups will be allowed to be added after this time to account for difficulties in travel planning or other valid reasons that are communicated about in advance.
These will be added in the order of their request to the Chief Judge and Convention Director via email or fax.
Participants who attempt to sign up less than 36 hours prior to the beginning of the specified competition will not be allowed to enter.
Changing Pairs partners is allowed up to 36 hours prior to the actual competition as long as the category does not change.
Adding or subtracting the members of a group routine is allowed up to 36 hours prior to the start of that competition.

\section{Competitions}

\subsection{Group Freestyle Restrictions}
Group Freestyle is divided in Large Groups and Small Groups
Each rider may enter each competition (Small Group, Large Group) only once.
For example: a rider can be in one small group routine and one large group routine but not two small group routines.

A rider may appear in a second Group Freestyle performance (Small Group, Large Group) with permission of the Chief Judge, to replace a rider due to illness, injury or other mishap.

\subsubsection{Small Group}
Minimum of three riders, maximum of eight.

\subsubsection{Large Group}
Minimum of nine riders, no maximum number of riders.

\section{Categories}

Artistic Freestyle competitions are organized either by age group (Age Group, Junior Expert, and Expert) or by category (Novice, Intermediate, and Expert) as determined by the host.

\subsection{Choosing Categories \label{subsec:freestyle-choosing_categories}}

\begin{tabular}{|l|c|}
\hline
\textbf{Category Name} & \textbf{Level} \\
\hline
Novice & 0-3 \\
\hline
Intermediate & 4-6 \\
\hline
Expert & 7-10 \\
\hline
\end{tabular}

Categories are determined by skill level.
The IUF Skill Levels are used as a guide to determine level of skill.
Skill level testing is not required; these numbers are just used as a point of reference.

For Pairs Freestyle the skill levels of the two riders should be averaged to determine category placement.

Riders may enter the competition category they wish according to the approximate skill
level of the skills planned for the routine.
Riders who wish to enter a category that falls outside the guidelines must communicate their choice and reasons to the Chief Judge before the competition.
The Chief Judge will review the choices to assure that riders enter categories that match their skills.

\subsection{Promoting Rider(s) to a Higher Category}
Because these categories are determined based on skill level and not age, it can be difficult to determine the correct category for any given routine.
Therefore, there may be a need to promote routines to a higher category after they have been evaluated.

A routine is allowed to have a maximum of three successfully performed skills that are deemed to be higher than the allowed level for the category.
Skills successfully performed is defined as performing the skill for a reasonable distance without falling, given the choreography of the routine.
When this limit of three is exceeded, the routine is to be promoted to the next most difficult level.
Clearly the skill levels are not an inclusive list of all the skills that may be performed in any given routine.
Therefore, the approximate difficulty level of each skill performed in any routine must be evaluated to determine whether or not the skill is too difficult for the given category.

It is up to the discretion of the Chief Judge as to whether or not a routine is promoted to a higher category.
The Chief Judge should take into account the opinions of the other judges when making this decision.

\section{Event Flow}

\subsection{Time Limits}

Artistic Freestyle competitions use the time limits outlined below.

\subsubsection{Time Limits For Age Groups}

\begin{tabular}{|l|c|c|}
\hline
\textbf{Competition} & \textbf{Age} & \textbf{Time Limit} \\
\hline
\multirow{4}{*}{Individual} & 0-14 & 2 minutes \\
\cline{2-3}
& 15+ & 3 minutes \\
\cline{2-3}
& Jr.~Expert & 3 minutes \\
\cline{2-3}
& Expert & 4 minutes \\
\hline
\multirow{4}{*}{Pair} & 0-14 & 2 minutes \\
\cline{2-3}
& 15+ & 3 minutes \\
\cline{2-3}
& Jr.~Expert & 3 minutes \\
\cline{2-3}
& Expert & 4 minutes \\
\hline
Small Group & All & 5 minutes \\
\hline
Large Group & All & 5 minutes \\
\hline
\end{tabular}

\subsubsection{Time Limits for Categories}
\begin{tabular}{|l|c|}
\hline
\textbf{Category Name} & \textbf{Time Limit} \\
\hline
Novice & 2 minutes \\
\hline
Intermediate & 3 minutes \\
\hline
Expert & 4 minutes \\
\hline
\end{tabular}


\subsection{Judging Method}

\subsubsection{Judging Method: Individual}
Riders' scores are divided into two parts called Technical and Performance, each receiving 50\% of the score.
Read the Artistic Freestyle Judging section to learn more.

\subsubsection{Judging Method: Pair}
Same as Individual Freestyle.
In Pairs, there is extra emphasis on teamwork; two person skills, etc.
(See Judging Criteria for a full description.)

\subsubsection{Judging Method: Group}
Same as Individual Freestyle, but with additional emphasis on teamwork and multiple person skills, such as formation riding.
Extra consideration will be given to account for widely different group sizes, relative skill levels, and relative ages of riders.
(See Judging Criteria for a full description.)

\subsubsection{Music Preparation}
\label{sec:freestyle_music}
In Artistic Freestyle events, music is included in the judging and competitors should use it.
It is recommended to have one or more backup copies of all music in case of loss or damage.
For recordable disks, competitors are also recommended to test their music on multiple players to make sure it will work at competition time.

Competitors must provide their music in a type that is supported, and has been announced by the Artistic Director.
All music must be clearly labeled with the competitor name(s), age group or category, event type (such as Pairs), and if needed, the track number.
Whenever possible, competition music should be the first track on the CD.
The DJ (music operator) is not responsible for any errors resulting from unsupported types or mislabeled tracks.

\subsection{Props and Decorations \label{subsec:freestyle_freestyle-rules_individual-freestyle-overview_props-and-decorations}}

\textbf{Props} are all items which are used by the rider in his/her performance and require a technical handling by the rider (for example typical objects like clubs, ribbons, hoop, etc.).
These items can be used to do a unicycling trick, like rope skipping with the unicycle.
However, they can also be employed to show a non-unicycling skill which supports and enhances the choreography, like the elaborate use of a hat.
Props have to be presented by the rider.
It is not mandatory to include props in the performance.
If none are used, the score will not be lower.

\textbf{Decoration:} In contrast to props, decoration is used to present the rider or clarify the theme of the performance.
Decoration does not require a technical handling by the rider.
For example other persons in costumes and background images.
Decoration is no personal contribution of the rider and therefore effects of the Decorations should not be judged.
On the contrary, Decorations can also be judged negatively if it distracts from the rider's performance.
For Junior Expert and Expert categories at Unicon, it is forbidden to use decorations (including people) that are too large, which the competitor cannot carry and/or put on by oneself.

For Props and Decoration neither fire nor sharp objects (such as juggling knives) are allowed.

\subsection{Riders Must Be Ready}
Riders who are not ready at their scheduled performance time may or may not be allowed to perform after the last competitor in their age group or category.
The Chief Judge will remember to consider language barriers, and that riders may be engaged in convention work to slow them down.
A rider may not perform before a different set of judges than those that judged the rest of their age group or category.

\section{Performance Set-Up}
Competitors are allowed a maximum of two minutes to set up their unicycles and props in the performing area.
Competitors who take too long risk being disqualified.
An extension of the set-up time can be given only by the Chief Judge and must be requested in advance.
Competitors must show a legitimate need when requesting more time, such as numerous props or complicated special effects.

\subsection{Rider's No-Signal Option}

A rider may have a well-planned routine to music that he or she knows is under the time limit, and does not wish for the acoustic signals to detract from his or her performance.
When riders sign up with the Rider Liaison they can request ``No acoustic signals.'' This will eliminate the `Start' signal, and the 30-second warning.%comment2016 "Rider can request this"
The Timer will still keep the time, and if the rider exceeds the time limit, the Timer will make the `double acoustic signal' to indicate the rider has run overtime.

\subsection{Start Of Performance}
The judging, the stopwatch, and the `performance' all start at the same time.
The Timer starts the watch at the beginning of the music, or at a signal from competitors, whichever comes first.
The signal can be a nod, wave, bow, verbal cue (``Start!'') or any clearly understandable means.
An acoustic signal (such as a whistle) will indicate that the timing and judging have started.
Any non-unicycling activities such as dancing, posing, acrobatics, etc., must be included within the time limit of the routine to be judged.
In all Freestyle routines, an acoustic signal will indicate when there are 30 seconds left.%comment2016 remove whistle
In all artistic events, two acoustic signals or a different signal will indicate the end of the riding time and end of the judging.

\subsection{End Of Performance}
The performance ends at a signal from the rider, such as a bow or ``Thank you,'' an obvious endpoint, or at the end of the time limit.
Nothing that occurs after the time limit may affect judging scores.

An acoustic signal will indicate the end of the time limit.
Any figures or performing that are done after the end of the time limit will not be judged Performing past the time limit will reduce the rider's score.
All time limits are maximums.
Riders need not fill the entire time, but a routine that is very short may suffer in points over a routine with more content.
However, a routine that is boring, repetitive or `padded' may lose points for being too long.
The rider must decide what makes the best performance.

\subsection{Clean-Up}
In unicycling, a clean, dry riding surface is essential.
After a performance, the riding area must be left the way it was before the performance.
Riders and their helpers must clear all props, unicycles, and debris from the performing area within two minutes.
The next rider may also be setting up during this time.

\subsection{Messy Performing Area}
Riders who are thinking of using messy props in their performances must carefully consider the above rule.
Popping balloons, dirt or powder, confetti, water, pies, etc.
may take longer than two minutes to remove.
Special permission must be received from the Chief Judge or Artistic Director before any such props are used.
Competitors who make messes they are unable to remove may be disqualified from the event.
