\chapter{Judges and Officials Rules}

\section{Artistic Officials}
\textbf{Artistic Director:} The head organizer and administrator of artistic events.
The Artistic Director's job starts well before the convention, arranging equipment for the gyms (or performing areas) and recruiting the other artistic officials.
With the Convention Host, the Artistic Director determines the operating systems, paperwork and methods to be used to run the events.
With the Chief Judge, the Artistic Director is in charge of keeping events running on schedule, and answers all questions not pertaining to rules and judging.
The Artistic Director is the highest authority on everything to do with the artistic events, except for decisions on rules and results.

\textbf{Chief Judge:} Like the Referee, the Chief Judge should be a thoroughly experienced person who must above all be objective and favor neither local, nor outside riders.
The Chief Judge must be thoroughly familiar with all of the artistic officials' jobs and all aspects of artistic rules.
The Chief Judge oversees everything, deals with protests, and answers all rules and judging questions.
The Chief Judge is responsible for seeing that all artistic officials are trained and ready, and that the artistic riding areas are correctly measured and marked on the floor.
The Chief Judge is also responsible for the accuracy of all judging point tabulations and calculations.

\textbf{Timer:} Keeps the time for all performances, and makes acoustic signals at key points in performances.

\textbf{Judge:} Rates the performances.
The various artistic categories require different judging qualities, and may use different judges.
All judges must be completely impartial, and must understand the rules and judging criteria.

\textbf{Announcer:} Operates sound system with DJ, and announces all upcoming riders and results of competitions.
May also provide color commentary between performances.

\textbf{DJ:} Operates sound system, plays all performance music, and keeps track of riders' music media and any special instructions.

\section{Training Officials}
Competitions cannot be started until all key artistic officials have been trained and understand their tasks.
For Artistic events, the Chief Judge is in charge.
For certain artistic events, a minimum level of judging experience is required.
See section \ref{sec:freestyle_judging-panel}.

\section{Interruption Of Judging}
An interruption of judging can result from material damage, injury or sudden illness of a competitor, or interference with a competitor by a person or object.
If this happens, the Chief Judge determines the amount of time left and whether any damage may be the fault of the competitor.
Re-admittance into competition must happen within the regulatory competition time.
If a routine is continued and the competitor was not at fault for the interruption, all devaluations coming forth from the interruption will be withdrawn.

\section{Freestyle Judging Panel \label{sec:freestyle_judging-panel}}
There are five (or more) judges each of Technical and Performance for Age Group, Jr.~Expert and Expert competitions (including Group).
All judges must attend a workshop provided as part of the convention schedule before the start of the Artistic Freestyle competitions.
Exceptions to workshop attendance are granted by the Chief Judge if judging rules have not changed since the previous judging experience and the judge has high Accuracy Scores.%comment2016 we don't have accuracy scores
Unless otherwise noted, judges at a Unicon must either speak English or have translation assistance for the specified language while judging.
Judges at other unicycle conventions should speak the dominant language of that convention or have translation assistance.

Judges' names must be provided to the Chief Judge (via email, FAX, or postal mail) by at least one month prior to the start of the unicycle convention and include the number of Freestyle conventions where they have been a competitor, judge, or simply in the audience.%comment2016 this doesn't even always happens
See section \ref{subsec:freestyle_judging-panel_nominating-freestyle-judges} for description of which teams/countries are required to provide judges.
Judges must be at least 14 years of age at the start of the event.
Judges are recommended to be a current Freestyle competitor, a former Freestyle competitor, an active coach of Freestyle routines, a proven judge at prior competitions, or an avid spectator who has observed many Freestyle routines.

\subsection{Selecting Judges \label{subsec:freestyle_judging-panel_selecting-judges}}
A person should not judge an event if he or she is:
\begin{itemize}
\item A parent, child or sibling of a rider competing in the event.
\item An individual or team coach, manager, trainer, colleague who is member of the same club specified in the registration form, colleague's family etc.
of a rider competing in the event.
\item A sponsor, part of a sponsoring organization or connected to an organization sponsoring any riders in the group to be judged.
\item A family member of another judge on the judging panel.
\end{itemize}
If the judging pool is too limited by the above criteria, restrictions can be eliminated starting from the bottom of the list and working upward as necessary only until enough judges are available.

\subsection{Assignment Of Age Group and Category Judges}
Judges will be chosen from the list of judges as provided in section \ref{subsec:freestyle_judging-panel_nominating-freestyle-judges}.
Judges who are competing in the event just before or just after the current category are strongly suggested to be eliminated from the list.
Judges will also be eliminated from the list for the current category as described in section \ref{subsec:freestyle_judging-panel_rating-judge-performance}.
The final selection of judges will be chosen based on their accuracy scores from the remaining list.%comment2016 we don't have accuracy scores

\subsection{Assignment Of Expert (And Junior Expert) Judges \label{subsec:freestyle_judging-panel_assignment-of-expert-judges}} %comment2016 redo whole section
Assignments for Expert and Jr.~Expert judges will be made by the Chief Judge using the most qualified of all judges available.
Qualifications are determined in the following order of importance:
\begin{itemize}
\item Highest judging accuracy scores obtained while judging age group (age groups judges must have a minimum of five entrants) or other Jr.~Expert and Expert events.
\item Greatest amount of Jr.~Expert and Expert judging experience.
\item Greatest amount of international judging experience.
\item Greatest number of Freestyle competition experienced (viewed, judged, or as a competitor).
\end{itemize}
Judges who are competing in the event just before or just after the current category are eliminated from the list.
Judges will also be eliminated from the list for the current category as described in section \ref{subsec:freestyle_judging-panel_selecting-judges}.
Judges will also be eliminated from the list if they exhibit Judging weaknesses during their Age Group judging as described in Section \ref{subsec:freestyle_judging-panel_rating-judge-performance}.
At Unicons, if more than five judges each of Technical and Performance remain, judges who have not judged at a previous Unicon will be removed from the list.
If there are still more than five each then the final list of judges for the category will be chosen by accuracy scores as defined in section \ref{subsec:freestyle_judging-panel_calculating-accuracy-scores}.

\subsection{Judging Panel May Not Change}
The individual members of the judging panel must remain the same for the entirety of an age groups or category; for example one judge may not be replaced by another except between age groups or categories.
In the event of a medical or other emergency, this rule can be waived by the Chief Judge.

\subsection{Rating Judge Performance \label{subsec:freestyle_judging-panel_rating-judge-performance}} %comment2016 we don't do this, This section should be reworked
Judges are rated by comparing their scores to those of other judges at previous competitions.
Characteristics of Judging Weaknesses
\begin{itemize}
\item \textbf{Excessive Ties:} A judge should be able to differentiate between competitors.
Though tying is most definitely acceptable, excessive use of tying defeats the purpose of judging.
\item \textbf{Group Bias:} If a judge places members of a certain group or nation significantly different from the other judges.
This includes a judge placing members significantly higher or significantly lower (a judge may be harsher on his or her own group members) than the other judges.
\item \textbf{Inconsistent Placing:} If a judge places a large number of riders significantly different from the average of the other judges.
\end{itemize}

\subsection{Re-Instating Judges} %comment2016 we don't do this, This section should be reworked
If a judge has been labeled as having a Judging Weakness, they may have a chance to be re-instated on the list by:
\begin{itemize}
\item Discuss with the Chief Judge the scores that were Tied, Biased, or Inconsistent.
\item Practice judge on at least two categories with at least 4 competitors.
\end{itemize}
If the practice judging shows no further examples of Judging Weakness, they may be reinstated on approval by the Chief Judge and Artistic Director.
If the Chief Judge and Artistic Director are the same person, then the next highest-ranking official must agree to the reinstatement.

\subsection{Calculating Accuracy Scores \label{subsec:freestyle_judging-panel_calculating-accuracy-scores}} %comment2016 we don't do this, This section should be reworked
The Chief judge should decide to replace a judge if he/she shows signs of weakness.
To find the right decision, the chief judge may use heuristics, statistical analytics, etc.\ as indicators.

\subsection{Nominating Freestyle Judges \label{subsec:freestyle_judging-panel_nominating-freestyle-judges}} %comment2016 this should really be clarified
Parties (Countries/Clubs) that participate at competitions must nominate judges in relation to the number of Artistic Freestyle participants they send (see table below).
After registration finishes, the chief judge will send a request to all parties.
The request contains the compiled number of minimum judges per party and a question to nominate the candidates.
Judge nominations include the experience of the judges (such as previous competitions, how long he/she has been judging, age group/expert judging or other relevant qualifications such as educations).

\begin{tabular}{|l|l|}
\hline
\textbf{Number of Participants per Party} & \textbf{Minimum Number of Judges per Party} \\
\hline
$<$5 & 0 \\
\hline
5-10 & 2 \\
\hline
11-20 & 3 \\
\hline
21-30 & 4 \\
\hline
$>$30 & 5 \\
\hline
\end{tabular}

\subsection{Not Providing Judges} %comment2016 see previous
At Competitions, parties that are unable to provide their required number of judges (either Group or Individual/Pairs) may have their competitors removed from that competition.
Exceptions will be granted on a special basis with a letter to the Chief Judge, Artistic Director, and Convention Director.
\textbf{Note:} A party that isn't able to nominate their minimum judges can ask a party that has more than the required number of minimum judges to nominate their additional judges as their own.

\subsection{Judges Workshop}
Judges should have read the rules prior to the start of the workshop.
The workshop will include a practice judging session.
Each judge will be required to sign a statement indicating they have read the rules, attended the workshop, agree to follow the rules, and will accept being removed from the list of available judges if their judging accuracy scores show Judging Weaknesses.%comment2016 accuracy scores

\section{Scoring}
To tabulate the scores in Artistic Freestyle events, each judge's scores in a category or age group are totaled and restated as a percentage of that judge's total points for that category or age group, i.e.\ the judge's relative preferences for each performer.
Then the percentages from all of the judges are totaled together.
This is done first in both Performance and Technical.
Once a percentage total for each competitor has been calculated for both Performance and Technical, these percentages are combined together to see the final results.

\subsection{Removing Scores}
All of the judges scores must be kept.
The Chief Judge has the power to remove scores only if they are deemed to be biased, inaccurate, or another extreme case.
%comment2016 do we want more text here?

\subsection{Ties}
In the case of a tie where more than one competitor has the same placing score after the above process, those riders will be ranked based on their placing scores for Technical.
The scoring process must be repeated using only the Technical scores for the tied riders to determine this rank.
If competitors' Technical ranking comes out equal, all competitors with the same score are awarded the same place.

\section{Artistic Freestyle Judging}
Judging for Individual, Pairs, and Group Freestyle is divided into two components, Technical and Performance.
Qualified judges may judge only Technical, only Performance, or both.
For each component, judges give three or four scores from 0 to 10, or 0 to 15, high scores being better.
Scores such as 2.0, 2.2, or even 2.25 are encouraged to help differentiate between riders of similar ability.

\subsection{Individual Freestyle -- Technical Score \label{sec:freestyle_individual-technical-score}}
The Technical part of the judging is broken into three parts.
Three scores will be given by each judge, values ranging from 0 to 10, or from 0 to 15 as follows:
\begin{itemize}
\item Quantity of Unicycling Skills And Transitions (0-10 points)
\item Mastery And Quality of Execution (0-15 points)
\item Difficulty And Duration (0-15 points)
\end{itemize}
\textbf{Technical Total:} 40 points

\subsubsection{Quantity of Unicycling Skills and Transitions}
\textbf{Quantity} is the number of unicycling skills and transitions successfully executed.
Transitions, before and after the skill, should also be counted.
If a dismount happens during transition but after a skill was successfully executed, only the completed skill is counted and the failed transition should not be counted.
For example, if a dismount happens during stand up gliding, only the transition from riding to stand up is counted.
If a dismount happens after stand up gliding and during the transition from the stand up gliding to riding, the previous transition into stand up and the stand up gliding are counted.

Only `unicycling skills' will be counted (see definition in chapter \ref{chap:general_definitions}).
For example, if a rider is juggling while idling, idling is counted as a unicycling skill and juggling will affect the Interpretation: Props and other Performance scores.
Performing many short skills with quick transitions can increase this score, but will decrease the score as related to the Duration score.

\textbf{Variety:} Different variations of the same type of skill are counted separately.
Skills should be chosen to work with the style of the performance, but performing exactly the same skill multiple times will decrease this score.

Examples:
\begin{itemize}
\item `Drag seat in front' and `drag seat in back' are counted independently.
\item The following variations of `stand up gliding one foot' will be counted differently;
	\begin{itemize}
 	\item Arabesque (The free leg is extended behind the body above hip height - at least a 90 degree angle)
	\item Knee hold (one hand supporting the knee of the free leg)
	\item Y-character balance (holding a straightened leg up with one hand and using other hand to form a Y shape)
	\item Catch-foot (the free leg being held in one or both hands)
	\item Biellmann (the free leg grasped from behind and pulled overhead in the Biellmann position)
	\end{itemize}
\item Face up spins are different from normal upright spins
\item Combinations of one-rotation spins/turns are different from continuous spins
\end{itemize}

\textbf{Originality:} In Artistic Freestyle, new skills are less important than in Flatland.%comment2016 why is flatland mentioned here?
However, skills with unique variations that are completely new or with new approaches will get more points.
Originality is mainly judged in Performance (section \ref{sec:freestyle_individual-performance-score}).

\subsubsection{Mastery And Quality of Execution}
\textbf{Mastery} is the amount of control shown by the rider(s) during their execution of the skills and transitions.
The body form should demonstrate good control and Mastery of the unicycle.
If a rider is showing good style (section \ref{subsec:freestyle_individual-performance-score_presence-execution}) during difficult skills, the Mastery score should be high.
Mastery of the unicycling skills is also required to perform the ``additional non-unicycling skills'', such as juggling, dancing, and acrobatics.

There are several viewpoints to check the Quality of Execution, such as Stability, Duration, Speed, Synchronization, and Fluidity of Transition.
These viewpoints don't have to be evenly weighted, but required to check.

\textbf{Duration:} Holding a skill for a longer amount of time and distance also indicates a higher level of mastery and difficulty for that skill.

\textbf{Stability:} High scores should not be given if unintentional jerky body movement, or a wandering spin or pirouette is shown occasionally.

\textbf{Speed:} High score is given when the rider controls the speed (faster or slower) of turns, spins, and transitions excellently.

\textbf{Synchronization:} Being synchronized with the rhythm of the music and timing accuracy should be judged.
High scores are awarded for a routine if timing of the skills is well planned and accurate.

\textbf{Fluidity of Transition:} High scores are given for transitions when the rider performs a skill straight into another skill quickly.
Low scores are given for transitions if several revolutions, idles, hops (or other setup-type skill) need to be performed before performing the more difficult skill - unless it is obvious that these are used to increase the overall choreography and timing of the routine.

\subsubsection{Difficulty And Duration}
The level of Difficulty is taken into account for successfully executed skills including transitions.
High scores are awarded for a routine packed with a number of skills all with high difficulty.
High scores should not be given if only one or two of the skills are of a high level.
Generally:
\begin{itemize}
\item Backward skills are more difficult than the same type of Forward skills.
\item `Seat against body' is easier than `Seat not touching body'.
\item Faster spins/turns with smaller diameter are more difficult than slower spins/turns with larger diameter.
\item `Stand up with a hand touching the seat' is easier than `stand up with neither hand touching the seat'.
\item `Jump up from the pedals to the frame removing both feet simultaneously' is more difficult than `Standup with one or both feet on the frame'.
\end{itemize}
If a rider is juggling while idling, for example, the difficulty of idling does not carry the same difficulty as idling without juggling.
The same applies for dancing, and acrobatics.

\textbf{Duration:} Holding a skill for a longer amount of time and distance also indicates a higher level of mastery and difficulty for that skill.

\subsection{Individual Freestyle -- Performance Score \label{sec:freestyle_individual-performance-score}}
The Performance half of the judging has been broken into four parts.
Four scores are to be given by each judge with values ranging from 0 to 10 as follows:
\begin{itemize}
\item Mistakes: Dismounts (0-10 points)
\item Presence/Execution (0-10 points)
\item Composition/Choreography (0-10 points)
\item Interpretation of the Music/Timing (0-10 points)
\end{itemize}
\textbf{Performance Total: 40 points}
Each part includes a definition of the terms as well criteria to be considered by the judges.

\subsubsection{Mistakes: Dismounts}
Low scores are given for routines with more than 8 major dismounts, therefore interrupting the flow of the routine.
Medium scores are given for a routine that has approximately 3 major dismounts and a few minor dismounts.
High scores are given for a routine with no major dismounts, and few or no minor dismounts.
Judges need to be able to differentiate between a planned dismount and an unplanned dismount.

\textbf{Major} dismounts are when the unicycle falls and/or a hand or any body part other than the rider's foot or feet touch the floor.
Major dismounts are also when the choreography of a rider's routine is clearly affected.

\textbf{Minor} dismounts are when the unicycle does not fall, only the rider's foot or feet touch down and the choreography of a rider's routine is not affected.
A minor dismount may also be counted when Judges cannot differentiate between a planned dismount and an unplanned dismount.

\textbf{Exception:}
Dismounts that occur while the rider is performing a seat drag skill have to be evaluated somewhat differently since the unicycle is already on the ground.
For these dismounts, the judges should use the current above language regarding minor and major dismounts but disregard the parts talking about the unicycle.
For example, if a rider is performing seat drag in back and steps off the unicycle with only their feet touching the ground, it would be considered a minor dismount unless the choreography of the routine is plainly affected.

\textbf{Scores are generated using the following calculations:}

\begin{tabular}{r l}
Score = 10 & $- 1.0\ \cdot$ (number of major dismount(s)) \\
 & $- 0.5\ \cdot$ (number of minor dismount(s)) \\
\end{tabular}

\subsection{Presence/Execution \label{subsec:freestyle_individual-performance-score_presence-execution}}
There are two parts to this section.
Each part does not need to be evenly weighted, but judges are required to consider both parts.

\textbf{Presence:} involvement of the rider physically, emotionally and intellectually as they translate the intent of the music and choreography.

\textbf{Execution:} quality of movement and precision in delivery.
This includes harmony of movement.

\textbf{Criteria:}
\begin{itemize}
\item Physical, emotional and intellectual presence
\item Carriage
\item Authenticity (individuality/personality)
\item Clarity of movement
\item Variety and contrast
\item Projection
\end{itemize}

\subsubsection{Composition/Choreography}
An intentional, developed and/or original arrangement of all types of movements according to the principles of proportion, unity, space, pattern, structure and phrasing.

\textbf{Criteria:}
\begin{itemize}
\item Purpose (Movements, tricks, costume and music to match for a unified experienced)
\item Harmony (Interaction between tricks and body movements)
\item Utilization (Utilization of space and pattern usage for proper floor coverage)
\item Dynamics
\item Imaginativeness (Imaginative, originality and inventiveness of purpose, movement and design)
\end{itemize}

\subsubsection{Interpretation of the Music/Timing}
The personal and creative \emph{musical realization} of the rhythm, character and content of music to movement in the performance area.

\textbf{Criteria:}
\begin{itemize}
\item Continuity and musical representation (Effortless proper musical realization to artfully characterize the routine)
\item Expression of the music's style, character and rhythm
\item Use of \emph{finesse} to reflect the nuances of the music
\item Timing
\end{itemize}

\textbf{Musical realization} can occur in four different ways:
\begin{itemize}
\item Analog: Representing the highlights/cues in the music through movements
\item Congruent: Representing every beat/tone/note in the music through movements
\item Contrastive: Movement that is contrary to the music (fast movements to slow music or vice versa) or putting moves on the off-beat
\item Autonomy: Movement and music are independent.
Music and movements don't need to match, yet can.
This offers the artist the most freedom in his creative process
\end{itemize}

\textbf{Finesse} is the refined, artful manipulation of nuances by the rider(s).
Nuances are the personal artistic ways of bringing variations to the intensity, tempo, and dynamics of the music made by the composer and/or musicians.

\subsection{Pairs Freestyle -- Additional Judging Criteria}
Pairs judges must consider the performance of two unicyclists together.
All judging criteria and the scoring from Individual Freestyle are used, but the additional factors below must also be considered.
The only exceptions are the scoring guidelines for Pairs Freestyle Difficulty and Duration as mentioned below in section \ref{subsec:freestyle_pairs-additional-judging-criteria_difficulty-duration}.

\subsubsection{Pairs Freestyle: Quantity of Unicycling Skills and Transitions \label{subsec:freestyle_pairs-additional-judging-criteria_quantity}}
Number of skills should be counted for each rider separately.
If a rider is not riding unicycle and performing non-unicycling skills while the other rider doing unicycling skills, only one skill for a rider is counted.

\textbf{Pairs Vs.\ Doubles:} `Doubles' refers to two riders on one unicycle.
In case of Doubles, the Quantity is counted as same as the skill by a single rider.

\subsubsection{Pairs Freestyle: Mastery and Quality of Execution}
\textbf{Synchronization:} Timing-synchronization with each other should be judged in Pairs routines.
High scores awarded for a routine if timing of every skills are well planed and accurate.
Even though riders do not do the same skill/movement at the same timing intentionally in pairs, timing accuracy of each movement can be measured as synchronization with rhythm of the music, in a manner similar to individual routines.
The same rules and chart from Individual Freestyle is to be used for Pairs Freestyle.

\subsubsection{Pairs Freestyle: Difficulty and Duration \label{subsec:freestyle_pairs-additional-judging-criteria_difficulty-duration}}
The Difficulty level of a multiple person act is determined by the overall level of difficulty displayed by the pair, not by the difficulty of feats presented by a single rider.
If one rider's skill level is a great deal higher than the other, judges must keep the Difficulty score somewhere between the levels of the two riders.
Number of skills should be counted for each rider separately.
If a rider is not riding unicycle and performing non-unicycling skills while the other rider doing unicycling skills, only one skill for a rider is counted.
A skill in which the two riders obviously support each other will score lower than the same skill performed separately.
Judges must be able to distinguish between `support' and `artistic contact.' Riders who are merely holding hands may not be supporting each other, but if their arms are locked, they probably are.

\textbf{Note:} Some skills are more difficult with riders holding hands, such as one foot riding, side-by-side.

\textbf{Pairs Vs.\ Doubles:} `Doubles' refers to two riders on one unicycle.
In case of Doubles, the Quantity is counted as same as the skill by a single rider.
Some Pairs performers use lots of doubles moves, with lifting, strength, and the associated difficulty.
Other Pairs acts use no doubles moves at all.
How to compare them?
Remember that the skill level of both riders is being judged.
If the `top' rider does not display much unicycling skill when he or she rides, judges must keep that in mind, and rate their average difficulty accordingly.
If the top rider never rides, one can argue that this is not a Pairs act, and give a major points reduction.
Doubles moves are difficult for both persons, but must be weighed carefully against non-doubles performances.

Duration can be increased if a rider pulls or pushes another rider with holding hands, but will decrease the score as related to Quantity (section \ref{subsec:freestyle_pairs-additional-judging-criteria_quantity}).

\subsubsection{Pairs Freestyle: Presence/Execution}
\textbf{Additional Criteria:}
\begin{itemize}
\item Unison and ``oneness''
\item Balance in performance between partners
\item Spatial awareness between partners
\end{itemize}

\subsubsection{Pairs Freestyle: Composition/Choreography}
\textbf{Additional Criteria:}
\begin{itemize}
\item Unity
\item Shared responsibility in achieving purpose by both
\end{itemize}

\subsubsection{Pairs Freestyle: Interpretation of the Music/Timing}
\textbf{Additional Criteria:}
\begin{itemize}
\item Relationship between the partners reflecting the character of the music
\end{itemize}

\subsection{Group Freestyle -- Additional Judging Criteria}
Everything for Individual and Pairs applies, including the scoring, plus these additional points.
Exceptions are in the scoring detailed in sections \ref{subsec:freestyle_group-additional-judging-criteria_difficulty-duration} and \ref{subsec:freestyle_group-additional-judging-criteria_dismounts}.
A group of several riders has many more options of what to do and how it can be presented.
Riders may all be of similar skill levels, or of widely different levels.
Some groups will be much larger than others.
These things all need to be considered when judging groups.

\subsubsection{Quantity of Unicycling Skills and Transitions}
Approximate number of skills may be counted for all members in total.
The number of skills should be weighted by the number of unicycling riders in the group.
If some riders are not on unicycles or are performing non-unicycling skills while the other riders doing unicycling skills, the count is reduced accordingly.

\subsubsection{Group Freestyle: Mastery and Quality of Execution}
\textbf{Synchronization:} Timing-synchronization with all members should be judged in Group routines.
High scores awarded for a routine if timing of every skills are well planed and accurate.
Even though sub-groups do not do the same skill/movement with other sub-groups at the same timing intentionally, timing accuracy of each movement can be measured as synchronization with rhythm of the music, in a manner similar to individual routines.

\textbf{Mastery} is the amount of control shown by the riders during their execution of the skills.
The body form should demonstrate good control and `mastery' of the unicycle.
Holding a skill for a longer amount of time also indicates a higher level of mastery for that skill.
Performing a skill multiple times can increase the Mastery portion of the score, but will decrease the score as related to Variety and Level of Difficulty.
If the group shows good style (section \ref{subsec:freestyle_individual-performance-score_presence-execution}) during difficult skills, the Mastery score should be high.

\subsubsection{Group Freestyle: Difficulty and Duration \label{subsec:freestyle_group-additional-judging-criteria_difficulty-duration}}
As in Pairs, judges must seek to find the average Level of Difficulty of what may be a widely varied group of riders.
Top level skills done by only one rider cannot bring the Difficulty score up to top level.
High scores should not be given if only one or two of the skills are of a high level even if done by all riders or with skills that are the same type but with minor variations.
All riders in the routine must be used effectively.
This means that if one or more riders are at a beginner level, they can still ride around in circles, carry banners, be carried by other riders, etc.
Riders should not be left standing on the side.

\textbf{Small Group Vs.\ Large Group:} Some groups will be much smaller or larger than others, and judges must include this information in their decisions.
Large groups may have a tendency toward formation riding and patterns, while smaller groups may focus more on difficult skills.
With so many possibilities, judges must compare many different factors to get an adequate judgment.
Large numbers alone should not earn a high difficulty score, and neither should a few difficult skills performed by a small number.
The judges must consider the group's size as a part of the overall performance, including the advantages or limitations that size has on the types of skills being performed.

\textbf{Level of difficulty} is for successfully executed skills.
High scores awarded for a routine packed with a number of skills that have a high variety.
Only `unicycling skills' will be judged; non-unicycling skills only affect Performance scores.
Dancing, juggling, and other non-unicycling skills can increase only the Presentation score, and have no influence on this score.

\subsubsection{Group Freestyle: Dismounts \label{subsec:freestyle_group-additional-judging-criteria_dismounts}}
At Unicon a minimum of four dismount judges must be appointed by the Chief Judge to count falls for group routines.
At smaller competitions a minimum of two dismount judges should be appointed by the Chief Judge to count falls for group routines.
These judges should be chosen to fairly represent the different groups present (e.g.\ four judges from four different countries).
Performance judges should not count falls.
Each dismount judge should count all of the major and minor dismounts of a group, not only dismounts in one part of the floor.
The counts from each judge should be averaged to create the dismount score for a group.

\textbf{Scores are generated using the following calculations:}
The number of dismounts should be weighted by the number of riders in the group.
The following formula is used:

\makebox{Mistake Score = \ $1.0\ \cdot$ (number of major dismount(s))} \\
\makebox[\textwidth][r]{\hfill$+ 0.5\ \cdot$ (number of minor dismount(s))}

Final Dismount Score $= 10 \, -$ mistake score/number of riders.
This score cannot be lower than 0.

\subsubsection{Group Freestyle: Presence/Execution}
In addition to the description for Individual Freestyle (section \ref{subsec:freestyle_individual-performance-score_presence-execution}), judges are looking for teamwork and cooperation.
Additionally, movements should cover the performing area uniformly and use all riders effectively.
