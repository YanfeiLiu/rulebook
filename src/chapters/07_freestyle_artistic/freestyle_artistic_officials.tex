\chapter{Judges and Officials Rules}

\section{Artistic Officials}
\textbf{Artistic Director:} The head organizer and administrator of artistic events.
The Artistic Director's job starts well before the convention, arranging equipment for the gyms (or performing areas) and recruiting the other artistic officials.
With the Convention Host, the Artistic Director determines the operating systems, paperwork and methods to be used to run the events.
With the Chief Judge, the Artistic Director is in charge of keeping events running on schedule, and answers all questions not pertaining to rules and judging.
The Artistic Director is the highest authority on everything to do with the artistic events, except for decisions on rules and results.

\textbf{Chief Judge:} Like the Referee, the Chief Judge should be a thoroughly experienced person who must above all be objective and favor neither local, nor outside riders.
The Chief Judge must be thoroughly familiar with all of the artistic officials' jobs and all aspects of artistic rules.
The Chief Judge oversees everything, deals with protests, and answers all rules and judging questions.
The Chief Judge is responsible for seeing that all artistic officials are trained and ready, and that the artistic riding areas are correctly measured and marked on the floor.
The Chief Judge is also responsible for the accuracy of all judging point tabulations and calculations.

\textbf{Timer:} Keeps the time for all performances, and makes acoustic signals at key points in performances.

\textbf{Judge:} Rates the performances.
The various artistic categories require different judging qualities, and may use different judges.
All judges must be completely impartial, and must understand the rules and judging criteria.

\textbf{Announcer:} Operates sound system with DJ, and announces all upcoming riders and results of competitions.
May also provide color commentary between performances.

\textbf{DJ:} Operates sound system, plays all performance music, and keeps track of riders' music media and any special instructions.

\section{Training Officials}
Competitions cannot be started until all key artistic officials have been trained and understand their tasks.
For Artistic events, the Chief Judge is in charge.
For certain artistic events, a minimum level of judging experience is required.
See section \ref{sec:freestyle_judging-panel}.

\section{Interruption Of Judging}
An interruption of judging can result from material damage, injury or sudden illness of a competitor, or interference with a competitor by a person or object.
If this happens, the Chief Judge determines the amount of time left and whether any damage may be the fault of the competitor.
Re-admittance into competition must happen within the regulatory competition time.
If a routine is continued and the competitor was not at fault for the interruption, all devaluations coming forth from the interruption will be withdrawn.

\section{Freestyle Judging Panel \label{sec:freestyle_judging-panel}}
At Unicon and continental championships there must be a minimum of five judges each of Technical and Performance for Age Group, Jr. Expert and Expert competitions (including Group). %comment2017 this should be "for all categories" edit me
At Unicon a minimum of four Dismount judges are required for group routines.
For smaller competitions there must be a minimum of three judges for Technical and Performance, and a minimum of two Dismount judges.
There must be an equal number of Technical and Performance judges.
All judges must attend a workshop provided as part of the convention schedule before the start of the Artistic Freestyle competitions.
Exceptions to workshop attendance are granted by the Chief Judge if judging rules have not changed since the previous judging experience and the judge has high Accuracy Scores.%comment2016 we don't have accuracy scores
Unless otherwise noted, judges at a Unicon must either speak English or have translation assistance for the specified language while judging.
Judges at other unicycle conventions should speak the dominant language of that convention or have translation assistance.

Judges' names must be provided to the Chief Judge (via email, FAX, or postal mail) by at least one month prior to the start of the unicycle convention and include the number of Freestyle conventions where they have been a competitor, judge, or simply in the audience.%comment2016 this doesn't even always happens
See section \ref{subsec:freestyle_judging-panel_nominating-freestyle-judges} for description of which teams/countries are required to provide judges.
Judges must be at least 14 years of age at the start of the event.
Judges are recommended to be a current Freestyle competitor, a former Freestyle competitor, an active coach of Freestyle routines, a proven judge at prior competitions, or an avid spectator who has observed many Freestyle routines.

Anything that happens in the course of the judging process (including, but not limited to, judge scores and discussions with the Chief Judge) is confidential and can not be discussed or shared.

\subsection{Selecting Judges \label{subsec:freestyle_judging-panel_selecting-judges}}
A person should not judge an event if he or she is:
\begin{itemize}
\item A parent, child or sibling of a rider competing in the event.
\item An individual or team coach, manager, trainer, colleague who is member of the same club specified in the registration form, colleague's family etc.
of a rider competing in the event.
\item A sponsor, part of a sponsoring organization or connected to an organization sponsoring any riders in the group to be judged.
\item A family member of another judge on the judging panel.
\end{itemize}
If the judging pool is too limited by the above criteria, restrictions can be eliminated starting from the bottom of the list and working upward as necessary only until enough judges are available.

\subsection{Assignment Of Age Group and Category Judges}
Judges will be chosen from the list of judges as provided in section \ref{subsec:freestyle_judging-panel_nominating-freestyle-judges}.
Judges who are competing in the event just before or just after the current category are strongly suggested to be eliminated from the list.
Judges will also be eliminated from the list for the current category as described in section \ref{subsec:freestyle_judging-panel_rating-judge-performance}.
The final selection of judges will be chosen based on their accuracy scores from the remaining list.%comment2016 we don't have accuracy scores

\subsection{Assignment Of Expert (And Junior Expert) Judges \label{subsec:freestyle_judging-panel_assignment-of-expert-judges}} %comment2016 redo whole section
Assignments for Expert and Jr.~Expert judges will be made by the Chief Judge using the most qualified of all judges available.
Qualifications are determined in the following order of importance:
\begin{itemize}
\item Highest judging accuracy scores obtained while judging age group (age groups judges must have a minimum of five entrants) or other Jr.~Expert and Expert events.
\item Greatest amount of Jr.~Expert and Expert judging experience.
\item Greatest amount of international judging experience.
\item Greatest number of Freestyle competition experienced (viewed, judged, or as a competitor).
\end{itemize}
Judges who are competing in the event just before or just after the current category are eliminated from the list.
Judges will also be eliminated from the list for the current category as described in section \ref{subsec:freestyle_judging-panel_selecting-judges}.
Judges will also be eliminated from the list if they exhibit Judging weaknesses during their Age Group judging as described in Section \ref{subsec:freestyle_judging-panel_rating-judge-performance}.
At Unicons, if more than five judges each of Technical and Performance remain, judges who have not judged at a previous Unicon will be removed from the list.
If there are still more than five each then the final list of judges for the category will be chosen by accuracy scores as defined in section \ref{subsec:freestyle_judging-panel_calculating-accuracy-scores}.

\subsection{Judging Panel May Not Change}
The individual members of the judging panel must remain the same for the entirety of an age groups or category; for example one judge may not be replaced by another except between age groups or categories.
In the event of a medical or other emergency, this rule can be waived by the Chief Judge.

\subsection{Rating Judge Performance \label{subsec:freestyle_judging-panel_rating-judge-performance}} %comment2016 we don't do this, This section should be reworked
Judges are rated by comparing their scores to those of other judges at previous competitions.
Characteristics of Judging Weaknesses
\begin{itemize}
\item \textbf{Excessive Ties:} A judge should be able to differentiate between competitors.
Though tying is most definitely acceptable, excessive use of tying defeats the purpose of judging.
\item \textbf{Group Bias:} If a judge places members of a certain group or nation significantly different from the other judges.
This includes a judge placing members significantly higher or significantly lower (a judge may be harsher on his or her own group members) than the other judges.
\item \textbf{Inconsistent Placing:} If a judge places a large number of riders significantly different from the average of the other judges.
\end{itemize}

\subsection{Reinstating Judges} %comment2016 we don't do this, This section should be reworked
If a judge has been labeled as having a Judging Weakness, they may have a chance to be reinstated on the list by:
\begin{itemize}
\item Discuss with the Chief Judge the scores that were Tied, Biased, or Inconsistent.
\item Practice judge on at least two categories with at least 4 competitors.
\end{itemize}
If the practice judging shows no further examples of Judging Weakness, they may be reinstated on approval by the Chief Judge and Artistic Director.
If the Chief Judge and Artistic Director are the same person, then the next highest-ranking official must agree to the reinstatement.

\subsection{Calculating Accuracy Scores \label{subsec:freestyle_judging-panel_calculating-accuracy-scores}} %comment2016 we don't do this, This section should be reworked
The Chief judge should decide to replace a judge if he/she shows signs of weakness.
To find the right decision, the chief judge may use heuristics, statistical analytics, etc.\ as indicators.

\subsection{Nominating Freestyle Judges \label{subsec:freestyle_judging-panel_nominating-freestyle-judges}}
Parties (Countries/Clubs) that participate at competitions must nominate judges in relation to the number of Artistic Freestyle participants they send (see table below).
For the purpose of counting judges, an ``Artistic Freestyle participant'' is defined as a routine.
(For example: If Country A has 3 Individual routines, 2 sets of Pairs routines, and 1 Small Group, then they have 6 participants and will need 2 judges.)
After registration finishes, the chief judge will send a request to all parties.
The request contains the compiled number of minimum judges per party and a question to nominate the candidates.
Judge nominations include the experience of the judges (such as previous competitions, how long he/she has been judging, age group/expert judging or other relevant qualifications such as educations).

\begin{tabular}{|l|l|}
\hline
\textbf{Number of Participants per Party} & \textbf{Minimum Number of Judges per Party} \\
\hline
$<$5 & 0 \\
\hline
5-10 & 2 \\
\hline
11-20 & 3 \\
\hline
21-30 & 4 \\
\hline
$>$30 & 5 \\
\hline
\end{tabular}

\subsection{Not Providing Judges} %comment2016 see previous
At Competitions, parties that are unable to provide their required number of judges (either Group or Individual/Pairs) may have their competitors removed from that competition.
Exceptions will be granted on a special basis with a letter to the Chief Judge, Artistic Director, and Convention Director.
\textbf{Note:} A party that isn't able to nominate their minimum judges can ask a party that has more than the required number of minimum judges to nominate their additional judges as their own.

\subsection{Judges Workshop}
Judges should have read the rules prior to the start of the workshop.
The workshop will include a practice judging session.
Each judge will be required to sign a statement indicating they have read the rules, attended the workshop, agree to follow the rules, and will accept being removed from the list of available judges if their judging accuracy scores show Judging Weaknesses.%comment2016 accuracy scores

\section{Scoring}
To tabulate the scores in Artistic Freestyle events, each judge's scores in a category or age group are totaled and restated as a percentage of that judge's total points for that category or age group, i.e.\ the judge's relative preferences for each performer.
Then the percentages from all of the judges are averaged per riders.
This is done first in Performance, Technical, and Dismount.
Once the percentage average value for each competitor has been calculated, these three averaged percentages are added together according to their weighting to see the final results.

\subsection{Removing Scores}
All of the judges scores must be kept.
The Chief Judge has the power to remove scores only if they are deemed to be biased, inaccurate, or another extreme case.
%comment2016 do we want more text here?

\subsection{Ties}
In the case of a tie where more than one competitor has the same placing score after the above process, those riders will be ranked based on their placing scores for Technical.
The scoring process must be repeated using only the Technical scores for the tied riders to determine this rank.
If competitors' Technical ranking comes out equal, all competitors with the same score are awarded the same place.
