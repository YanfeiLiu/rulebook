\chapter{Artistic Freestyle Judging}
\section{Introduction}
Judging for Individual, Pairs, and Group Freestyle is divided into three components, Technical, Performance, and Dismounts.
They are weighted as follows:

\begin{tabular}{l l l}
 & Technical & 45\% \\
 & Performance & 45\% \\
 & Dismounts & 10\% \\
\end{tabular}\\

Qualified judges may judge one or more components (Technical, Performance, Dismounts) per competition category, except in the case of group freestyle.
For each component, there are subcategories each of which \textbf{is always scored out of 10 points} (for the ease of judging).
High scores are better.
Scores such as 2.0, 2.2, or even 2.25 are encouraged to help differentiate between riders of similar ability.
The subcategories for each component may be weighted differently as described below.

\subsection{Pairs and Group Freestyle -- Additional Judging Criteria}
The judging criteria listed in the following sections applies to Individual, Pairs, and Group Freestyle. In many sections, there is additional judging criteria for Pairs and/or Group Freestyle that judges also need to consider.

\section{Technical Score \label{sec:freestyle_technical-score}}
The Technical part of the judging is broken into three subcategories.
These subcategories are weighted as follows:

\begin{tabular}{l l l}
 & Quantity of Unicycling Skills And Transitions & 33.33...\% \\
 & Mastery And Quality of Execution & 33.33...\% \\
 & Difficulty And Duration & 33.33...\% \\
\end{tabular}

\subsection{Quantity of Unicycling Skills and Transitions}
\textbf{Quantity} is the number of unicycling skills and transitions successfully executed.
Transitions, before and after the skill, should also be counted.
If a dismount happens during transition but after a skill was successfully executed, only the completed skill is counted and the failed transition should not be counted.
For example, if a dismount happens during stand up gliding, only the transition from riding to stand up is counted.
If a dismount happens after stand up gliding and during the transition from the stand up gliding to riding, the previous transition into stand up and the stand up gliding are counted.

Only `unicycling skills' will be counted (see definition in chapter \ref{chap:general_definitions}).
For example, if a rider is juggling while idling, idling is counted as a unicycling skill and juggling will affect the Interpretation: Props and other Performance scores.
Performing many short skills with quick transitions can increase this score, but will decrease the score as related to the Duration score.

\textbf{Variety:} Different variations of the same type of skill are counted separately.
Skills should be chosen to work with the style of the performance, but performing exactly the same skill multiple times will decrease this score.

Examples:
\begin{itemize}
\item `Drag seat in front' and `drag seat in back' are counted independently.
\item The following variations of `stand up gliding one foot' will be counted differently;
	\begin{itemize}
 	\item Arabesque (The free leg is extended behind the body above hip height - at least a 90 degree angle)
	\item Knee hold (one hand supporting the knee of the free leg)
	\item Y-character balance (holding a straightened leg up with one hand and using other hand to form a Y shape)
	\item Catch-foot (the free leg being held in one or both hands)
	\item Biellmann (the free leg grasped from behind and pulled overhead in the Biellmann position)
	\end{itemize}
\item Face up spins are different from normal upright spins
\item Combinations of one-rotation spins/turns are different from continuous spins
\end{itemize}

\textbf{Originality:} Skills with unique variations that are completely new or with new approaches will get more points.
Originality is mainly judged in Performance (section \ref{sec:freestyle_performance-score}).

\subsubsection{Pairs Freestyle:}
Number of skills should be counted for each rider separately.
If a rider is not riding a unicycle and performing non-unicycling skills while the other rider doing unicycling skills, only one skill for a rider is counted.

\textbf{Pairs Vs.\ Doubles:} `Doubles' refers to two riders on one unicycle.
In case of Doubles, the Quantity is counted as same as the skill by a single rider.

\subsubsection{Group Freestyle:}
Approximate number of skills may be counted for all members in total.
The number of skills should be weighted by the number of unicycling riders in the group.
If some riders are not on unicycles or are performing non-unicycling skills while the other riders doing unicycling skills, the count is reduced accordingly.

\subsection{Mastery And Quality of Execution}
\textbf{Mastery} is the amount of control shown by the rider(s) during their execution of the skills and transitions.
The body form should demonstrate good control and Mastery of the unicycle.
If a rider is showing good style during difficult skills, the Mastery score should be high.
Mastery of the unicycling skills is also required to perform the ``additional non-unicycling skills'', such as juggling, dancing, and acrobatics.

There are several viewpoints to check the Quality of Execution, such as Stability, Duration, Speed, Synchronization, and Fluidity of Transition.
These viewpoints don't have to be evenly weighted, but required to check.

\textbf{Duration:} Holding a skill for a longer amount of time and distance also indicates a higher level of mastery and difficulty for that skill.

\textbf{Stability:} High scores should not be given if unintentional jerky body movement, or a wandering spin or pirouette is shown occasionally.

\textbf{Speed:} High score is given when the rider controls the speed (faster or slower) of turns, spins, and transitions excellently.

\textbf{Synchronization:} Being synchronized with the rhythm of the music and timing accuracy should be judged.
High scores are awarded for a routine if timing of the skills is well planned and accurate.

\textbf{Fluidity of Transition:} High scores are given for transitions when the rider performs a skill straight into another skill quickly.
Low scores are given for transitions if several revolutions, idles, hops (or other setup-type skill) need to be performed before performing the more difficult skill - unless it is obvious that these are used to increase the overall choreography and timing of the routine.

\subsubsection{Pairs and Group Freestyle:}
\textbf{Synchronization:} Timing-synchronization with each other should be judged in Pairs and Group routines.
High scores awarded for a routine if timing of every skills are well planed and accurate.
Even though riders do not do the same skill/movement at the same timing intentionally, timing accuracy of each movement can be measured as synchronization with rhythm of the music, in a manner similar to individual routines.

\subsection{Difficulty And Duration}
The level of Difficulty is taken into account for successfully executed skills including transitions.
High scores are awarded for a routine packed with a number of skills all with high difficulty.
High scores should not be given if only one or two of the skills are of a high level.
Generally:
\begin{itemize}
\item Backward skills are more difficult than the same type of Forward skills.
\item `Seat against body' is easier than `Seat not touching body'.
\item Faster spins/turns with smaller diameter are more difficult than slower spins/turns with larger diameter.
\item `Stand up with a hand touching the seat' is easier than `stand up with neither hand touching the seat'.
\item `Jump up from the pedals to the frame removing both feet simultaneously' is more difficult than `Standup with one or both feet on the frame'.
\end{itemize}
Only `unicycling skills' will be judged; non-unicycling skills only affect Performance scores.
Dancing, juggling, and other non-unicycling skills can only affect the Performance score, and have no influence on this score.
However, if a rider is juggling while idling, for example, the difficulty of idling does not carry the same difficulty as idling without juggling.
The same applies for dancing and acrobatics.

\textbf{Duration:} Holding a skill for a longer amount of time and distance also indicates a higher level of mastery and difficulty for that skill.

\subsubsection{Pairs Freestyle:}
The Difficulty level of a multiple person act is determined by the overall level of difficulty displayed by the pair, not by the difficulty of feats presented by a single rider.
If one rider's skill level is a great deal higher than the other, judges must keep the Difficulty score somewhere between the levels of the two riders.
Number of skills should be counted for each rider separately.
If a rider is not riding a unicycle and performing non-unicycling skills while the other rider doing unicycling skills, only one skill for a rider is counted.
A skill in which the two riders obviously support each other will score lower than the same skill performed separately.
Judges must be able to distinguish between `support' and `artistic contact.' Riders who are merely holding hands may not be supporting each other, but if their arms are locked, they probably are.

\textbf{Note:} Some skills are more difficult with riders holding hands, such as one foot riding, side-by-side.

\textbf{Pairs Vs.\ Doubles:} `Doubles' refers to two riders on one unicycle.
In case of Doubles, the Quantity is counted as same as the skill by a single rider.
Some Pairs performers use lots of doubles moves, with lifting, strength, and the associated difficulty.
Other Pairs acts use no doubles moves at all.
How to compare them?
Remember that the skill level of both riders is being judged.
If the `top' rider does not display much unicycling skill when he or she rides, judges must keep that in mind, and rate their average difficulty accordingly.
If the top rider never rides, one can argue that this is not a Pairs act, and give a major points reduction.
Doubles moves are difficult for both persons, but must be weighed carefully against non-doubles performances.

Duration can be increased if a rider pulls or pushes another rider with holding hands, but will decrease the score as related to Quantity.

\subsubsection{Group Freestyle:}
As in Pairs, judges must seek to find the average Level of Difficulty of what may be a widely varied group of riders.
Top level skills done by only one rider cannot bring the Difficulty score up to top level.
High scores should not be given if only one or two of the skills are of a high level even if done by all riders or with skills that are the same type but with minor variations.
All riders in the routine must be used effectively.
This means that if one or more riders are at a beginner level, they can still ride around in circles, carry banners, be carried by other riders, etc.
Riders should not be left standing on the side.

\textbf{Small Group Vs.\ Large Group:} Some groups will be much smaller or larger than others, and judges must include this information in their decisions.
Large groups may have a tendency toward formation riding and patterns, while smaller groups may focus more on difficult skills.
With so many possibilities, judges must compare many different factors to get an adequate judgment.
Large numbers alone should not earn a high difficulty score, and neither should a few difficult skills performed by a small number.
The judges must consider the group's size as a part of the overall performance, including the advantages or limitations that size has on the types of skills being performed.

\section{Performance Score \label{sec:freestyle_performance-score}}
The Performance part of the judging is broken into three subcategories.
These subcategories are weighted as follows:

\begin{tabular}{l l l}
 & Presence/Execution & 33.33...\% \\
 & Composition/Choreography & 33.33...\% \\
 & Interpretation of the Music/Timing & 33.33...\% \\
\end{tabular}\\

Below you can find a definition of each subcategory as well criteria to be considered by the judges.

\subsection{Presence/Execution}
There are two parts to this section.
Each part does not need to be evenly weighted, but judges are required to consider both parts.

\textbf{Presence:} involvement of the rider physically, emotionally and intellectually as they translate the intent of the music and choreography.

\textbf{Execution:} quality of movement and precision in delivery.
This includes harmony of movement.

\textbf{Criteria:}
\begin{itemize}
\item Physical, emotional and intellectual presence
\item Carriage
\item Authenticity (individuality/personality)
\item Clarity of movement
\item Variety and contrast
\item Projection
\end{itemize}

\subsubsection{Pairs Freestyle:}
\textbf{Additional Criteria:}
\begin{itemize}
\item Unison and ``oneness''
\item Balance in performance between partners
\item Spatial awareness between partners
\end{itemize}

\subsubsection{Group Freestyle:}
\textbf{Additional Criteria:}
\begin{itemize}
\item Teamwork and cooperation
\item Use of all riders effectively
\end{itemize}

\subsection{Composition/Choreography}
An intentional, developed and/or original arrangement of all types of movements according to the principles of proportion, unity, space, pattern, structure and phrasing.

\textbf{Criteria:}
\begin{itemize}
\item Purpose (Movements, tricks, costume and music to match for a unified experienced)
\item Harmony (Interaction between tricks and body movements)
\item Utilization (Utilization of space and pattern usage for proper floor coverage)
\item Dynamics
\item Imaginativeness (Imaginative, originality and inventiveness of purpose, movement and design)
\end{itemize}

\subsubsection{Pairs Freestyle:}
\textbf{Additional Criteria:}
\begin{itemize}
\item Unity
\item Shared responsibility in achieving purpose by both
\end{itemize}

\subsubsection{Group Freestyle:}
\textbf{Additional Criteria:}
\begin{itemize}
\item Uniform coverage of the performance area
\end{itemize}

\subsection{Interpretation of the Music/Timing}
The personal and creative \emph{musical realization} of the rhythm, character and content of music to movement in the performance area.

\textbf{Criteria:}
\begin{itemize}
\item Continuity and musical representation (Effortless proper musical realization to artfully characterize the routine)
\item Expression of the music's style, character and rhythm
\item Use of \emph{finesse} to reflect the nuances of the music
\item Timing
\end{itemize}

\textbf{Musical realization} can occur in four different ways:
\begin{itemize}
\item Analog: Representing the highlights/cues in the music through movements
\item Congruent: Representing every beat/tone/note in the music through movements
\item Contrastive: Movement that is contrary to the music (fast movements to slow music or vice versa) or putting moves on the off-beat
\item Autonomy: Movement and music are independent.
Music and movements don't need to match, yet can.
This offers the artist the most freedom in his creative process
\end{itemize}

\textbf{Finesse} is the refined, artful manipulation of nuances by the rider(s).
Nuances are the personal artistic ways of bringing variations to the intensity, tempo, and dynamics of the music made by the composer and/or musicians.

\subsubsection{Pairs Freestyle:}
\textbf{Additional Criteria:}
\begin{itemize}
\item Relationship between the partners reflecting the character of the music
\end{itemize}

\section{Dismount Score \label{sec:freestyle_dismount-score}}
The Dismount Score is calculated based on the number of major and minor falls as outlined below.
Judges need to be able to differentiate between a planned dismount and an unplanned dismount.

\textbf{Major} dismounts are when the unicycle falls and/or a hand or any body part other than the rider's foot or feet touch the floor.
Major dismounts are also when the choreography of a rider's routine is clearly affected.

\textbf{Minor} dismounts are when the unicycle does not fall, only the rider's foot or feet touch down and the choreography of a rider's routine is not affected.
A minor dismount may also be counted when Judges cannot differentiate between a planned dismount and an unplanned dismount.

\textbf{Exception:}
Dismounts that occur while the rider is performing a seat drag skill have to be evaluated somewhat differently since the unicycle is already on the ground.
For these dismounts, the judges should use the current above language regarding minor and major dismounts but disregard the parts talking about the unicycle.
For example, if a rider is performing seat drag in back and steps off the unicycle with only their feet touching the ground, it would be considered a minor dismount unless the choreography of the routine is plainly affected.

\textbf{Individual and Pairs Freestyle Dismount Score Calculation:}

\[
\textrm{Dismount Score} = 10 - \textrm{\small{number of major dismounts}} - 0.5 \cdot \textrm{\small{number of minor dismounts}}
\]

\subsubsection{Group Freestyle:}
At Unicon a minimum of four dismount judges must be appointed by the Chief Judge to count falls for group routines.
At smaller competitions a minimum of two dismount judges should be appointed by the Chief Judge to count falls for group routines.
These judges should be chosen to fairly represent the different groups present (e.g.\ four judges from four different countries).
Neither Performance nor Technical judges should count falls for group freestyle.
Each dismount judge should count all of the major and minor dismounts of a group, not only dismounts in one part of the floor.
The counts from each judge should be averaged to create the dismount score for a group.

\textbf{Group Freestyle Dismount Score Calculation:}
The number of dismounts should be weighted by the number of riders in the group.
The following formula is used:

\[
\textrm{Mistake Score} = 1.0 \cdot \textrm{\small{number of major dismounts}} + 0.5 \cdot \textrm{\small{number of minor dismounts}}
\]
\[
\textrm{Final Dismount Score} = 10 - \frac{\textrm{mistake score}}{\sqrt{\textrm{number of riders}}}
\]

The Final Dismount Score cannot be lower than 0.
