\chapter{Competitor Rules}

\section{Safety}

Riders must wear shoes, no additional safety equipment is needed.

\section{Unicycles}

Any number of regular unicycles (see definitions in chapter \ref{chap:general_definitions}) may be used.

\section{Rider Identification}

No rider identification is required.

\section{Protests}

Protests must be filed on an official form within 15 minutes of the posting of event results.
Protest is only possible against mistakes in calculation or other mistakes not connected to a judge's subjective score.
The Chief Judge must resolve all protests within 30 minutes of receipt of the written form.

\begin{comment2016}
There probably needs to be some text about how protests apply to the preliminary round/battle format of the competition.
The following rule also applies to protests, but may not be accurate.
\end{comment2016}

\section{Results}
Final results will be continuously announced and/or posted for public view.
Results sheets will be posted after each category of an event.
The protest period begins at this point.

\section{Event Flow}

\subsection{Riders Must Be Ready}
The Chief Judge chooses how to handle the riders who are not ready at their scheduled competition time.
They may be disqualified or allowed to perform after the last competitor in their category.

\subsection{Preliminary Round}
Each rider's preliminary round is divided in two parts, \textit{timed preliminary} and \textit{last trick}.
Top scoring riders of the preliminary round will continue to the battle finals.

\subsubsection{Timed Preliminary}
The preliminary round will last one minute.
Any tricks completed after the one minute has elapsed will not be counted.
If the rider is in a combo when the time ends, they are not allowed to start another trick and extend the combo, they must end their combo with their current trick.
Once the time is up, the rider moves from \textit{timed preliminary} to \textit{last trick}.

\subsubsection{Last Trick (Preliminary)}
There are two attempts for the last trick in the preliminary round.
The rider is not obligated to use all attempts or to try the same trick every attempt.
Riders may skip an attempt.
Only the last attempt will be scored.
A failed attempt does not subtract from the score.

\subsection{Battles}
In a Flatland battle, two riders compete head-to-head, taking turns performing tricks.
Battles are separated into two parts: \textit{timed battle} and \textit{last trick}.
The winner of each battle is determined immediately after the battle by the judges.
The winner continues to the next battle and the loser is eliminated, unless the battle is in a double-elimination bracket.

\subsubsection{Timed Battle}
Each battle will last two minutes, except for the final four battles.
These semifinal and final battles will last two to four minutes as agreed upon by both battling riders.
If they disagree, these battles will default to three minutes in length.
The rider with better ranking from the preliminary round chooses which rider starts the battle.
There are two countdown timers, one for each rider (each with half the battle duration).
The corresponding rider's timer will be started and stopped when they start and stop riding.
Any tricks completed after the rider's time has elapsed will not be counted.
If the rider is in a combo when the time ends, they are not allowed to start another trick and extend the combo, they must end their combo with their current trick.
After one rider's time runs out, the other rider will ride for their remaining time.
Once the time is up for both riders, the riders go from \textit{timed battle} to \textit{last trick}.

\subsubsection{Last Trick (Battles)}
There are three last trick attempts for each rider in battles.
The rider who started the battle starts the last trick.
Riders will take turns attempting their last trick.
Riders are not obligated to use all attempts or to try the same trick every attempt.
Riders may skip an attempt.
Only the last attempt will be scored.
Other failed attempts do not subtract from the score.

\subsection{Number of Competitors Entering Battles}
The highest scoring competitors from the preliminary round proceed onto the final battles.
The number of competitors that move onto the finals is determined by a vote from the judges, but it cannot be more than 16 riders.
Only a simple majority is needed for the vote.
If a number of other 4, 8 or 16 is chosen, byes are used to expand the group of rider to the next largest bracket.
(For example, 11 riders would use the 16 rider bracket, and the top 5 riders would get a bye for the first round of battles.)

\subsubsection{Battle Assignments}
Battles will proceed according to the following brackets.
The use of the double elimination bracket is optional.

http://www.printyourbrackets.com/pdfbrackets/4teamDouble.pdf \\
http://www.printyourbrackets.com/pdfbrackets/8teamDouble.pdf \\
http://www.printyourbrackets.com/pdfbrackets/16teamdouble.pdf
