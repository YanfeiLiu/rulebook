\chapter{Judges and Officials Rules}

\section{Flatland Director}

The Flatland Director is the head organizer and administrator of Flatland.
With the convention host, they are responsible for the event logistics, the equipment and the system used to run the event.
They must select the Chief Judge.
They are in charge of keeping the event on schedule and answering questions about the event.
The Flatland Director is the highest authority on everything to do with the Flatland competition, except for decisions on rules and results.

\section{Chief Judge}

The Chief Judge is the head Flatland official.
They are responsible for selecting the judges, overseeing the competition, handling protests, ensuring the event rules are followed, and answering questions about the rules and judging.
The Chief Judge is also responsible for the accuracy of all judging point tabulations and calculations.

An interruption of judging can result from material damage, injury of a competitor, or interference with a competitor by a person or object.
The Chief Judge determines if the rider is at fault.
If they are not, the Chief Judge chooses when to insert the rider back into the competition, along with the rider's remaining time.
The Chief Judge may be the same person as the Flatland director.

\section{Judges}

\subsection{Judging Panel}

There must always be an odd number of judges to prevent ties.

\subsection{Selecting Judges}
A person should not judge an event if he or she is:
\begin{itemize}
\item A parent, child or sibling of a rider competing in the event.
\item A coach, manager, trainer or colleague of the same club/team as a rider competing in the event.
\item More than one judge from the same country on the same judging panel.
\end{itemize}

If the judging pool is too limited by the above criteria, restrictions can be eliminated starting from the bottom of the list and working upward as necessary, but only until enough judges are available.

\subsection{Judging Panel May Not Change}

The individual members of the judging panel must remain the same for an entire category.
In the event of an emergency, this rule can be waived by the Chief Judge.

\subsubsection{Rating Judge Performance}
Judges are rated by comparing their scores to those of other judges at previous competitions.
If a judge's performance is determined to be too weak, they may be removed from the judging panel.

Characteristics of Judging Weaknesses:
\begin{itemize}
\item \textbf{Excessive Ties:}
Using ties frequently (it defeats the purpose of judging.)
\item \textbf{Bias:}
Placing members of certain groups or nations significantly lower or higher than other judges.
\item\textbf{Inconsistence:}
Ranking a large number of riders significantly different from the average of other judges.
\end{itemize}

\subsection{Training}
The judges workshop is set by the Flatland Director or Chief Judge.
Either the Flatland Director or the Chief Judge run the workshop.
The workshop must be help before the competition.
Judges should have read the rules prior to the start of the workshop.
The workshop will include a practice session.
Each judge will read the rules, attend the workshop, agree to follow the rules and agree to their potential removal from the list of available judges if they show excessive judges weaknesses, as determined by the Chief Judge.

\section{Flatland Judging and Scoring}

\subsection{Judging Criteria}

Preliminary rounds and battles are judged using the following criteria.
Each category is scored out of 10 points, and weight differently

\textbf{Difficulty} (25\% of score): \\*
Score is given for technical difficulty of the tricks and combos landed during the battle/preliminary.

\textbf{Consistency} (23\% of score): \\*
Score is given for number of landed trick/combos on total of number of tricks/combos attempted during the battle/preliminary.

\textbf{Flow} (20\% of score): \\*
Score is given for cleanliness and style of rider during the battle/preliminary.

\textbf{Variety} (18\% of score): \\*
Score is given for variation in the types of tricks done during the battle/preliminary.

\textbf{Last Trick} (14\% of score): \\*
Score is given for technical difficulty, novelty, creativity, and flow.\\*
The rider is not obligated to use all attempts or to try the same trick every attempt.
Only the last attempt will be scored.
Other failed attempts do not subtract from the score.

Guide on how to score points for last trick:
\begin{itemize}
\item 0 point: nothing landed or unworthy trick
\item 1 point: passable trick
\item 2 point: okay trick
\item 3 points: good trick
\item 4 points: great trick
\item 5 points: insane trick
\end{itemize}

\subsection{Preliminary Round Scoring}
At the end of every preliminary run, the judges must enter or write down the rider's scores.
Once all preliminary runs are over, the scores for each rider are tabulated by adding up the scores from each judges and then the riders are ranked according to their total number of points.

If there are two riders with equal points, the riders' ``last trick'' score is used to break the tie.
If the riders' ``last trick'' scores are equal and the riders are moving on to battles, the judges choose the order of placement of the two riders by a simple vote of majority.

\subsection{Battle Scoring}
For battles, judges must decide on a single rider to vote on, they cannot tie the riders.
Judges are not required to write down scores for each category during battles.
Judges must determine a winner individually.
The Chief Judge collects the results from each judge and the winner is chosen by simple majority.
The winner of each battle is then announced directly.

\subsection{Sportsmanship}
If a rider distracts or delays the competition or shows unsportsmanlike conduct, the Chief Judge may choose to warn or eliminate that rider.

\subsection{Finals/Semi-Finals}
The winner and loser of the final battle round take first and second place in the competition.
The losers of the semi-final battle round compete in the ``small final'' battle for third place and fourth place.
The small final is the only required battle from the ``losers bracket.''
