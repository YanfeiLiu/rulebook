\chapter{Competitor Rules}

\section{Safety}

All riders must wear a helmet and shoes as defined in chapter \ref{chap:general_definitions}.
Gloves, shin guards and knee protection are recommended.

\section{Unicycles}

Any unicycle may be used.
There is no restriction on changing unicycles during the competition.

\section{Rider Identification}

The rider number must be visible on the rider or unicycle.

\section{Event Flow}

\subsection{Start}

Riders must begin on the start pallet, mounted.
It is acceptable for riders to hop in place to balance however riders will be disqualified if the tire contact point crosses the front of the pallet before the signal.
The start signal will be given by the Starter by whistle to begin the race.
Line judges and the starter are responsible for identifying false starts.

\subsection{Race}

Once riders have crossed the start line, they must traverse the course without being disqualified.
If a rider's unicycle makes contact with the ground while traversing the course, they can return to the last obstacle successfully completed or at an earlier obstacle in the course.

For example, if a rider loses balance on a skinny and jumps to the ground (while remaining mounted), they must go back to the obstacle touched before the skinny.

\subsection{Disqualification}

Disqualification is instant and permanent for the attempt or heat.
A disqualified rider must stop and clear the course without interfering with judges or other riders.
Disqualification happens as follows:
\begin{enumerate}
\item False start: Starting before the signal has been given by the Starter.
\item Allowing any part of the rider's body to touch the ground or obstacle.
If loose clothing brushes against the ground or obstacle but does not influence the rider's balance, then this is acceptable.
\item Allowing any part of the unicycle except the tire, rim, spokes, crank arms, pedals, or bearing caps to touch the ground or obstacle.
\item Yelling or verbally distracting the opponents or judges.
\end{enumerate}

\subsection{Finish}

A rider finishes the race once their wheel fully crosses over the finish line.
Once a rider has traversed the last obstacle of the course, they may touch the ground until they cross the finish line.
Timekeepers and Finishers will determine the results.

\section{Preliminary Rounds}

Competitors have 2 or 3 individual attempts to complete the course.
Determining if there should be 2 or 3 attempts is decided by the Trials Director.
If the number of competitors is less than 10, the results from the preliminary round will be the final results.

\section{Finals}

Finals can be run in three different ways depending on the number of competitors.
At Unicon, the trials director must follow the format indicated in this section depending on number of competitors.
For smaller events where materials and construction time is reduced, the trials director may choose from the formats without having to determine event by number of competitors.

\subsection{Four Rider Heats}

If the number of competitors is 30 or more, finals must be composed of the fastest 16 times from the preliminary round and will be organized with 8 heats of 4 riders.
4 identical lines will need to be constructed.
For four rider heats, the first two riders to cross the finish line move forward and the last two riders are eliminated.
The heats will be organized as follows:

\textbf{Quarter Finals:}\\
Heat 1: Seed 1 versus Seed 5 versus Seed 9 versus Seed 13\\
Heat 2: Seed 3 versus Seed 7 versus Seed 11 versus Seed 15\\
Heat 3: Seed 2 versus Seed 6 versus Seed 10 versus Seed 14\\
Heat 4: Seed 4 versus Seed 8 versus Seed 12 versus Seed 16

\textbf{Semi Finals:}\\
Heat 5: Heat 1 1st, Heat 1 2nd, Heat 2 1st, Heat 2 2nd\\
Heat 6: Heat 3 1st, Heat 3 2nd, Heat 4 1st, Heat 4 2nd

\textbf{5th-8th Place Round:}\\
Heat 7: Heat 5 3rd, Heat 5 4th, Heat 6 3rd, Heat 6 4th

\textbf{Finals:}\\
Heat 8: Heat 5 1st, Heat 5 2nd, Heat 6 1st, Heat 6 2nd

\subsection{Two Rider Heats}

If the number of competitors is between 10 and 30, finals will be composed of the fastest 8 times form the preliminary round and will be organized by 8 heats of 2 riders.
2 identical lines will need to be constructed.
For two rider heats, the first rider to cross the finish line moves forward and the last rider is eliminated.

\textbf{Quarter Finals:}\\
Heat 1: Seed 1 versus Seed 5\\
Heat 2: Seed 3 versus Seed 7\\
Heat 3: Seed 2 versus Seed 6\\
Heat 4: Seed 4 versus Seed 8

\textbf{Semi Finals:}\\
Heat 5: Heat 1 Winner versus Heat 2 Winner\\
Heat 6: Heat 3 Winner versus Heat 4 Winner

\textbf{3rd-4th Place Round:}\\
 Heat 7: Heat 5 Loser versus Heat 6 Loser

\textbf{Finals:}\\
Heat 8: Heat 5 Winner versus Heat 6 Winner

\subsection{Prelim Results as Final Results}

If the numbers of competitors is less than 10, prelim results will be the final results (no finals necessary).

\section{Prohibited Activities}
No rider may attempt any obstacle prior to the start of the competition.
Intentional modification of a section by riders or spectators is prohibited.
Note that kicking objects to test stability does not constitute intentional modification if an object moves.
If the course is unintentionally modified or broken by a rider, they should inform the Event Director or Course Setter who will return the obstacle to its original form if possible.
