\chapter{Judges and Officials Rules}

\section{Speed Trials Officials}

\subsection{Speed Trials Director}

The Speed Trials Director is the head organizer and administrator of speed trials events.
With the Convention Host, the Speed Trials Director determines the course, obtains permis, interfaces with the community, and determines the system used to run the event.
The Speed Trials Director is responsible for the logistics and equipment for all trials events.
Speed Trials Director is in charge of keeping events running on schedule, and answers all questions not pertaining to rules and judging.
The Speed Trials Director is the highest authority on everything to do with the trials events, except for decisions on rules and results.

\subsection{Chief Judge}

The Chief Judge is the head Speed Trials official, whose primary job is to make sure the competitors follow the rules.
The Chief Judge makes all final decisions regarding rule infractions.
The Chief Judge is responsible for resolving protests.

\subsection{Starter}
The starter is responsible for giving each competitor a fair and equal start.
There must be only one starter for an event to maintain consistency.
The starter must use a whistle to signal the start of the race.
The starter must make sure all riders are ready, alert and given the signal to begin the race.
The starter must check for false starts.
The starter must begin the race with random delays between 1 and 10 seconds.
The starter must be positioned close to the start line as possible but must not interfere or impede riders.

\subsection{Line Judge}

Each rider must have a separate line judge assigned to them.
The line judges are responsible for checking for false starts.
As riders progress over the course, the line judge must make sure that their assigned rider is successfully cleaning the course.
Should the rider fall or dismount, the line judge must indicate the location where the rider last made contact before dismounting.

\subsection{Timekeepers/Finisher}
Timekeepers are responsible for keeping time during Prelims.
Timekeepers are unnecessary during finals since finals are heats.
During prelims, the timekeepers must press their stop watches when the races commences and when the rider crosses the finish line.
A minimum of two timekeepers must be used to record time.
The average of the two times must be calculated to determined the final score for the attempt.
During finals, two Finishers will watch the finish line to determine order of riders.
Time keepers must be positioned close to the finish line for accurate results.

\section{Rematch}

At the discretion of the Chief Judge, a rematch for a heat (finals) may be needed if the winner is unclear.
The riders in question will redo the heat without other competitors.For example, if the 2nd and 3rd place rankings are unclear, a new heat will begin with only 2nd and 3rd for tiebreak.

Rematches will not be provided for broken parts or flat tyres once the competitor in question has entered the course.
